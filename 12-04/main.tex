\documentclass[a4paper,12pt]{article}

\usepackage[french,italian]{babel}
\usepackage[T1]{fontenc}
\usepackage[utf8]{inputenc}
\frenchspacing 
\title{Appunti Fisica I}
\author{Luca Seggiani}
\date{12 Aprile 2024}

\usepackage{amsmath}

\begin{document}
\maketitle
\textbf{Urti elastici monodimensionali} \\
Calcoliamo i valori delle velocità di due corpi a seguito di un urto monodimensionale. Impostiamo la conservazione della quantità
di moto:
$$ \vec{P_i} = \vec{P_f} = m_1v_{1i} + m_2v_{2f} = m_1v_{1f} + m_2v_{2f} $$
$$ -\Delta\vec{P_1} = \Delta\vec{P_2} = m_1(v_{1i} - v_{1f}) = m_2(v_{2f} - v_{2i}) $$
e la conservazione dell'energia:
$$ \frac{1}{2}m_1v_{1i}^2 + \frac{1}{2}m_2v_{2i}^2 = \frac{1}{2}m_1v_{1f}^2 + \frac{1}{2}m_2v_{2f}^2 $$
con alcuni raccoglimenti, si ha che:
$$ \frac{1}{2}m_1(v_{1i}^2 - v_{1f}^2) = \frac{1}{2}m_2(v_{2f}^2 - v_{2i}^2) $$
che è molto simile alla conservazione della quantità di moto. Riscriviamo infatti come:
$$ \frac{1}{2}m_1(v_{1i} - v_{1f})(v_{1i} + v_{1f}) = \frac{1}{2}m_2(v_{2f} - v_{2i})(v_{2f} + v_{2i})$$
L'uguaglianza dei termini di questa equazione che presentano differenze di velocità è assicurata dalla conservazione
dell'energia. Resta quindi soltanto:
$$ v_{1i} + v_{1f} = v_{2i} + v_{2f} \Rightarrow v_{1i} - v_{2i} = v_{2f} - v_{1f} $$
ovvero le velocità relative dei due oggetti si conservano dopo l'urto. Abbiamo quindi che:
$$ 
\left\{\begin{array}{l}
  v_{1i} - v_{2i} = v_{2f} - v_{1f} \cr \\
  m_1(v_{1i} - v_{1f}) = m_2(v_{2f} - v_{2i}) \cr
\end{array}\right.
$$
Da questo sistema possiamo ricavare algebricamente le velocità $v_{1f}$ e $v_{2f}$:
$$ v_{1f} = \frac{m_1-m_2}{m_1+m_2}v_{1i} + \frac{2m_2}{m_1+m_2}v_{2i} $$
$$ v_{2f} = \frac{m_2-m_1}{m_1+m_2}v_{2i} + \frac{2m_1}{m_1+m_2}v_{1i} $$

\textbf{Urti perfettamente anelastici monodimensionali} \\
Gli urti elastici sono più complessi da modellizzare, in quanto disponiamo solamente di un'equazione per
risolvere il problema, la conservazione della quantità di moto ($P$). Nell'urto perfettamente anelastico
si ha però che dopo l'urto i due corpi procedono assieme attaccati l'uno all'altro, con la stessa velocità $v_f$.
Avremo quindi che:
$$ \vec{P} = m_1v_{1i} + m_2v_{2i} = (m_1 + m_2)v_f, \quad v_f = \frac{m_1v_{1i} + m_2v_{2i}}{m_1+m_2}$$
dove $v_f$ era l'unica incognita del sistema. Possiamo allora considerare l'energia cinetica:
$$ K_f - K_i = \frac{1}{2}(m_1 + m_2) v_f^2 - \frac{1}{2}m_1v_{1i}^2 - \frac{1}{2}m_2v_{2i}^2 = -\frac{1}{2}\frac{m_1m_2}{m_1+m_2}(v_{1i} - v_{2i})^2<0 $$
ovvero notiamo che a seguito del processo di urto l'energia cinetica del sistema è diminuita (è stata "usata" nella deformazione
degli oggetti in collisione, oppure trasformata in suono, energia termica, ecc...).
\par\smallskip
\textbf{Pendolo balistico} \\
Vediamo un esempio: un pendolo balistico usato per rilevare la velocità dei proiettili. Il pendolo è costituito da un blocco sospeso
da due funi. Il blocco, inizialmente fermo, una volta colpito dal proiettile (di urto completamente anelastico, il proiettile rimane infatti conficcato nel blocco),
sale di una certa distanza verso l'alto, sottoposto alla spinta infertagli. Le condizioni iniziali sono quindi
$v_{1i} > 0, \quad v_{2i} = 0$. Il sistema è soggetto sia a forze impulsive
che si esplicano durante l'urto, sia alla forza peso, che però non consideriamo durante l'urto. Abbiamo innanzitutto
che la velocità finale del pendolo sarà:
$$ v_f = \frac{m_1v_{1i}}{m_1+m_2} $$
Notiamo adesso che dopo l'urto, l'energia si conserva: tutte le forze non conservative agiscono solamente nell'istante
dell'urto. Posso quindi applicare la conservazione dell'energia:
$$ \frac{1}{2}(m_1+m_2)v_f^2 = (m_1+m_2)gh, \quad v_f = \sqrt{2gh} $$
Se diversamente alle nostre ipotesi l'urto fosse stato elastico, oltre ad una situazione particolarmente pericolosa, avremo
che l'energia cinetica si conserverebbe. Possiamo quindi impostare:
$$ 
\left\{\begin{array}{l}
    v_{1f} = \frac{m_1-m_2}{m_1+m_2}v_{1i} \cr \\
    v_{2f} = \frac{2m_1}{m_1+m_2}v_{1i}
\end{array}\right.
$$

\section{Moto di un corpo rigido}
Per corpo rigido si intende un sistema di punti materiali le cui distanze l'uno dall'altro non variano nel tempo. In
altre parole, un corpo rigido non subisce deformazioni.
Ricordiamo che il centro di massa di un corpo rigido è dato da:
$$ \vec{r_{CM}} = \frac{1}{M}\int\vec{r}dm = \frac{1}{m}\vec{r}\rho dV = \frac{1}{V}\int \vec{r} dV $$
Si rimanda per le altre formule agli appunti relativi al centro di massa.
\par\smallskip
\textbf{Moto di pura traslazione} \\
In un moto di pura traslazione, tutte le particelle del corporigido subiscono lo stesso spostamento nello stesso intervallo
di tempo, ergo hanno la stessa velocità (che corrisponderà obbligatoriamente con la velocità del CM). La velocità dei vari
punti rispetto al centro di massa sarà allora nulla.
$$ \vec{v_{CM}} = \frac{d\vec{r_{CM}}}{dt},\quad \vec{v_{CM}} = \frac{\int \vec{v} dm}{M} = \vec{v}{\int dm} {M} = \vec{v}, \quad \vec{P_{CM}} = M\vec{v_{CM}} = M\vec{v} $$
Se il corpo non è di pura traslazione vale:
$$ \vec{v_{CM}} = \frac{\int \vec{v}(x,y,z)dm}{M}, \quad \vec{P_{CM}} = M\vec{v_{CM}} = M\frac{\int \vec{v}(x,y,z)dm}{M} $$
Vediamo quindi l'accelerazione:
$$ \vec{a_{CM}} = \frac{\int \vec{a}dm}{M} = \vec{a} \frac{\int dm}{M} = \vec{a}, \quad M\vec{a_{CM}} = M\vec{a}$$
ovvero in ogni punto del corpo l'accelerazione è la stessa. Come prima, non fosse stato il moto di pura traslazione, vale:
$$ \vec{a_{CM}} = \frac{\int \vec{a}(x,y,z)dm}{M}, \quad M\vec{a_{CM}} = \int \vec{a}(x,y,z)dm $$
\par\smallskip
\textbf{Prima equazione cardinale per un corpo rigido} \\
In modo analogo a quanto detto sui sistemi, su un corpo rigido si può dire che la derivata della quantità di moto
totale è uguale alla risultante delle sole forze esterne:
$$ \frac{d\vec{P_{CM}}}{dt} = M\vec{a_{CM}} = \sum \vec{R}^{est} $$
\par\smallskip
\textbf{Secondo teorema del CM per un corpo rigido} \\
Il centro di massa si muove come un punto di massa totale $M$ soggetto alla sola risultante delle forze esterne al 
sistema. Questo ci permette di enunciare la conservazione della quantità di moto: quando la risultante delle forze esterne e nulla,
la quantità di moto del corpo rigido resta costante in modulo, direzione e verso.
\par\smallskip
\textbf{Lavoro delle forze interne} \\
Essendo il lavoro dipendente dalla variazione di distanza (spostamento), avremo che il lavoro totale svolto dalle forze interne,
o, in simboli:
$$ dL_{ij} = \vec{F_{ij}} \cdot d\vec{r_i} + \vec{F_{ji}} \cdot d\vec{r_j} = \vec{F_{ij}} \cdot d\vec{r_i} - \vec{F_{ij}} \cdot d\vec{r_j} = \vec{F_{ij}} \cdot (d\vec{r_i} - d\vec{r_j}) = 0 $$
Ciò deriva dal fatto che, per definizione di corpo rigido, le variazioni di posizione fra i punti materiali che compongono il sistema e il centro di massa è nulla,
e lo è quindi anche fra i punti materiali stessi.
Questo ci permette di affermare che in un corpo rigido solo le forze esterne determinano la variazione della quantità di moto,
ovvero la prima equazione cardinale. Ad esempio, prendiamo la forza di gravità:
$$ M\vec{a_{CM}} = \int \vec{a} dm = \int \vec{g} dm = Mg $$
\end{document}
