\documentclass[a4paper,12pt]{article}

\usepackage[french,italian]{babel}
\usepackage[T1]{fontenc}
\usepackage[utf8]{inputenc}
\frenchspacing 
\title{Appunti Fisica I}
\author{Luca Seggiani}
\date{17 Maggio 2024}

\begin{document}
\maketitle
\section{Introduzione alla magnetostatica}
Studiando l'elettrostatica, abbiamo dato una definizione di \textbf{campo elettrico}. Vediamo adesso un campo
collegato al campo elettrico, il \textbf{campo magnetico}. Il campo magnetico viene generato dai \textbf{magneti},
che possono essere magneti permanenti, oppure elettromagneti (che saranno visti nel dettaglio in seguito). 
E' fondamentale notare che non esiste un equivalente magnetico della carica elettrica: in altre parole, 
non esistono \textbf{monopoli} magnetici, e i magneti si presentano sempre come dipoli con un polo nord e un polo sud.
N.B.: Il campo magnetico è un costrutto dell'elettrodinamica classica. In realtà, campo magnetico e campo elettrico
sono manifestazioni dello stesso fenomeno attraverso punti di vista diversi. Il campo magnetico è un'effetto relativistico
del campo elettrico, e viceversa.
\par\smallskip
\textbf{Forza di Lorentz} \\
Il campo magnetico influenza le cariche elettriche, proprio come faceva il campo elettrico. La forza applicata dal
campo magnetico $\vec{B}$ ad una carica $q$ in moto con velocità $\vec{\mathrm{v}}$ è:
$$ \vec{F} = q\vec{\mathrm{v}} \times \vec{B}$$
Si osserva che la forza applicata dal campo elettrico dipende sia dalla carica che dalla sua velocità. Il campo magnetico
$\vec{B}$ si misura in Tesla ($1\mathrm{T} = 1\frac{\mathrm{N}}{\mathrm{A}\cdot\mathrm{m}}$). Notiamo che, visto
che il prodotto vettoriale restituisce un vettore perpendicolare alla velocità della particella (e al campo magnetico),
la forza di Lorentz non varia il \textit{modulo} della sua velocità. In altre parole, il lavoro della forza di Lorentz è nullo. \\
Possiamo conciliare la formula della forza di Lorentz col modello di Drude-Lorentz. Iniziamo con l'esprimere la forza
applicata su una carica $q$ da campo elettrico e campo magnetico (la cui formula sarebbe quella che prende effettivamente
il nome di \textit{forza di Lorentz}, bensì solitamente ci si riferisce con questo nome solamente alla parte magnetica):
$$ \vec{F} = q(\vec{E} + \vec{\mathrm{v}} \times \vec{B}) $$
Possiamo da questo ricavare l'accelerazione che una particella immersa in un certo campo elettromagnetico subisce:
$$ \frac{d\vec{\mathrm{v}}}{dt} = \vec{a} = \frac{q}{m}(\vec{E} + \vec{\mathrm{v}} \times \vec{B}) $$
Possiamo quindi considerare (come avevamo già fatto nello studio della conducibilità attraverso Drude-Lorentz), due velocità medie
$<\vec{\mathrm{v}}_i>$ e $<\vec{\mathrm{v}}_{i + 1}>$, corrispondenti rispettivamente all'attimo appena prima e appena dopo un'urto
col reticolo metallico o con un'altra particella. Come sappiamo, quanto accade fra un'urto e un altro è effettivamente irrilevante
in quanto le particelle continuano a muoversi di moto rettilineo uniforme, e la loro quantità di moto quindi non cambia. Si ha allora,
stabilito un cammino libero medio $\tau$:
$$ <\vec{\mathrm{v}_{i + 1}}>\, =\, < \vec{\mathrm{v}}_i + \frac{q\tau}{m}(\vec{E} + \vec{\mathrm{v}_i} \times \vec{B}) > \, =  \frac{q\tau}{m}(\vec{E} + \vec{\mathrm{v}_i} \times \vec{B}) $$
come velocità media (potremo forse assimilarla alla velocità di deriva) di una particella. Dividiamo nelle due componenti:
$$ < \vec{\mathrm{v}} > = \frac{q\tau}{m}\vec{E} + \frac{q\tau}{m}(\vec{\mathrm{v}} \times \vec{B}) $$
Prendiamo come approssimazione che l'effetto $ \frac{q\tau}{m}(\vec{\mathrm{v}} \times \vec{B}) $ del campo magnetico
sia molto minore dell'effetto $ \frac{q\tau}{m}\vec{E} $ del campo elettrico. Potremo allora trovare un'approssimazione di $\vec{\mathrm{v}}$:
$$ \vec{\mathrm{v}} \approx   \frac{q\tau}{m}\vec{E} $$
Possiamo quindi risostituire nella formula per la forza di Lorentz:
$$ \vec{F} = q(\vec{E} + \vec{\mathrm{v}} \times \vec{B}) =  \vec{F} = q(\vec{E} +  \frac{q\tau}{m}\vec{E} \times \vec{B}) $$
Questo ci suggerisce che l'applicazione della forza di Lorentz al modello di Drude-Lorentz fornisce a livello microscopico risultati
consistenti con quelli che osserviamo a livello macroscopico. Sempre attraverso Drude-Lorentz, abbiamo, riguardo ad una certa densità di corrente $J = nqv_d$, e \textit{solo riguardo al campo magnetico}:
$$ d\vec{F} = nAdl\cdot qv_d \times \vec{B} = J \cdot A \cdot dl \times \vec{B} = I \cdot dl \times \vec{B} $$
Su un certo spostamento di cui $dl$ è elemento infintesimo. Passando ad integrale, si ha:
$$ \vec{F} = \int_{l_1}^{l_f} I \cdot dl \times \vec{B} = I\vec{l} \times \vec{B} $$
\end{document}
