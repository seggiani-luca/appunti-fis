\documentclass[a4paper,12pt]{article}

\usepackage[french,italian]{babel}
\usepackage[T1]{fontenc}
\usepackage[utf8]{inputenc}
\frenchspacing 
\title{Appunti Fisica I}
\author{Luca Seggiani}
\date{8 Aprile 2024}

\begin{document}
\maketitle
\section{Quantità di moto}
La quantità di moto è una grandezza vettoriale definita come:
$$ \vec{P} = m\vec{v} $$
dove $m$ è la massa di un corpo e $v$ la sua velocità in un dato istante. Si misura in 
kg$\cdot$m/s.
\par\smallskip
\textbf{Quantità di moto e leggi di Newton} \\
La quantità di moto ci permette di riformulare il primo principio della dinamica, riguardante l'inerzia, specificando che
la quantità di un punto materiale isolato resta costante: la sua massa non varia, e per inerzia nemmeno la sua velocità.
$$ \Delta \vec{p} = \vec{p}(f) - \vec{p}(i) = \int_i^f \vec{F} = 0 \Rightarrow \vec{p} = \mathrm{const.} $$
Se invece una forza (e quindi un'accelerazione) agisce sul corpo cambiandone la velocità, la quantità di moto varierà come:
$$ \frac{d\vec{p}}{dt} = \frac{d(m\vec{v})}{dt} = m\vec{a} = \vec{F} $$
\section{Impulso di una forza}
L'impulso di una forza è definito, fra due istanti $t_0$ e $t$, come:
$$ \vec{I}_{t_0\rightarrow t} = \int_{t_0}^t = \vec{F}dt' $$
L'impulso è una grandezza vettoriale espressa in kg$\cdot$m/s, la stessa unità di misura della quantità di
moto. Si può quindi dimostrare:
\par\smallskip
\textbf{Teorema dell'impulso} \\
La variazione della quantità di moto $\Delta \vec{p}$ di un punto materiale è pari all'impulso totale delle forze esterne:
$$ \Delta \vec{p} = \Delta \vec{p}(t) - \Delta \vec{p}(t_0) = \int_{t_0}^t \vec{F} dt'$$
Questo si può dimostrare da:
$$ \int_{t_0}^t \vec{F}dt' = \int_{t_0}^t m\vec{a}dt' = [m\vec{v}]_{t_0}^t = m\vec{v}(t) - m\vec{v}(t_0) = \vec{p}(t) - \vec{p}(t_0) = \Delta \vec{p}$$
\section{Moto di un corpo composto}
Un corpo composto non'è altro che un corpo formato da un insieme di punti materiali. Quando si considerano sistemi
formati da più punti materiali connessi fra loro, occorre fare una distinzione fra:
\begin{itemize}
  \item \textbf{Forze interne}: forze che sono interne al sistema, esercitate fra i corpi appartenenti al sistema fra di loro;
  \item \textbf{Forze esterne} forze che sono esterne al sistema. esercitate dall'ambiente sui punti materiali.
\end{itemize}
\par\smallskip
\textbf{Dinamica di un sistema di punti materiali} \\
Per ciauscino dei punti materiali di un sistema possiamo scrivere:
$$ m\vec{a_i} = \vec{R_i} $$
Dove $\vec{R_i}$ è la risultante di tutte le forze sul punto materiale, interne ed esterne.
$$ \sum m_i\vec{a_i} = \sum R^{est}_i+ \sum_{j \neq i} \vec{F_{ij}} = \sum R^{est}_i + \sum \sum_{j \neq i} \vec{F_{ij}} $$
dove tutte le sommatorie da senza pedice sono da 1 a $n$ numero di punti materiali nel sistema.
\par\smallskip
\textbf{Centro di massa di un sistema} \\
Definiamo come centro di massa di un sistema il "punto medio" della massa:
$$ \vec{r_{CM}} = \frac{\sum m_i\vec{r_i}}{\sum m_i} $$
Definendo la massa del sistema $M$ come: $\sum m_i$ possiamo riscrivere come:
$$ \vec{r_{CM}} = \frac{\sum m_i\vec{r_i}}{M} $$
Le coordinate del punto di massa su $x,y$ e $z$ saranno a questo punto:
$$ \vec{x_{CM}} = \frac{\sum m_i\vec{x_i}}{\sum m_i}, \quad \vec{y_{CM}} = \frac{\sum m_i\vec{y_i}}{\sum m_i}, \quad \vec{z_{CM}} = \frac{\sum m_i\vec{z_i}}{\sum m_i} $$
Si può definire l'impulso del centro di massa $\vec{P_{CM}}$ come l'impulso sulla massa $M$:
$$ \vec{P_{CM}} = M\vec{v_{CM}} = \sum m_i\vec{v_i}$$
Dove dovremo definire la velocità del centro di massa $\vec{V_{CM}}$:
$$ \vec{V_{CM}} =  \frac{d\vec{r_{CM}}}{dt} = \frac{d}{dt} \frac{\sum m_i\vec{r_i}}{M} = \frac{1}{m}\frac{d}{dt}\sum m_i\vec{r_i} = \frac{1}{M} \sum m\vec{v_i} $$
Questo ci permette di dire, riguardo all'impulso del centro di massa:
$$ \vec{P_{CM}} = \sum \vec{p_i} = \sum m_i\vec{a_i} = M\vec{V_{CM}}$$
\par\smallskip
\textbf{Primo teorema del centro di massa} \\
La quantità di moto totale di un sistema (ergo la somma delle quantità di moto di ogni punto del sistema) è uguale
alla quantità di un punto materiala avente massa uguale alla massa del sistema e velocità uguale alla velocità del centro di
massa del sistema:
$$ \vec{P_{CM}} = M\vec{V_{CM}} = \sum m_i\vec{a_i} $$
\par\smallskip
\textbf{Forze interne di sistemi di punti materiali} \\
Se prendiamo un sistema formato da più punti materiali soggetti sia a forze interne (fra di loro), sia a forze esterne
(esercitate da altri corpi sul sistema), abbiamo dalle equazioni precedenti:
$$ \vec{P_{CM}} = M\vec{V_{CM}} \Rightarrow \sum m_i\vec{a_i} = M\vec{a_{CM}} $$
$$ \sum m_i\vec{a_i} = \sum R^{est}_i + \sum \sum_{j \neq i} \vec{F_{ij}} $$
Se prendiamo il termine $\sum R^{est}_i + \sum \sum_{j \neq i} \vec{F_{ij}}$, abbiamo che la sommatoria
$\vec{F}_{12} + \vec{F}_{21} + ...$ non è altro che la somma di tutte le forze interne del sistema, e quindi di ogni forza con la sua
reazione, e non potrà che essere 0. Questo ci permette di enunciare la:
\par\smallskip
\textbf{Prima equazione cardinale dei sistemi} \\
La variazione della quantità di moto di un sistema di punti materiali è uguale alla risultante delle forze esterne
agenti sul sistema:
$$ \frac{d}{dt}\vec{P_{CM}} = \sum m_i\vec{a_i} = \sum \vec{R_i^{est}} $$
\par\smallskip
\textbf{Secondo teorema del centro di massa} \\
Il centro di massa di un sistema si muove come un punto materiale di massa uguale alla massa del sistema e sottoposto alla risultante
delle forze esterne agenti sul sistema:
$$ M\vec{a_{CM}} = \sum \vec{R_i^{est}} $$
\section{Applicazioni di quantità di moto e centro di massa}
Vediamo alcuni esempi che possono essere modellizati attraverso centro di massa e quantità di moto.
\par\smallskip
\textbf{Esplosione di un proiettile a mezz'aria} \\
Poniamo di avere un proiettile, come quelli già studiati, il cui moto comincia dall'origine, con una certa velocità $v_0$
formante un certo angolo $\theta$ con l'asse x. Nel punto di altezza massima del proiettile, esso esplode dividendosi in due pezzi.
Il primo pezzo cade direttamente verso il basso, accelerato dalla gravita, mentre l'alto prosegue di moto parabolico. Si vuole
calcolare le posizioni dove atterrano i due frammenti del proiettile. Inizialmente possiamo dire che le forze coinvolte nell'esplosione
sono tutte interne, ergo non influenzano la posizione del centro di massa. Questo significa che il centro di massa proseguirà di moto
parabolico. Calcoliamo quindi la "gittata" del centro di massa, applicando le formule già note del moto dei proiettili:
$$ x_{CM} = R = \frac{v_0^2\sin{2\theta}}{g} $$
Possiamo allora impostarne la posizione x:
$$ x_{CM} = \frac{m_1x_1 + m_2x_2}{m_1+m_2} $$
Dove $x_1$ e $x_1$ saranno rispettivamente la posizione del frammento caduto verso il basso e del frammento che prosegue il moto.
$x_1$ sarà quindi nota, essendo essa coincidente con il punto di altezza massima della parabola descritta dal proettile prima della
caduta, ovvero la metà della gittata del C.M.:
$$ x_1 = \frac{R}{2} = \frac{v_0^2\sin{2\theta}}{2g} $$
Questo ci porta a scrivere la gittata completa:
$$ R = \frac{m\frac{R}{2} + mx_2}{m_1+m_2} \Rightarrow x_2 = \frac{3}{2}R = \frac{3}{2}\frac{v_0^2\sin{2\theta}}{g}$$
$x_1$ e $x_2$ sono i valori cercati (si noti che il frammento con posizione orizzontale $x_1$ resta sulla stessa ascissa finchè non
tocca terra).
\par\smallskip
\textbf{Disco rotante con massa sul perimetro} \\
Si abbia un disco in rotazione, quindi in moto circolare uniforme, di raggio $r$ e di massa $M$. Sull'estremità del disco, ovvero
sul suo perimetro, è disposta una massetta $m$, che chiaramente ruoterà solidalmente al disco. Il centro di massa di questo sistema
sarà in costante rotazione, in quanto verrà sfasato dalla presenza della massetta $m$. In particolare, possiamo impostare:
$$ x_{CM} = \frac{mx_1 + Mx_2}{m+M}, \quad y_{CM} = \frac{my_1 + My_2}{m+M} $$
Scelto un sistema di coordinate cartesiane centrato sul disco in rotazione, si avrà che $x_2$, ovvero il raggio che collega
il centro del disco a se stesso, sarà 0, mentre $x_1$, ovvero il raggio che collega il centro del disco alla massetta, è il raggio
stesso del disco. Questo ci permette di ottenere un raggio $r_{CM}$, che collega il centro del disco al centro di massa:
$$ r_{CM} = \frac{mr}{m+M} $$
Applicando le solite formule del moto circolare uniforme avremo velocità e accelerazione:
$$ v = \omega \frac{mR}{m+M}, \quad a = \omega^2 \frac{mr}{m+M} $$

\section{Corpo rigido}
Vediamo alcune nozioni preliminari per il calcolo del centro di massa del corpo rigido. Si inizia con le definizioni
di densità:
\begin{itemize}
  \item \textbf{Densità (di volume) di massa} \\
    La densità di massa  $\rho_m$ è definita come la derivata della massa rispetto al volume:
    $$ \rho_m = \frac{dM}{dV}$$
    si misura in $\frac{kg}{m^3}$.
  \item \textbf{Densità superficiale di massa} \\
    La densità superficiale di massa  $\sigma_m$ è definita come la derivata della massa rispetto alla superficie:
    $$ \sigma_m = \frac{dM}{dA}$$
    si misura in $\frac{kg}{m^2}$. La densità superficiale di massa viene usata nei casi in cui lo spessore dell'oggeto considerato
    sia trascurabile.
  \item \textbf{Densità lineare di massa} \\
    La densità lineare di massa  $\lambda_m$ è definita come la derivata della massa rispetto alla lunghezza:
    $$ \lambda_m = \frac{dM}{dl}$$
    si misura in $\frac{kg}{m}$. La densità lineare di massa viene utilizzata nei casi in cui l'unica grandezza rilevante del corpo è la lunghezza.
\end{itemize}
Notiamo che in generale, $\rho_m, \sigma_m$ e $\lambda_m$ sono funzioni delle dimensioni (grandezze estensive) dei corpi considerati.
\par\smallskip
\textbf{Calcolo del centro di massa di un corpo rigido} \\
Se per un sistema contenente un numero finito di punti materiali avevamo:
$$ \vec{r_{CM}} = \frac{\sum m_i\vec{r_i}}{M} = \frac{1}{M} \sum m_i\vec{r_i} $$ 
Per un corpo rigido di forma qualsiasi, sarà necessario passare al corrispettivo continuo della sommatoria, cioè l'integrale:
$$ \vec{r_{CM}} = \frac{1}{M} \int \vec{r_i}dm $$
che diventerà, sulla base delle definizioni precedenti potremo esprimere:
$$ \vec{r_{CM}} = \frac{1}{M} \int \vec{r} \rho(\vec{r})dV $$
in funzione del volume $V$ (o di altre grandezze estensive nel caso si decida di utilizzarle).
Vediamo quindi come si svolge il calcolo dei centri di massa di alcuni corpi:
\begin{itemize}
  \item 
    \par\smallskip
    \textbf{Sbarra con densità lineare di massa costante} \\
    Supponiamo di avere una sbarra di larghezza e profondità trascurabile, e di cui ci interessa quindi solamente
    la lunghezza $L$. Calcoleremo allora la densità lineare di massa $\lambda_m\frac{dM}{dL} = \frac{M}{L}$ (densità lineare
    costante). Applichiamo allora l'equazione precedente per la posizione del centro di massa $\vec{R_{CM}}$:
    $$ x_{CM} = \frac{1}{M}\int_0^L x\lambda_m dx = \frac{\lambda_m}{M} \cdot \frac{x^2}{L}\Big|^L_0  = \frac{\lambda_m}{M} \frac{L^2}{2} = \frac{L}{2} $$
    che corrisponde al punto medio della sbarra, come l'intuito potrebbe suggerire.
  \item
    \par\smallskip
    \textbf{Lamina a profilo semicircolare} \\
    Prendiamo una lamina di forma semicircolare, abbastanza sottile da trascurarne la profondità, e calcoliamone
    il baricentro. Prima di tutto sarà necessario trovare la densità superficiale di massa $\sigma_m = \frac{dM}{dA}$:
    $$ A = \frac{\pi r^2}{2} \Rightarrow \sigma_m = \frac{2M}{\pi r^2} $$
    Il centro di massa avrà posizione $x = 0$ per la simmetria della semicirconferenza. Resta da calcolare la posizione $y$.
    Possiamo impostare l'integrale prendendo "strisce" di altezza $\Delta y$ dalla posizione $y = 0$ alla posizione $y = R$
    sull'asse verticale della circonferenza. La larghezza di queste strisce sarà sempre uguale a $2x$, con $x$ la larghezza
    di uno dei due segmenti in cui l'asse centrale divide la striscia presa sulla circonferenza. L'area della striscia sarà
    quindi $2xy$. Ciò si esprime come:
    $$ y_{CM} = \frac{1}{M}\int_0^r \sigma_m 2xydy $$
    per eliminare la $x$ possiamo prendere dall'equazione della circonferenza:
    $$ x^2 + y^2 = r^2 \Rightarrow x = \sqrt{r^2 - y^2} $$
    da cui:
    $$ y_{CM} = \frac{1}{M}\int_0^r \sigma_m 2y\sqrt{r^2 - y^2}dy $$
    L'integrale di $y\sqrt{r^2 - y^2}$ è:
    $$ \int y\sqrt{r^2 - y^2} dy = -\frac{1}{3}(r^2 - y^2)^\frac{3}{2}, \quad \int_0^r y\sqrt{r^2 - y^2} dy = -\frac{1}{3}(r^2 - y^2)^\frac{3}{2} \Big|^r_0 = \frac{1}{3}r^3 $$
    da cui:
    $$ y_{CM} = \frac{1}{M} \sigma_m \frac{2}{3}r^3 = \frac{1}{M}\frac{2M}{\pi r^2} \frac{2}{3}r^3 = \frac{4r}{3\pi} $$
  \item 
    \par\smallskip
    \textbf{Lamina a profilo triangolare} \\
    Facciamo un'esempio simile al precedente, ma con un profilo triangolare retto anziché semicircolare. Chiameremo $a$ e $b$
    i lati del triangolo. Adottiamo la metodologia usuale: calcoliamo innanzitutto la densità di massa superficiale $\sigma_m$:
    $$ A = \frac{ab}{2} \Rightarrow \sigma_m = \frac{2M}{ab} $$
    E adottiamo un approccio di calcolo integrale. Possiamo pensare di dividere il triangolo in strisce orizzontali e verticali
    per calcolare le posizioni $x$ e $y$ del centro di massa. Avremo che per la $x$, la striscia ha area: 
    $$ x \cdot by $$
    su altezza $y$ con larghezza $x$. Volendo eliminare la $y$, riscriviamola in funzione di $x$:
    $$ x \cdot \frac{b}{a} x $$
    dove notiamo fra l'altro il termine $\frac{b}{a} = \tan{\theta}$ angolo al centro. Impostiamo quindi l'integrale.
    $$ x_{CM} = \frac{1}{M} \int_0^a \sigma_ x \cdot \frac{b}{a} x dx = \frac{1}{M} \frac{2M}{ab} \frac{b}{a} \int_0^a x^2 dx = \frac{2}{a^2} \frac{a^3}{3} = \frac{2}{3}a$$
    Per il calcolo di $y$, il procedimento è analogo, con l'unica differenza il fatto che che la larghezza in $y$ più piccoli è
    più grande, e viceversa, ovvero l'area è:
    $$ y \cdot \frac{a}{b}(b-y) $$
    da cui l'integrale:
    $$ y_{CM} = \frac{1}{M} \int_0^b \sigma_m y \cdot \frac{a}{b}(b-y) dy = \frac{2}{b^2}\int_0^b (yb-y^2) dy = \frac{2}{b^2} (b \frac{y^2}{2} \Big|_0^b - \frac{y^3}{3} \Big|_0^b)= \frac{1}{3}a$$
    abbiamo quindi che il baricentro del triangolo è in $(\frac{2}{3}a, \frac{1}{3}b)$. Questo risultato si può avere anche, più
    direttamente, dalle formule per il baricentro di un triangolo qualsiasi:
    $$ x_{CM} = \frac{x_A+x_B+x_c}{3}, \quad y_{CM} = \frac{y_A+y_B+y_c}{3} $$
    sui punti $A = (0,0), \quad B = (a,0), \quad C = (a,b)$.
  \item 
    \par\smallskip
    \textbf{Semianello rigido} \\
    Prendiamo un'esempio apparentemente simile a quello della semicirconferenza: quello di un semianello rigido di 
    raggio $r$ e massa $M$. Considereremo la densità lineare di massa, che sarà uguale a $\lambda_m = \frac{M}{\pi r}$.
    Per la simmetria, come nell'esempio semicircolare, la posizione orizzontale del centro di massa sarà $x=0$. Resta
    da calcolare la componente verticale: può rendersi utile in questo caso un passaggio alle coordinate polari, con:
    $$ x = r\cos{\theta}, \quad y r\sin{\theta} $$
    da cui l'integrale:
    $$ y_{CM} = \frac{1}{M}\int_0^\pi \lambda_m R \cdot R \sin{\theta} d\theta = \frac{1}{M} \frac{M}{2\pi} r^2 (-\cos{\theta}\Big|^\pi_0) = \frac{2r}{\pi}$$
    
\end{itemize}
\end{document}
