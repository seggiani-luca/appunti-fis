\documentclass[a4paper,12pt]{article}

\usepackage[french,italian]{babel}
\usepackage[T1]{fontenc}
\usepackage[utf8]{inputenc}
\frenchspacing 
\title{Appunti Fisica I}
\author{Luca Seggiani}
\date{14 Marzo 20204}

\begin{document}
\maketitle
\section{Forze di contatto}
La forza di attrito si esprime parallelamente alle superfici di contatto fra i corpi. Si tratta di un interazione
di contatto, mediata dalle interazioni elettromagnetiche fra le superfici dei corpi. Si oppone al moto dei corpi, e ha valore
variabile rispetto al loro stato di quiete o moto:
\begin{itemize}
  \item \textbf{Attrito statico}: è la forza che mantiene fermo un'oggetto in quiete. In questo caso la forza di attrito
    ha valore massimo.
  \item \textbf{Attrito dinamico}: si esprime quando un corpo si muove ("scivola") su una superficie scabra, 
    e agisce nella direzione opposta a quella del movimento.
\end{itemize}
In generale, con $F_s$ forza di attrito  statico, $F_d$ forza di attrito dinamico, e $N$ una forza applicata ad un corpo,
avremo che l'equilibrio si raggiunge con:
$$ F_s \leq N\mu_s \quad F_d = N\mu_d $$
$\mu_s$ e $\mu_d$ sono rispettivamente i coefficienti di attrito statico e dinamico, compresi fra circa 0.05 e 1.5.
In genere, $\mu_d < \mu_s$, e le due grandezze non dipendono all'area di contatto. La forza di attrito si esprime in direzione
parallela a quella del moto, proporzionalmente alla forza normale che il corpo applica sulla superficie.
\par\smallskip
\textbf{Attrito sul piano inclinato in discesa} \\
Su un piano scabro, con un corpo in quiete, abbiamo che:
$$ f_{s} = \mu_sN, \quad N = mg\cos{\theta} $$
Perchè il corpo sia in quiete, avremo bisogno che:
$$ mg\sin{\theta} = f_{s} = \mu_sN = \mu_smg\cos{\theta}$$
da cui si ricava che:
$$ \mu_s = \tan{\theta} $$
ovvero il coefficiente di attrito statico corrisponde alla tangente dell'angolo di pendenza massimo in cui il corpo
è in grado di restare fermo senza scivolare.
Nel caso il corpo sia in movimento, avremo allora che:
$$ mg\sin{\theta} > f_d, \quad \tan{\theta} > \mu_s $$
da cui potremo ricavare l'accelerazione sull'asse parallelo al piano:
$$ a_x = g(\sin{\theta - \mu_d\cos{\theta}}) $$

\end{document}
