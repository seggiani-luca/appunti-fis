\documentclass[a4paper,12pt]{article}

\usepackage[french,italian]{babel}
\usepackage[T1]{fontenc}
\usepackage[utf8]{inputenc}
\frenchspacing 
\title{Appunti Fisica I}
\author{Luca Seggiani}
\date{5 Marzo 2024}

\begin{document}
\maketitle
\section{Moto di un punto materiale sul piano e nello spazio}
Definiamo una traiettoria (o supporto di una curva) $r(t)$ sulle diverse componenti x, y e z:
$$ \vec{r}(t) = (x(t), y(t), z(t)) = \vec{OP}(t), \quad O, P \in \vec{r}(t) $$
a questo punto posizione, velocità ed accelerazione di un punto materiale in moto su tale traiettoria
saranno vettori tridimensionali nello spazio (o bidimensionali sul piano).
\par\smallskip
\textbf{Vettore spostamento} \\
Individuiamo due momenti nel tempo $t_1$ e $t_2$, e ricaviamo da $\vec{r}(t)$ due posizioni lungo la curva
($r_1$ e $r_2$). A questo punto lo spostamento di un punto materiale in moto sulla curva che si sposta dal 
primo al secondo punto sarà:
$$ \Delta \vec{r} = \vec{r}(t_2) - \vec{r}(t_1) = \vec{r_2} - \vec{r_1} $$
oppure, definito un certo intervallo di tempo $\Delta t = t_2 - t_1$:
$$ \Delta \vec{r} = \vec{r}(t + \Delta t) - \vec{r}(t) $$
\textbf{Velocità} \\
Iniziamo col definire la velocità media, proprio come era stato fatto per il moto rettilineo uniforme:
$$ \vec{V_m} = \frac{\Delta \vec{r}}{\Delta t} = \frac{\vec{r_2} - \vec{r_1}}{t_2 - t_1}
= \frac{\vec{r}(t + \Delta t) - \vec{r}(t)}{\Delta t} $$
e la velocità istantanea:
$$ \vec{V} = \frac{d\vec{r}}{dt} = \lim_{\Delta t \rightarrow 0} \vec{V_m} = 
\lim_{\Delta t \rightarrow 0} \frac{\vec{r}(t + \Delta t) - \vec{r}(t)}{\Delta t}$$
vediamo che sulle singole componenti, la velocità $\vec{V_m}$ sarà:
$$ \vec{V} = (V_x, V_y, V_z) = (\frac{dx}{dt}, \frac{dy}{dt}, \frac{dz}{dt}) = (\dot{x}, \dot{y}, \dot{z}) $$
Si noti che geometricamente la velocità calcolata in un certo punto è la tangente a $\vec{r}(t)$ in quel punto.
\par\smallskip
\textbf{Accelerazione} \\
Definiamo, come prima, accelerazione media:
$$ \vec{a_m} = \frac{\Delta \vec{V}}{\Delta \vec{t}} = \frac{\vec{V}(t_2) - \vec{V}(t_1)}{t_2-t_1} $$
ed istantanea:
$$ \vec{a}(t) = \lim_{\Delta t \rightarrow 0} \frac{\vec{v}(t + \Delta t) - \vec{t}}{\Delta t} = \frac{d^2r}{dt^2} $$
sulle singole componenti:
$$ \vec{a}(t) = \frac{dv_x}{dt}\hat{i} + \frac{dv_y}{dy}\hat{j} + \frac{dv_z}{dt}\hat{k}
= \frac{d^2x}{dt^2}\hat{i} + \frac{d^2y}{dt^2}\hat{j} +\frac{d^2z}{dt^2}\hat{k} = \ddot{x}\hat{i} + \ddot{y}\hat{j} + \ddot{z}\hat{k} $$
È fondamentale notare che, fosse $|v_1| = |v_2|$, non è per forza detto che $a=0$! L'accelerazione infatti
sarà influenzata sia dalla variazione longitudinale di velocità del mio punto materiale che dalla sua variazione
in quanto a direzione. Poniamo i due versori $\hat{u}_T$ e $\hat{u}_N$, rispettivamente nella direzione tangente
e radiale del mio punto materiale. Avremo allora:
$$ \vec{v} = v\hat{u}_T, \quad \vec{a} = \frac{dv}{dt}\hat{u}_T + v \frac{d\hat{u}_T}{dt} $$
$\hat{u}_T$ dipende dal tempo, ma
$$ \frac{d(\hat{u}_T \cdot \hat{u}_T)}{dt} = 0 = 2\hat{u}_T \cdot \frac{d(\hat{u}_T)}{dt} \Rightarrow \frac{d\hat{u}_T}{dt} \perp \hat{u}_T $$
ovvero $a_n$, accelerazione lungo la direzione radiale (o normale, ma comunque stabilita secondo
un sistema di riferimento levogiro) della curva è perpendicolare all'accelerazione longitudinale (o tangente) $a_t$. Vediamo poi:
$$ \vec{a} = a_t\hat{u}_T + a_n\hat{u}_N $$
$$ \vec{a} = \frac{dv}{dt}\hat{u}_T + v\frac{d\hat{u}_T}{dt} $$
da cui si nota che:
\begin{itemize}
  \item $a_t$ è legata alla variazione del modulo della velocità,
  \item $a_n$ alla variazione della sua direzione.
\end{itemize}
Si dice che $a_n$ è l'accelerazione centripeta diretta verso l'interno della traiettoria. Inoltre, approssimando
un qualsiasi intorno della curva con un segmento di circonferenza, vediamo che l'accelerazione centripeta è diretta
proprio verso il centro di tale circonferenza. Dal punto di vista delle forze, l'accelerazione centripeta è proprio
ciò che serve a mantenere il corpo lungo la traiettoria descritta da $\vec{r}(t)$ (ed è l'unica forza / accelerazione
in gioco! Non esiste alcuna forza centrifuga, è solamente apparente). \\
Riassumiamo le formule trovate finora:
$$ \vec{a} = \vec{a_n} + \vec{a_t} $$
$$ a_t = |\vec{a_t}| = \frac{d|\vec{v}}{dt}, \quad a_n = |\vec{a_n}| = \frac{v^2}{r} $$
e in totale:
$$ |\vec{a}| = \sqrt{a_n ^ 2 + a_t ^ 2}$$
\par\smallskip
Per concludere l'argomento, osserviamo come, invece di partire dalla definizione di spostamento derivando fino
all'accelerazione, possiamo procedere per la strada inversa: partendo dall'accelerazione e integrando fino allo
spostamento. Partiamo quindi da accelerazione a velocità:
$$ \vec{a} = (a_x, a_y, a_z), \quad t > t_0 $$
$$ a_x = \frac{dv_x}{dt} = dv_x = a_xdt = v_x(t) - v_x(t_0) = \int_{v_{x_0}}^{v_x} dv_x = \int_{t_0}^{t} a_xdt $$
E torniamo da velocità a spostamento:
$$ v_x(t) = v_x(t_0) + \int_{t_0}^{t} a_x(t')dt' $$
potrò procedere analogamente sulle altre componenti, utilizzando oppurtunamente i versori $\hat{i}, \hat{j}$ e $\hat{k}$.

\section{Moti piani su coordinate polari}
Avendo visto la definizione di sistema di coordinate polari, con $(R, \theta)$ raggio e angolo rispetto all'asse
delle ascisse, definiamo:
$$ \vec{R} = (x, y, 0) = (R\cos{\theta}, R\sin{\theta}, 0) $$
$$ \hat{\hat{R}} = (x / R, y / R, 0) = (\cos\theta, \sin\theta, 0) $$
$$ \vec{\theta} = (-\sin\theta, \cos\theta, 0) $$
Generalmente, $\vec{R}$ individuerà un certo punto su una circonferenza di raggio $R$, e $\vec{\theta}$ sarà
la direzione della sua velocità (tangenziale alla circonferenza).
\par\smallskip
\textbf{Moto piano vario} \\
Nel caso di moto piano vario, quindi non ben definito su una circonferenza o qualsasi altra funzione analitica, 
avremo banalmente:
$$ \theta = \theta(t), \quad R = R(t) $$
\par\smallskip
\textbf{Moto piano circolare uniforme non uniforme} \\
Nel caso almeno il luogo su cui avviene il moto sia una circonferenza, potremo stabilire:
$$ \theta = \theta(t), \quad R = a $$
Vediamo ora più nel dettaglio un moto circolare a velocità costante.
\section{Moto piano circolare uniforme}
Nel caso il moto avvenga su una circonferenza e a velocità costante, potremo definire completamente velocità e posizione:
$$ \theta = \omega_0t + \theta_0, \quad R = a $$
si noti che qua e nel caso precedente a è il raggio della nostra circonferenza.
\par\smallskip
\textbf{Velocità angolare} \\
Definiamo la "posizione angolare", cioè semplicemente l'angolo, come un vettore nella direzione $\vec{z}$ di modulo
proporzionale all'angolo stesso:
$$ \vec{\theta} = (0, 0, \theta) = \theta\hat{z} $$
a questo punto la velocità angolare (misurata in $\frac{\mathrm{rad}}{\mathrm{s}}$, o visto che l'angolo è una 
grandezza adimensionale (rapporto di due lunghezze (parentesi innestate)), semplicemente $\mathrm{s} ^ {-1}$) non sarà altro che:
$$ \vec{\omega} = \frac{d\theta}{dt} = (0, 0, \dot{\theta}) = \hat{z}\dot{\theta} $$
\par\smallskip
\textbf{Velocità tangenziale} \\
Definiamo ora la velocità tangenziale, ovvero quella effettiva del punto materiale lungo la tangente alla circonferenza:
$$ \vec{v} = \frac{d\vec{R}}{dt} = \frac{d}{dt}(R\cos{\theta}, R\sin{\theta}, 0)
= (-R\dot{\theta}\sin{\theta}, R\dot{\theta}\cos{\theta}, 0) $$
$$ = R\dot{\theta}(-\sin{\theta}, \cos{\theta, 0})  = \dot{\theta}R\hat{\theta} = \omega R\hat{\theta} $$
dove $\dot{\theta}$ è il versore tangente alla traiettoria del punto materiale. Vediamo inoltre rispetto al valore assoluto:
$$ |\vec{v}| = \omega R $$
e rispetto alla componente radiale:
$$ v_r = \dot{R} = 0 $$
ovvero il raggio non cambia (ed era infatti costante da ipotesi). \\
Riportiamo quindi il vettore finale, in coordinate polari:
$$ \vec{v} = (v_r, v_\theta, v_z) = (0, \omega R, 0), \quad \vec{v} \perp \vec{R} $$
dove è inoltre riportato il fatto che la velocità è perpendicolare al vettore posizione sulla circonferenza $\vec{R}$
in qualsiasi punto. \\
Dimostriamo adesso un fatto importante riguardo alla velocità, conseguenza dell'asse scelto
per $\vec{\omega}$, usando il prodotto vettoriale:
$$ \vec{v} = \vec{\omega} \times \vec{R} $$
$$ = \dot{\theta}\hat{z} \times \vec{R} = \dot{\theta}\hat{z} \times (\hat{x}R\cos{\theta} + \hat{y}R\sin{\theta}) $$
$$ = \dot{\theta}R(\hat{z} \times \hat{x}\cos{\theta} + \hat{z} \times \hat{y}\sin{\theta}) = \dot{\theta}R(\hat{y}\cos{\theta} - \hat{x}\sin{\theta}) $$
$$ = \omega R \dot{\theta} $$

\textbf{Accelerazione centripeta} \\
Basandoci proprio su quest'ultima equivalenza, studiamo il valore dell'accelerazione centripeta che mantiene il nostro punto
punto materiale sulla circonferenza:
$$ \vec{a} = \frac{d(\vec{\omega} \times \vec{R})}{dt} = \frac{d\vec{\omega}}{dt} \times \vec{R} + \vec{\omega} \times \frac{d\vec{R}}{dt}$$
$$ = \vec{\omega} \times \frac{d\vec{R}}{dt} = \vec{\omega} \times (\vec{\omega} \times \vec{R}) $$
Notiamo come il primo termine della somma ottenuta nella terza equazione si annulli in quanto la derivata della velocità
angolare $\omega$, costante, sarà nulla. Risolviamo quindi l'ultimo termine usufruendo
della relazione:
$$ \vec{A} \times (\vec{B} \times \vec{C}) = \vec{B}(\vec{A} \cdot \vec{C}) - \vec{C}(\vec{A} \cdot \vec{B}) $$
ottenendo:
$$ \vec{\omega} (\vec{\omega} \cdot \vec{R}) - \vec{R}(\vec{\omega} \cdot \vec{\omega}) = -\omega^2 R \hat{R} $$
$$ -\omega = -\frac{v^2}{r^2}, \quad -\omega^2 R \hat{R} = \frac{-v^2 R}{R^2} \hat{R} = -\frac{V^2}{R} \hat{R} $$
notiamo che l'ultima e la terzultima formula, equalmente valide, hanno segno negativo: questo perchè, scelto
un riferimento levogiro sulla circonferenza, il versore $\hat{R}$ viene ad orientarsi verso l'esterno della circonferenza,
mentre per definizione la forza centripeta è diretta verso il centro della circonferenza.
\par\smallskip
\textbf{Periodo e pulsazione} \\
Infine, riprendiamo brevemente le nozioni di periodo e pulsazione applicate al moto circolare uniforme. Diciamo
che la velocità angolare $\omega$ può essere chiamata anche pulsazione. A questo punto, trovato un periodo $T$, avremo:
$$ VT = |\omega|RT = 2\pi R \Rightarrow T = \frac{2\pi}{|\omega|} $$
ricordando anche la frequenza $v$, definita come:
$$ v = \frac{1}{|T|} $$
\end{document}
