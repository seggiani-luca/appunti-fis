\documentclass[a4paper,12pt]{article}

\usepackage[french,italian]{babel}
\usepackage[T1]{fontenc}
\usepackage[utf8]{inputenc}
\frenchspacing 
\title{Appunti Fisica I}
\author{Luca Seggiani}
\date{10 Maggio 2024}

\usepackage{amsmath}

\begin{document}
\maketitle
\section{Teorema di König per il momento angolare}
Per un corpo rigido in rotazione con velocità $\omega$, il momento rispetto ad un asse di simmetria di distribuzione
della masse, o un asse ad esso parallelo, vale per il teorema di König:
$$ \vec{L} = M\vec{R}_{CM} \times \vec{V}_{CM} + I\omega$$
dove il primo termine rappresenta il momento angolare del centro di massa rispetto al polo scelto, e il secondo termine rappresenta
il momento angolare dovuto alla rotazione attorno al centro di massa.
\par\smallskip
\textbf{Dimostrazione} \\
Dimostriamo adesso questo risultato: iniziamo con la definizione di momento angolare per un sistema di punti materiali:
$$ \vec{L} = \sum m_i(\vec{R_i} \times \vec{V_i}) $$
Adesso scomponiamo $\vec{R_i}$ e $\vec{V_i}$ in $(\vec{R'_i} + \vec{R}_{CM})$ e $(\vec{V'_i} + \vec{V}_{CM})$ rispettivamente,
ovvero posizione e velocità del corpo rigido e del punto $i$ rispetto al punto rigido. Lo svolgimento dei calcoli porterà a:
$$ L = \sum m_i (\vec{R'_i} + \vec{R}_{CM}) \times (\vec{V'_i} + \vec{V}_{CM}) $$
$$ = \sum m_i (\vec{R'_i} \times \vec{V'_i}) + \sum m_i(\vec{R'_i} \times \vec{V}_{CM}) + \sum m_i ( \vec{R}_{CM} \times \vec{V'_i}) + M(\vec{R}_{CM} + \vec{V}_{CM}) $$
Fra questi termini notiamo $\sum m_i(\vec{R'_i} \times \vec{V}_{CM})$ e $\sum m_i ( \vec{R}_{CM} \times \vec{V'_i})$, entrambi nulli, in quanto il primo rappresenta
la posizione del CM nel sistema di riferimento del CM , e il secondo la quantità di moto del CM nel sistema di riferimento del CM. Abbiamo quindi:
$$ L =  \sum m_i (\vec{R'_i} \times \vec{V'_i})  + M(\vec{R}_{CM} + \vec{V}_{CM}) $$
Riconosciamo il primo termine: è la quantità di moto del corpo rigido attorno ad un asse, divisa nelle componenti $\hat{z}$ e $\hat{u}$:
$$ L_z = I\omega, \quad L_u = \sum m_i\omega R_i^2 \sin{\theta_i} \cos{\theta_i} $$
Assumendo un asse su una distribuzione di massa simmetrica, o un asse ad esso parallelo, avremo:
$$ \vec{L} = M\vec{R}_{CM} \times \vec{V}_{CM} + I\omega$$
che è quanto volevamo dimostrare.
\section{Pendolo fisico o composto}
Vediamo il comportamento di un corpo rigido imperniato su un assediverso da quello passante per il suo centro di massa. 
Potremo innanzitutto studiare la condizione di equilibrio, imponendo:
$$ 
\left\{\begin{array}{l}
    \vec{R} + m\vec{g} = 0 \cr \\
    \vec{\tau} = 0
\end{array}\right.
$$
Ovvero la risultante di reazione vincolare e gravita è nulla, e il momento delle forze è nullo.
Per la prima equazione potremo vedere i due assi:
$$ 
\left\{\begin{array}{l}
    \vec{R}_x = 0 \cr \\
    \vec{R_y} = m\vec{g}
\end{array}\right.
$$
Mentre per il momento avremo:
$$ \vec{\tau} = 0 \Rightarrow \vec{R}_{CM} \times Mg = -\vec{R}_{C} MG \sin{\theta} = 0, \quad \theta = 0$$
Ovvero, il corpo è in equilibrio se il centro di massa si trova esattamente il polo, come l'intuito potrebbe suggerire. \\
Più interessante è il caso di non equilibrio, dove il sistema si comporta effettivamente come un oscillatore armonico per piccole
oscillazioni. Avremo la prima cardinale:
$$ \vec{R} + m\vec{g} = m\vec{a_{CM}} $$
e la seconda cardinale:
$$ \vec{\tau} = I\vec{\alpha} = \vec{R}_{CM} \times m\vec{g} $$
Scriviamo $\alpha$ come $\ddot{\theta}$, ovvero la derivata seconda della posizione angolare:
$$ I\ddot{\theta} = -\vec{R}_{CM} mg\sin{\theta} $$
Il fattore $\sin{\theta}$ risulta svantaggioso: per piccole oscillazioni, possiamo prendere il primo termine dello sviluppo di Taylor:
$$ \ddot{\theta} + \frac{\vec{R}_{CM} \times m\vec{g}}{I}\theta = 0 $$
Da cui otterremo la pulsazione (indicata con $\Omega$ per distinguerla dalla velocità angolare $\theta$), il periodo e la posizione angolare in funzione del tempo:
$$ \Omega = \sqrt{\frac{\vec{R}_{CM} \times m\vec{g}}{I}}, \quad T = \frac{2\pi}{\Omega}, \quad \theta(t) = A\cos(\Omega t + \phi) $$
\par\smallskip
\textbf{Urti fra corpi rigidi e punti materiali} \\
Vediamo adesso come studiare urti fra corpi rigidi, vincolati e non:
\begin{itemize}
  \item \textbf{Urti fra corpi rigidi non vincolati e punti materiali} \\
    In un urto fra un corpo rigido non vincolato e un punto materiale, si conserveranno:
    \begin{itemize}
      \item Quantità di moto del sistema (sui due assi),
      \item Momento angolare rispetto a un \textit{qualunque} polo.
    \end{itemize}
    Nel caso l'urto sia elastico, si conserverà anche l'energia. Notiamo che questo da solo non ci permette
    di assumere che l'energia si conservi in ogni caso: dovremo prima determinare il tipo di urto.
    L'indifferenza nella scelta del polo significa che potremo scegliere un polo ottimo per il tipo di situazione
    che stiamo analizzando (spesso il centro di massa).
  \item \textbf{Urti fra corpi rigidi vincolati e punti materiali} \\
    Vediamo adesso il caso in cui il corpo sia vincolato attorno a un qualche asse. La presenza del vincolo
    comporterà l'azione di una forza impulsiva applicata dal vincolo sul corpo rigido, identica all'impulsiva generata dalla collisione del punto
    materiale e opposta in verso. Chiamando questa forza $\vec{R}$ reazione, avremo sul tempo dell'impulso $\tau$:
    $$ I = \int_0^\tau \vec{R}(t)dt' = \vec{P}(\tau) - \vec{P}(0) = \Delta \vec{P} $$
    ergo non si conserva la quantità di moto. \\
    Si conserverà invece il momento angolare sul polo coincidente con l'asse del vincolo: questo perchè qualsiasi forza impulsiva applicata sul punto del 
    vincolo avra raggio distanza $\vec{R_i} = 0$ dal polo, e quindi momento angolare nullo. \\
    Come prima, poi, nel caso l'urto sia elastico, si conserverà anche l'energia.
\end{itemize}
\end{document}
