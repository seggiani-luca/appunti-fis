\documentclass[a4paper,12pt]{article}

\usepackage[french,italian]{babel}
\usepackage[T1]{fontenc}
\usepackage[utf8]{inputenc}
\frenchspacing 
\title{Appunti Fisica I}
\author{Luca Seggiani}
\date{20 Maggio 2024}
\begin{document}
\maketitle
\par\smallskip
\textbf{Un caso di circuitazione del campo magnetico} \\
Vediamo la circuitazione del campo magnetico su una spira percorsa da corrente. Prendiamo, a scopo di esempio, 
una firma semicircolare di raggio $R$ (si considerino, se necessario, unità di misurà S.I.). Il campo
magnetico sarà in questo caso perpendicolare alla base della semicirconferenza. Possiamo dividere
la forza totale applicata dal campo magnetico nella forza applicata sulla parte curva della spira 
(la semicirconferenza in sé per sé), e la forza applicata sul diametro di base:
$$ \vec{F} = \vec{F_s} + \vec{F_d} $$
Avevamo che, dalla forza di Lorentz:
$$ \vec{F} = I\int_A^B dl \times \vec{B} $$
Visto che nell'integranda compare il prodotto vettoriale con $\vec{B}$, l'unico spostamento che ci interessa è
quello sull'asse perpendicolare a $\vec{B}$, ergo l'asse x. Abbiamo allora:
$$ \vec{F_s} = I\int_A^B dl \times \vec{B} = 2IRB\hat{z}, \quad \vec{F_b} = I\int_A^B dl \times \vec{B} = -2IRB\hat{z} $$
da cui:
$$ \vec{F} = 2IRB\hat{z} - 2IRB\hat{z} = 0 $$
Come si poteva già verificare dal fatto che il percorso della spira è chiuso (ergo circuitazione nulla per il campo magnetico).
\par\smallskip
\textbf{Momento di una spira} \\
Consideriamo il caso di una spira rettangolare, di lati $a$ e $b$, immersa in un campo magnetico e percorsa da corrente. Sia l'asse verticale della spira perpendicolare
al campo magnetico, $a$ la lunghezza dei lati verticali, $b$ la lunghezza dei lati orizzontali. La forza di Lorentz applicata sui lati
orizzontali della spira sarà identica è diretta verso l'alto e verso il basso rispettivamente, quindi a risultante nulla. La forza di Lorentz
applicata sui lati verticali dipenderà dalla corrente e dal campo magnetico, e tendera a ruotare la spira, allineandone la normale
col campo magnetico nel caso di corrente positiva (antioraria guardandola con campo uscente), l'inverso nel caso di corrente negativa.
Quantitaviamente, la forza applicata sarà, per un singolo lato di lunghezza $a$:
$$ \vec{F} = aI \times \vec{B}, \quad F = aIB\sin{\theta} $$
Dove $\theta$ è l'angolo formato dalla normale della spira con il campo magnetico. Da questo si ricava il momento della forza:
$$ \vec{M} = aIB \cdot b\sin{\theta} \, \hat{z} $$
Sulla base di questo definiamo il \textbf{momento magnetico}:
$$ \vec{m} = AI\hat{n} $$
Sulla normale $\hat{n}$ della spira. Attraverso il momento magnetico, possiamo ricavare il \textbf{momento meccanico} (quello calcolato prima) come:
$$ \vec{M} = \vec{m} \times \vec{B} $$ 
\par\smallskip
\textbf{Piccole oscillazioni di una spira} \\
Possiamo esprimere il momento meccanico di una spira $\hat{M}$ come la derivata del momento d'inerzia rispetto al tempo:
$$ \vec{M} = \frac{d\vec{L}}{dt} = I\vec{\alpha} $$
da cui, sostituendo:
$$ I\ddot{\theta}\hat{z} = -mB\sin{\theta}\hat{z}$$
Osserviamo che questa è un'equazione differenziale che descrive un moto armonico. Trascurando i versori, abbiamo:
$$ \ddot{\theta} + \frac{mB}{I}\theta = 0 $$
da cui ricaviamo pulsazione e periodo:
$$ \omega = \sqrt{\frac{mB}{I}}, \quad T = 2\pi\sqrt{\frac{I}{mB}} $$
\par\smallskip
\textbf{Considerazioni energetiche sul momento di una spira} \\
Calcoliamo l'energia potenziale della spira, posta nel campo con un certo angol $\theta$. Notiamo innanzitutto che la spira percorsa da
corrente forma effettivamente un \textbf{dipolo magnetico}: attraverso la regola della mano destra, possiamo individuare la direzione del campo:
ruotando le dita nella direzione della corrente sarà la direzione indicata dal pollice. Possiamo allora dire che l'energia potenziale (o il lavoro
fatto per raggiungere tale energia potenziale) per un moto rotazionale è, analogamente a quanto avevamo definito su moti traslazionali:
$$ dL = \vec{F} \cdot ds \sim dL = \vec{\tau} \cdot d\theta $$
In questo il momento della forza $\tau$ sarà il momento meccanico $\vec{M} = AI\hat{n} \times \vec{B}$:
$$ dL = AI\hat{n} \times \vec{B} \, d\theta = AI \cdot B\sin{\theta} \, d\theta $$
Passando ad integrale si ha:
$$ L = \int dL = \int AI \cdot B\sin{\theta} \, d\theta = -AIB \cos{\theta} $$
Questo è in accordo col fatto che $\frac{d}{d\theta}L = \vec{\tau} = \vec{M}$:
$$ \frac{d}{d\theta} (-AIB\cos{\theta}) = AIB\sin{\theta} = \vec{M} $$
Notiamo che questo sistema è analogo a quello formato da un dipolo elettrico immerso in un campo elettrico. In questo,
il momento magnetico $\vec{m} = AI\hat{n}$ equivale al momento di dipolo $\vec{p} = qd\hat{n}$, e vale $\tau = \vec{p} \times \vec{E}$.
\par\smallskip
\textbf{Effetto Hall} \\
Descriviamo adesso l'effetto Hall, fenomeno che può essere usato per verificare la natura effettiva dei portatori di carica di un materiale conduttore.
Poniamo di avere una sbarra di materiale conduttore di lunghezza arbitraria, con una differenza di potenziale fra le estremità tale da generare un campo
elettrico (che assumeremo diretto verso destra), e quindi una corrente al suo interno. Immergiamo adesso la sbarra in un campo elettrico perpendicolare
al campo elettrico (che assumiamo entrante). Abbiamo la densità di carica $J = nq\vec{v_d}$, dove figura la \textit{velocità di deriva} dei portatori
di carica, e la forza di Lorentz $\vec{F} = q(\vec{E} + \vec{v} \times \vec{B})$. I portatori di carica, passando attraverso il campo magnetico, spinti dal
campo elettrico, subiranno una spinta verso l'alto o verso il basso a seconda del loro segno. Questo porterà alla registrazione
di una differenza di carica (e quindi di potenziale) sull'estremità superiore e inferiore della sbarra. Per la precisione, abbiamo che se i portatori
di carica sono negativi (come nel caso degli elettroni), la parte inferiore si carica positivamente e la parte superiore negativamente, mentre se
sono positivi (ad esempio, lacune elettroniche) abbiamo l'effetto inverso.
\par\smallskip
\textbf{Moto di particelle immerse in un campo magnetico} \\
Vediamo il comportamento di particelle carice immerse in un campo magnetico, in due casi: quello dove la velocità iniziale è
tutta perpendicolare al campo, e quello in cui la velocità iniziale possiede una componente parallela al campo.
\begin{itemize}
  \item \textbf{Velocità iniziale perpendicolare al campo} \\
    Nel caso in cui una particella si trovi all'interno di un campo magnetico con una velocità $\vec{v}$ perpendicolare al campo, varrà la forza
    di Lorentz: $\vec{F} = q\vec{v} \times \vec{B}$, diretta in direzione sempre ortogonale alla velocità della particella. In questa configurazione di
    forze, la particella subisce un'accelerazione centripeta che la porta a descrivere una circonferenza, muovendosi di moto circolare uniforme. Possiamo
    ricavere, raggio, pulsazione e periodo di tale moto attraverso la formula per l'accelerazione centripeta in un moto circolare uniforme:
    $$ a = \frac{v^2}{r}, \quad ma = qvB\sin{\theta} $$
    notiamo che $\theta = \frac{pi}{2}$ (forza sempre ortogonale alla velocità), quindi:
    $$ a = \frac{qvB}{m}, \quad \frac{v^2}{r} = \frac{qvB}{M} $$
    da cui si ricava:
    $$ r = \frac{mv}{qB}, \quad \omega = \frac{v}{r} = \frac{qB}{m}, \quad v = \omega r = \frac{qBr}{m}, \quad T = \frac{2\pi}{\omega} = \frac{2\pi m}{qB} $$
  \item \textbf{Velocità iniziale con componente parallela al campo} \\
    Nel caso la particella iniziale presenti una velocità con componente parallela al campo, avremo che tale componente non influenzerà
    il moto circolare. Più precisamente, potremo dire:
    $$ \vec{v} = \vec{v}_\parallel + \vec{v}_\perp $$
    Varranno tutte le formule precedenti, rispetto alla sola $\vec{v}_\perp$:
    $$ r = \frac{mv_\perp}{qB}, \quad \omega = \frac{v_\perp}{r} = \frac{qB}{m}, \quad v_\perp = \omega r = \frac{qBr}{m}, \quad T = \frac{2\pi}{\omega} = \frac{2\pi m}{qB} $$
    Mentre sull'asse parallelo si conserverà $v_\parallel$. Il moto risultante sarà quindi a traiettoria elicoidale, con pulsazione e periodo
    uguali a quelle del caso planare. Potremo calcolare il passo dell'elica descritta dalla particella come:
    $$ d = v_\parallel \cdot T = v_\parallel \cdot \frac{2\pi m}{qB} $$
\end{itemize}
\section{Legge di Biot-Savart}
Il campo magnetico non è generato esclusivamente da magneti permanenti: nel 1819 Oersted scoprì che anche un filo percorso da corrente
genera un campo magnetico. Un formalismo per la definizione di questo campo è dato dalla cosiddetta prima legge di Laplace, detta anche \textbf{Legge di Biot-Savart}:
$$ d\vec{B} = \frac{\mu_0}{4\pi} \frac{Id\vec{s} \times \vec{r}}{r^2} $$
Diamo un significato ai simboli della formula. Innanzitutto, $\frac{\mu_0}{4\pi} = k_m$ detta \textbf{costante magnetica}, è il corrispettivo
magnetico della costante elettrica $k_e$. Si noti che in questi appunti $k_m$ viene indicata quasi sempre come $\frac{\mu_0}{4\pi}$, e la
lettera k viene riservata alla costante elettrica. Nel caso sia necessario, si useranno esplicitamente $k_m$ e $k_e$. $\mu_0$ è la permittività
magnetica del vuoto, corrispettivo magnetico della permettività elettrica del vuoto $\epsilon_0$, e assume il valore di $4\pi \cdot 10^{-7} \frac{\mathrm{T} \cdot \mathrm{m}}{\mathrm{A}}$, com'è di
facile verifica dalla legge di Biot-Savart. Vediamo quindi il vettore $d\vec{s}$ e il versore $\hat{r}$. La legge di Biot-Savart si riferisce
al campo magnetico infinitesimo $d\vec{B}$ generato da un circuito filiforme (un filo) su un certo punto ad esso esterno $\vec{P}$. A questo i punto, il vettore
$d\vec{s}$ è l'elemento infinitesimo del filo $s$, e il versore $\hat{r}$ è il versore rivolto da $ d\vec{s}$ al punto $\vec{P}$. A questo punto abbiamo
quindi che il campo magnetico generato da un filo sui punti circostanti ha la forma di circonferenze concentriche al filo, che ruotano in senso antiorario rispetto alla direzione
di scorrimento della corrente (come dalla legge della mano destra), e diminuisce come l'inverso del quadrato della distanza. Inoltre, possiamo calcolare il contributo sul campo di ogni elemento infinitesimale $d\vec{s}$, prendendo
l'integrale:
$$ \vec{B} = \frac{\mu_0 I}{4\pi} \int \frac{d\vec{s} \times \hat{r}}{r^2} $$
\par\smallskip
\textbf{Applicazioni di Biot-Savart} \\
Vediamo adesso l'applicazione della legge di Biot-Savat (prima formula di Laplace) ad alcune distribuzioni circuitali:
\begin{itemize}
  \item \textbf{Campo magnetico generato da un filo rettilineo} \\
    Vediamo il caso più semplice: quello di un filo conduttore attraversato da una corrente $I$, che genera un campo magnetico in un certo punto $\vec{P}$.
    Tracciamo i segmenti che si staccano dal filo con angolo $\theta$ (sempre orientato nella stessa direzione) dagli estremi di un segmento di lunghezza $2s$ e centro coincidente
    con $d\vec{s}$, e raggiungono il punto $\vec{P}$. Questa situazione non è dissimile da quella che abbiamo studiato per cercare il campo elettirico generato da un filo uniformemente
    carico. Possiamo quindi applicare Biot-Savart per il calcolo dell'elemento infinitesimo di campo:
    $$ d\vec{B} = \frac{\mu_0}{4\pi} \frac{Id\vec{s} \times \vec{r}}{r^2} $$
    Iniziamo col notare che il prodotto vettoriale fra $d\vec{s}$ e $\hat{r}$ è uguale a:
    $$ d\vec{s} \times \hat{r} = ds \sin{\theta} $$
    essendo $\pi - \theta$ uguale all'angolo compreso fra $d\vec{s}$ e $\hat{r}$, e:
    $$ \sin{\theta} = \sin{(\pi - \theta)} $$
    Adesso dobbiamo trovare il valore di $d\vec{s}$ e $r$.
    Iniziamo con $d\vec{s}$. Notiamo che il rapporto fra $R$ e l'elemento di lunghezza $s$ è uguale alla tangente di $\frac{\pi}{2} - \theta$:
    $$ \frac{R}{s} = \tan{\left(\frac{\pi}{2} - \theta\right)} = \cot{\theta} $$
    e quindi:
    $$ \frac{R}{s} = \cot{\theta} \Rightarrow \frac{s}{R} = \frac{1}{\cot{\theta}} = \tan{\theta}, \quad s = R\tan{\theta} $$
    derivando da entrambe le parti, con un po' di calcolo discutibile, abbiamo:
    $$ \frac{ds}{d\theta} = \frac{d}{d\theta} R\tan{\theta} = \frac{R}{\sin^2{\theta}} \Rightarrow ds = \frac{Rd\theta}{\sin^2{\theta}} $$
    abbiamo un'espressione di $ds$. Troviamo quindi $r$, notando che:
    $$ r\sin{\theta} = R, \quad r = \frac{R}{\sin{\theta}} $$
    Possiamo adesso sostituire nella legge di Biot-Savart:
    $$ d\vec{B} = \frac{\mu_0I}{4\pi} \frac{Rd\theta}{\sin^2{\theta}} \frac{\sin^2{\theta}}{R^2} = \frac{\mu_0 I}{4\pi R} \sin{\theta} \, d\theta $$
    Questo si presta all'integrazione del termine $\sin{\theta}\,d\theta$, per trovare il campo generato da un filo di lunghezza arbitraria. Per il filo
    la cui lunghezza è determinato da un certo angolo $\theta'$ che si stacca dal filo conduttore, si ha:
    $$ \vec{B} =  \frac{\mu_0 I}{4\pi R} \int_{\theta'}^{\pi - \theta'} \sin{\theta} \, d\theta =  \frac{\mu_0 I}{4\pi R} \Big| -\cos{\theta} \Big|^{\pi - \theta'}_{\theta'}  =  \frac{\mu_0 I}{2\pi R}\cos{\theta'} $$
    Per un filo di lunghezza arbitraria $2s$ possiamo porre $\tan{\theta'} = \frac{R}{s}$. Il caso di interesse sarà però quello
    di un filo di lunghezza infinita. In questo caso l'angolo $\theta'$ tenderà a $0$, e si avrà:
    $$ \vec{B} = \frac{\mu_0I}{2\pi R} = k_m \frac{2I}{R}$$
    Notiamo le similiarità col campo generato da un filo uniformemente carico:
    $$ \vec{E} = \frac{\lambda}{2\pi\epsilon_0 R} $$
  \item \textbf{Campo magnetico generato da una spira} \\
    Studiamo adesso il campo magnetico generato da una spira circolare di raggio $R$. Anche qui, il caso è analogo a quanto avevamo già visto riguardo
    al campo elettrico generato da una distribuzione simile. Poniamo $\theta$ come l'angolo che collega un punto sulla circonferenza a $\vec{P}$, punto a distanza
    $x$ dal centro della spira. Potremo nuovamente prendere in considerazione Biot-Savart:
    $$ d\vec{B} = \frac{\mu_0}{4\pi} \frac{Id\vec{s} \times \vec{r}}{r^2} $$
    Vediamo quindi che l'elemento $d\vec{s}$ e il versore $\hat{r}$ sono sempre perpendicolari, ergo:
    $$ d\vec{s} \times \hat{r} = ds $$
    Noto l'angolo $\theta$, l'elemento di campo si ridurrà a:
    $$ d\vec{B} = \frac{\mu_0}{4\pi} \frac{ds}{r^2}\cos{\theta} $$
    Basterà quindi trovare $r$ e $\cos{\theta}$. Per quanto riguarda $r$, ricaviamo con Pitagora:
    $$ r^2 + R^2 + x^2 $$
    Mentre per $\cos{\theta}$, calcoliamo direttamente dal rapporto fra il raggio e la distanza $r$ appena trovata:
    $$ \cos{\theta} = \frac{R}{\sqrt{R^2 + x^2}} $$
    Sostituendo in Biot-Savart, abbiamo:
    $$ d\vec{B} = \frac{\mu_0 I}{4\pi} \frac{ds}{R^2 + x^2} \frac{R}{\sqrt{R^2+x^2}} = \frac{\mu_0I}{4\pi} \frac{R}{(R^2+x^2)^{\frac{3}{2}}}ds $$
    Passiamo alla versione integrale per trovare il campo totale, notando che $ds$ è l'elemento di una circonferenza di raggio $R$.
    $$ \vec{B} = \frac{\mu_0I}{4\pi} \frac{R}{(R^2+x^2)^{\frac{3}{2}}}\int ds = \frac{\mu_0I}{4\pi} \frac{R}{(R^2+x^2)^{\frac{3}{2}}} 2\pi R = \frac{\mu_0IR^2}{2(R^2+x^2)^{\frac{3}{2}}} $$
    Da questo risultato possiamo ricavare il camp magnetico in un punto esattamente interno alla spira:
    $$ \vec{B}_0 = \frac{\mu_0I}{2R} $$
    e un'approssimazione per raggi molto piccoli (o distanze molto grandi):
    $$ x >> a \Rightarrow B \approx \frac{\mu_0IR^2}{2R^3} $$
    per un momento magnetico (corrente per area, $\mu = I(\pi R^2)$) $\mu$ avremo:
    $$ \vec{B} = \frac{\mu_0}{2\pi} \frac{\mu}{R^3}$$
    che notiamo essere consistente con il campo generato da un dipolo:
    $$ \vec{E} = \frac{2kp}{R^3} $$
    Questo è naturale, in quanto una spira in corrente è effettivamente un dipolo magnetico. Si potrebbe
    adesso porre la questione di come mai, visto che il campo generato da un magnete qualsiasi (ergo un dipolo) decade
    come $\frac{1}{R^3}$, nella legge di Biot-Savart figura il termine $\frac{1}{R^2}$. Si può dire che sebbene il contributo
    di ogni segmento infinitesimo della distribuzione filiforme decada come $\frac{1}{R^3}$, il loro contributo collettivo va
    come $\frac{1}{R^2}$.
\end{itemize} 
\end{document}
