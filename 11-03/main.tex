\documentclass[a4paper,12pt]{article}

\usepackage[french,italian]{babel}
\usepackage[T1]{fontenc}
\usepackage[utf8]{inputenc}
\frenchspacing 
\title{Appunti Fisica I}
\author{Luca Seggiani}
\date{11 Marzo 2024}

\usepackage{amsmath}

\begin{document}
\maketitle
\section{Moto dei proiettili}
Studiamo adesso il moto dei proiettili, ovvero corpi lanciati con una certa velocità iniziale $\vec{v_0}$ e
da lì in poi soggetti all'accelerazione di gravità $g$. Il moto avviene nel piano individuato da $g$ e $\vec{v_0}$,
e può essere diviso nelle componenti:
$$
\begin{aligned}
\left\{\begin{array}{l}
  a_x = 0 \cr \\
  a_y = -g 
\end{array}\right.
& 
\left\{\begin{array}{l}
  v_x = v_{0x} = \mathrm{const.} \cr \\
  v_y = v_{0y} - gt 
\end{array}\right.
&
\left\{\begin{array}{l}
  x = x_0 + v_{0x}t \cr \\
  y = y_0 + v_{0y}t - \frac{1}{2}gt^2 
\end{array}\right.
\end{aligned}
$$

troviamo adesso dei risultati interessanti riguardo al moto dei proiettili. Possiamo innanzitutto
determinare:
\par\smallskip
\textbf{Traiettoria} \\
Elminiamo $t$ dalle equazioni della posizione ricavando $t$ dalla prima e sostituendolo nella seconda:
$$ x = x_0 + v_{0x}t, \quad t = \frac{x-x_0}{v_{0x}} $$
$$ y = y_0 + v_{0y}(\frac{x-x_0}{v_{0x}})-\frac{1}{2}g(\frac{x-x_0}{v_{0x}})^2 $$
impostiamo un sistema di riferimento cartesiano centrato sul punto da dove viene lanciato il proiettile,
ponendo quindi $x_0$ e $y_0$ coordinate iniziali uguali a zero. Inoltre, definiamo $\vec{v_0}$ in coordinate polari,
ovvero in funzione dell'angolo di lancio $\theta$ e del suo modulo $v_0$:
$$ x_0, y_0 = 0, \quad \vec{v_0} = (v_0\cos{\theta}, v_0\sin{\theta}) $$
ottenendo l'equazione finale della traiettoria:
$$ y = \tan{\theta}x - \frac{gx^2}{2(v_0\cos{\theta})^2}$$
Determinamo adesso la distanza percorsa dal proiettile, ovvero la:
\par\smallskip
\textbf{Gittata} \\
Cerchiamo il punto dove il proiettile tocca terra, ponendo la seconda equazione dello spostamento a 0:
$$ 0 = v_{0y}t - \frac{1}{2}gt^2, \quad \frac{1}{2}gt^2 = v_{0y}t $$
da cui, escludendo la soluzione triviale $t = 0$ (sarebbe il punto da cui lo lanciamo, ch ovviamente coincide con terra), abbiamo:
$$ t = 2\frac{v_{0y}}{g} $$
sostituendo nella prima equazione:
$$ x = v_{0x}t, \quad x = 2\frac{v_{0x}v_{0y}}{g} $$
Notiamo che per la proprietà trigonometrica:
$$ \sin{2x} = 2\cos{x}\sin{x} $$
$v_{0x}v_{0y}$ vale:
$$ \frac{v_0^2\sin{(2\theta})}{2} $$
da cui la formula finale per la gittata:
$$ \frac{v_0^2 \sin{(2\theta)}}{g} $$
da cui è tra l'altro ovvio che la gittata massima si otterrà ad un angolo $\theta = 45^\circ$ ($\sin{2\frac{\pi}{4}} = 1$).


\end{document}
