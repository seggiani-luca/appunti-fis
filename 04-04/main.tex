\documentclass[a4paper,12pt]{article}

\usepackage[french,italian]{babel}
\usepackage[T1]{fontenc}
\usepackage[utf8]{inputenc}
\frenchspacing 
\title{Appunti Fisica I}
\author{Luca Seggiani}
\date{4 Aprile 2024 (svegliati è primavera!)}

\usepackage{amsmath}

\begin{document}
\maketitle
\section{Sistemi di riferimento inerziali}
Si riportano alcuni esempi di applicazione della relatività galileiana.
\subsection{Treno in moto}
Facciamo l'esempio di un treno in moto con velocità $v_0$ costante. Sul treno, un passeggero lascia cadere
un gesso che si muove spinto dall'accelerazione gravitazionale $g$. Sia $S'$ il sistema di riferimento del treno,
e $S$ il sistema di riferimento di laboratorio (poniamo della banchina), e siano i due sistemi di riferimento
coincidenti nell'istante in cui si lascia cadere il gesso. Possiamo applicare le trasformazioni di Galileo:
$$ 
\left\{\begin{array}{l}
  \vec{r} = \vec{r'} + \vec{v_0}t \cr \\
  \vec{v} = \vec{v'} + \vec{v_0} \cr \\
  \vec{a} = \vec{a'}
\end{array}\right.
$$
in $t=0$ le coordinate del gesso sono identiche nei due sistemi, e il gesso ha velocità iniziale nulla
e accelerazione $-gt\hat{y}$ sull'asse verticale identica nei due sistemi. Sia $(x_0, y_0)$ la posizione
iniziale del gesso, la sua posizione rispetto al treno sarà allora:
$$ \vec{r'} = (x_0, y_0 -\frac{1}{2}gt^2) $$
Applicando le trasformazioni galileiane avremo che, rispetto al laboratorio:
$$ 
\vec{a} = -g\hat{y}, \quad
\vec{v} = v_0\hat{x} - gt\hat{y}, \quad
\vec{r} = (x_0 + v_0t, y_0 - \frac{1}{2}gt^2) 
$$
Avremo allora che il tempo impiegato a toccare terra sarà uguale nei due sistemi di riferimento sarà identico:
$$ t^* = \sqrt{\frac{2y_0}{g}}$$
mentre le velocità saranno diverse:
$$ \vec{v'}(t^*) = -\sqrt{2gy_0}\hat{y}, \quad \vec{v}(t^*) = v_0\hat{x} - \sqrt{2gy_0}\hat{y} $$
\subsection{Angolo della pioggia}
Calcoliamo l'angolo ottimale a cui va tenuto un ombrello se ci sposta con velocità $v_u$ di 3m/s
e la pioggia cade a $v_p$ di 8m/s. Applicando le trasformazioni avremo che:
$$ \vec{v_p'} = \vec{v_p} - v_u\hat{x}$$
L'angolo di $\vec{v_p}'$, ovvero la velocità della pioggia nel sistema di riferimento dell'uomo, sarà allora:
$$ \tan^{-1}{\frac{v_u}{v_p}} = 20.55^\circ$$
\subsection{Barca con velocità relativa}
Una barca parte dall'origine e si dirige verso la riva opposta di un fiume di larghezza $L$ 
dove la corrente scorre con velocità $V_A$ costante. La barca, di per sé, si muove con velocità $V_B$ costante
verso la riva, mantenendosi sempre perpendicolare al fiume. Calcoliamo allora il punto $K$ dove la barca arriva a riva.
Avremo un sistema di riferimento $S'$ solidale all'acqua, dove:
$$ \vec{v_0} = -V_A\hat{x}, \quad \vec{v'} = V_B\hat{y} $$
$$ \vec{v} = \vec{v'} + \vec{v_0} = (V_A, V_B) $$
Il tempo impiegato a toccare riva sarà:
$$ t = \frac{L}{V_B} $$
E avremo quindi che il punto $K$ sarà dato da:
$$ \vec{K} = (V_At, V_Bt) = (V_At, L) $$
\\
Mettiamo che il punto $K$ si trovi oltre una pericolosa cascata: avremo quindi bisogno di opporci alla corrente
per raggiungere un secondo punto, detto $H$, più vicino. La velocità $V_B$ della barca non sarà quindi perpendicolare
alla riva, ma formerà con essa un certo angolo $\theta$:
$$ \vec{v'} = \vec{V_B} = -V_B\cos{\theta}\hat{x} + V_B\sin{\theta}\hat{y} $$
da cui:
$$ \vec{v} = (V_A - V_B\sin{\theta}, V_B\cos{\theta}) = (0, V_B\cos{\theta}) $$
Il tempo di raggiungimento della riva sarà allora:
$$ t = \frac{L}{V_B\cos{\theta}} $$
Abbiamo chiaramente che:
$$ \frac{L}{V_B} \leq \frac{L}{V_B\cos{\theta}} $$
ovvero, opporsi alla corrente significa evitare la cascata, ma anche impiegare un tempo maggiore a raggiungere
la riva!
\section{Sistemi di riferimento uniformemente accelerati}
Prendiamo adesso in considerazione un sistema di riferimento mobile $SM$ in moto con velocità $\vec{v_t}$ e accelerazione
costante $\vec{a_t}$. Il moto d $SM$ va considerato come puramente traslatorio, i suoi assi non ruotano. La relazione
fra le posizioni diventa quindi:
$$ \vec{r} = \vec{r'} + \vec{r}_{OO'}(t) $$
che deriveremo in:
$$ \vec{v} = \vec{v'} + \vec{v_t}, \quad \vec{v_t} = \frac{d\vec{r}_{OO'}}{dt} $$
$$ \vec{a} = \vec{a'} + \vec{a_t}, \quad \vec{a_t} = \frac{d\vec{v_t}}{dt} $$
$$ \vec{a} = \vec{a'} + \vec{a_r} $$
Ovvero:
$$ 
\left\{\begin{array}{l}
    \vec{r} = \vec{r'} + \vec{r}_{OO'}(t) \cr \\
  \vec{v} = \vec{v'} + \vec{v_t} \cr \\
  \vec{a} = \vec{a'} + \vec{a_T}
\end{array}\right.
$$
Si noti che nel sistema di riferimento accelerato non vale l'inerzia. Sein un sistema di riferimento
inerziale si ha:
$$ \sum \vec{F}  = m\vec{a} = m(\vec{a} + \vec{a_r})$$
nel sistema accelerato questa legge non vale:
$$ ma' = \sum \vec{F} - m \vec{a_T} $$
E si osserva una forza, detta forza apparente, pari a $\vec{F_a} = -m\vec{a_T}$:
$$ \sum \vec{F} -m\vec{a_T} = \sum(\vec{F}+\vec{F_{app}})$$
Una forza apparente non è data da l'interazione fra due corpi, ma dall'inerzia. Per essa non vale il
terzo principio della dinamica (azione e reazione). La stessa cosa non si osserva in un sistema inerziale
(e qui dovrebbe diventare chiaro) in quanto:
$$ m\vec{a_T} = 0 \Rightarrow \vec{F_{app}}, \quad m\vec{a} = \sum \vec{F} + 0$$
\subsection{Treno in frenata}
Consideriamo lo stesso treno di prima, che dopo un tratto a velocità costante $v_0$, inizia a frenare e si arresta
in tempo $t$. Un passeggero di massa $m$, che resta seduto durante la frenata, percepisce una certa forza
apparente, pari a $ma_T$, dove $a_T$ è l'accelerazione del treno in frenata, data da:
$$ a_T = \frac{v_0}{t} $$
Quando il treno si ferma, rispetto al treno il viaggiatore è fermo, ergo $|\vec{a}| = 0$, e:
$$ m\vec{a'} = \sum (\vec{F} + \vec{F_{app}}) = 0 \Rightarrow \sum \vec{F} = -\vec{F_{app}} = m\vec{a_T} $$
\subsection{Cartellina sul tettuccio}
Un signore sbadato dimentica sul tettuccio della sua automobile una cartellina, prima di partire di moto rettilineo uniformemente
accelerato. Vogliamo calcolare il coefficiente di attrito statico minimo fra la cartellina e il tettuccio dell'automobile
necessario a non farla scivolare via. Nel sistema di riferimento della cartellina, avremo una forza apparente data
dall'accelerazione dell'automobile:
$$ F_{app} = -m\vec{a_T} $$
E una forza d'attrito:
$$ F_s = mg \mu_s $$
diretta nella direzione dello spostamento dell'automobile. La condizione di equilibrio sarà raggiunta quando
queste due forze (di cui una apparente!) si eguaglieranno in modulo (in modo da annullarsi):
$$ m\vec{a_T} = mg \mu_s, \quad \mu_s = \frac{\vec{a_T}}{g} $$
Mettiamo che il coefficiente di attrito effettivo $\mu_s$ sia minore del coefficiente ottimale appena calcolato. In
questo caso la cartellina scivola via opposta da una forza di attrito dinamico con coefficiente $\mu_d$. Possiamo
calcolare l'accelerazione subita dalla cartellina nel sistema di riferimento dell'automobile e di laboratorio (fermo rispetto
alla cartellina e all'automobile). Da terra vedremo che l'unica accelerazione a cui è sottoposta la cartellina
è quella dell'attrito dinamico:
$$ m\vec{a} = F_d = mg\mu_d $$
mentre dal punto di vista dell'automobile l'accelerazione è pari alla differenza fra l'accelerazione impressa dall'attrito
dinamico e quella propria dell'automobile stessa:
$$ m\vec{a'} = F_d - ma_T = mg\mu_d - ma_T $$
\subsection{Treno accelerato}
Un treno si muove sui binari di moto rettilineo uniformemente accelerato. In un certo istante, diciamo $t = 0$ (con treno
in partenza da fermo), viene lanciata una monetina con una certa velocità iniziale verso l'alto $v_0$. Ci poniamo il problema
di calcolare la distanza sul pavimento del treno a cui atterra la monetina, ovvero lo spazio percorso dalla monetina 
lungo l'asse $x$, nel sistema di riferimento del treno, una volta caduta nuovamente a terra. Impostando le relazioni:
$$ 
\left\{\begin{array}{l}
    \vec{r} = \vec{r'} + \vec{r}_{OO'}(t) \cr \\
  \vec{v} = \vec{v'} + \vec{v_t} \cr \\
  \vec{a} = \vec{a'} + \vec{a_T}
\end{array}\right.
$$
osserviamo che, sul sistema di riferimento del treno non è accelerato rispetto all'asse y, ergo sull'asse verticale
il moto uniformemente accelerato della monetina in caduta libera si svolge in modo identico a quanto si sarebbe svolto
in un sistema di riferimento inerziale. A questo punto la velocità percorsa dalla monetina sarà uguale alla distanza percorsa
dal treno nel tempo che la monetina impiega a cadere nuovamente a terra (questo rispetto al treno, avessimo voluto calcolare
la distanza rispetto a terra avremmo dovuto tenere conto della velocità iniziale della monetina lungo la direzione
di spostamento del treno, che potrebbe essere stata maggiore di zero per $t > 0$). Possiamo quindi calcolare il
tempo di caduta della monetina sull'asse verticale:
$$ h = v_0t-\frac{1}{2}gt^2, \quad t_1 = 0, \quad t_2 = \frac{2v_0}{g}$$
dove le due soluzioni della quadratica rappresentano il punto di partenza a $t=0$, quando la monetina è chiaramente al livello del pavimento,
e il punto di atterraggio che a noi interessa. A questo punto calcoliamo la distanza percorsa dal treno nel tempo calcolato (che potremmo
chiamare $t*$):
$$ d = \frac{1}{2}at^{*2} = \frac{2av_0^2}{g^2} $$
dove si è chiaramente assunto spostamento e velocità iniziale del treno come nulle. \\
Potremo allora porci il problema, leggermente più complesso, di ricavare la traiettoria percorsa dalla monetina
nel sistema di riferimento del treno, come in quello di laboratorio (che non sono uguali!):
\begin{itemize}
  \item \textbf{Traiettoria nel S.R. di laboratorio: } possiamo applicare le relazioni impostate prima,
    con le leggi orarie dei moti verticali ed orizzontali della monetina:
    $$ \vec{r} = \vec{r'} + \vec{r}_{OO'}(t), \quad \vec{r}_{OO'}(t) = \vec{r}_{OO'}(0) + \vec{v}_{OO'}(0) + \frac{at^2}{2}\hat{x} $$
    $$ \vec{r'} = -\frac{at^2}{2}\hat{x} + (-\frac{gt^2}{2} + v_0t)\hat{y} $$
    da cui:
    $$ \vec{r} = (-\frac{gt^2}{2} + v_0t)\hat{y} + \vec{r}_{OO'}(0) + \vec{v}_{OO'}(0) $$
    che è l'equazione (simmetrica) di una parabola.
  \item \textbf{Traiettoria nel S.R. del treno: } nel sistema di riferimento del treno, potremo assumere $g'$ come la nostra forza di gravità,
    composta non solo dalla componente verticale $g\hat{y}$, ma anche dalla componente orizzontale data dall'accelerazione del treno $-a_T\hat{x}$.
    Modulo e angolo di questa forza saranno rispettivamente:
    $$  |g'| = \sqrt{a^2 + g^2}, \quad \theta = \tan^-1(\frac{a}{g})$$
    Il risultato, come nel caso precedente, sarà una parabola, ma questa volta asimmetrica: l'asse sarà infatti
    parallelo alla direzione della "nuova" accelerazione di gravità percepita all'interno del treno, e avrà quindi angolo:
    $$ \theta = \tan^-1(\frac{a}{g}) $$
\end{itemize}
Può essere interessante considerare la traiettoria della monetina nel caso il treno si muova di moto rettilineo uniforme anzichè uniformemente accelerato.
\begin{itemize}
  \item \textbf{Traiettoria nel S.R. di laboratorio: } visto che la monetina viene lanciata con una velocità iniziale pari a quella del
    treno, si ha dalle relazioni galileiane che la velocità sull'asse x è:
    $$ v_x = v'_x + v_0, \quad v'_x = v_0 \Rightarrow v_0 = 0 $$
    ergo la monetina resta ferma sull'asse x, parte da un punto sul pavimento del treno e atterra esattamente al solito posto.
  \item \textbf{Traiettoria nel S.R. del treno: } il caso sarà qui analogo agli esempi visti prima, la legge oraria sarà infatti:
    $$ \vec{r} = (v_0t, v_it - \frac{1}{2}gt^2) $$
    che è la traiettoria di una parabola simmetrica (come conseguenza, qui anche più chiara, del teorema della funzione implicita). 
\end{itemize}
\end{document}
