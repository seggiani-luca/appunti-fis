\documentclass[a4paper,12pt]{article}

\usepackage[french,italian]{babel}
\usepackage[T1]{fontenc}
\usepackage[utf8]{inputenc}
\frenchspacing 
\title{Appunti Fisica I}
\author{Luca Seggiani}
\date{26 Febbraio 2024}

\begin{document}
\maketitle

\section{Introduzione al corso}
\textit{"Che cos'è la fisica?"} La fisica è la disciplina che si occupa della descrizione quantitativa dei
fenomeni naturali.
Fondamentale alla fisica è il metodo scientifico, che si applica attraverso:

\begin{itemize}
  \item Osservazione del fenomeno
  \item Realizzazione di misure
  \item Formulazione di un'ipotesi
  \item Sperimentazione dell'ipotesi
  \item Formalizzazione di una teoria (o formulazione di una nuova ipotesi)
\end{itemize}

per la realizzazione di misure, è necessaria la scelta di apposite:

\section{Unità di misura}
Le unità di misura permettono di confrontare grandezze fisiche a campioni scelti prima della misurazione. Si
possono fare le seguenti distinzioni:

\textbf{Grandezze scalari e vettoriali}
\begin{itemize}
  \item Grandezze scalari:
    grandezze che possono essere rappresentate da semplici numeri reali (e.g. temperatura, massa, tempo...)
  \item Grandezze vettoriali:
    grandezze che vengono rappresentate da vettori sul piano o nello spazio (e.g. spazio, accelerazione, forza...)
\end{itemize}

\newpage

\textbf{Grandezze intensive ed estensive}
\begin{itemize}
  \item Grandezze intensive:
    grandezze che non dipendono dall'estensione fisica di un corpo ma dalle proprietà delle sostanze che lo
    compongono
  \item Grandezze estensie:
    grandezze che dipendono dall'estensione fisica di un corpo
\end{itemize}

\textbf{Grandezze fondamentali e derivate}
\begin{itemize}
  \item Grandezze fondamentali:
    grandezze che vengono definite indipendentemente da altre
  \item Grandezze derivate:
    grandezze che si definiscono sulla base delle grandezze fondamentali
\end{itemize}

Le grandezze fondamentali (e le loro unità di misura nel S.I.) sono:
\begin{center}
\begin{tabular}{|c|c|c|}
  Grandezza & U. D. M. & Simbolo \\
  \hline
  Spazio & Metro & m \\
  \hline
  Massa & Kilogrammo & kg \\
  \hline
  Tempo & Secondo & s \\ 
  \hline
  Temperatura & Kelvin & K \\
  \hline
  Intensità di corrente & Ampere & A \\
  \hline
  Intensità luminosa & Candela & Ca \\
  \hline
  Quantita di materia & Mole & mol \\
\end{tabular}
\end{center}

\section{Analisi dimensionale}
In fisica le equazioni devono essere dimensionalmente coerenti, ovvero le grandezze espresse in entrambi i lati
dell'equazione devono essere omogenee fra di loro (avere la stessa unità di misura)
Per verificare la correttezza di un'equazione si può effettuare l'analisi dimensionale, ad esempio:
$$ x = \frac{1}{2}at^2 $$ (spostamento $x$ di un corpo sottoposto ad accelerazione $a$ in tempo $t$) diventerà:
$$ [L] = \frac{[L]}{[t^2]}[t^2] $$ che è chiaramente vero. \\
(Nota: i coefficienti dei termini non figurano nell'analisi dimensionale)

\newpage

\section{Errori, precisione ed accuratezza}
L'errore misura la discrepanza fra la grandezza effettiva che vogliamo misurare e il valore restituito dallo
strumento di misura utilizzato. Esempio: un righello con riportati centimetri e millimetri restituirà valori
del tipo: $ 15.2cm \pm 1mm $, ovvero 15 centimetri e 2 millimetri con un'errore di 1 millimetro.

Di una misurazione ottenuta da uno strumento possiamo conoscere la precisione e l'accuratezza:
\begin{itemize}
  \item Accuratezza:
    L'accuratezza di una misura rappresenta la differenza (quindi l'errore) fra la misura e la grandezza effettiva
  \item Precisione:
    La precisione di una misura è data dal numero delle sue cifre significative 
\end{itemize}

Le comuni operazioni algebriche modificano la precisione di una misura in tale modo:
    \begin{itemize}
      \item Somma / Differenza:
        La precisione di una somma o una differenza è data dal numero minore di posizioni decimali occupate degli operandi
      \item Prodotto / Rapporto:
        La precisione di un prodotto o di un rapporto è data dal numero minore di cifre significative degli operandi
    \end{itemize}


\end{document}
