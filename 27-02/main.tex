\documentclass[a4paper,12pt]{article}

\usepackage[french,italian]{babel}
\usepackage[T1]{fontenc}
\usepackage{gensymb}
\usepackage{amsmath}
\usepackage[utf8]{inputenc}
\frenchspacing 
\title{Appunti Fisica I}
\author{Luca Seggiani}
\date{27 Febbraio 2024}

\begin{document}
\maketitle

\section{Sistemi di riferimento}
Criterio generale per la scelta di un sistema di riferimento è la semplicità (ad esempio, è conveniente
scegliere sistemi con basi fra loro ortogonali e levogiri). Vediamo alcuni esempi di sistemi di riferimento
comuni su una retta, su curve parametriche, sul piano e nello spazio:

\section{Sistemi di riferimento monodimensionali}
Su una determinata retta $r$, è sufficiente scegliere un'origine $O$ e una direzione di scorrimento della 
retta. Avremo quindi:
$$ P(x) $$
sulla singola coordinata $x$ per indicare qualsiasi punto su $r$. Il concetto si estende facilmente a curve
parametriche di qualsiasi tipo, assumendo la nostra coordinata (adesso $t$) come fattore di scorrimento sulla 
curva ($C$):
$$ C(t) = (C_x(t), C_y(t), C_z(t)) $$

\section{Sistemi di riferimento bidimensionali}
Su un piano possiamo definire i due sistemi di riferimento comuni:
\begin{itemize}
  \item Coordinate cartesiane:
    individuo univocamente ogni punto con una coppia ordinata $ (x, y) $ (ascissa e ordinata)
  \item Coordinate polari:
    allo stesso modo, individuo ogni punto con l'angolo $\theta$ all'origine e la distanza $\rho$ dalla stessa.
\end{itemize}

Si può agilmente passare da coordinate cartesiane a polari e viceverso in questo modo:
\begin{itemize}
  \item Polari $\rightarrow{}$ cartesiane:
    notato che $\rho$ è l'ipotenusa di un triangolo rettangolo di angolo $\theta$, ho;
    $$ x = \rho \cos{\theta} \quad y = \rho \sin{\theta} $$
  \item Cartesiane $\rightarrow{}$ polari:
    ottengo $\rho$ con pitagora:
    $$ \rho = \sqrt{x^2 + y^2} $$
    e $\theta$ applicando:
    $$ \theta = \arctan{\frac{y}{x}} $$
    ricordando il quadrante di appartenenza di $(x, y)$ (l'inversione di $\tan(x)$ comporta la restrizione
    del dominio a $[-\frac{\pi}{2}, \frac{\pi}{2}]$)
\end{itemize}

\section{Sistemi di riferimento tridimensionali}
Come per i sistemi bidimensionali, in 3 dimensioni abbiamo:
\begin{itemize}
  \item Coordinate cartesiane:
    individuo univocamente ogni punto con una tripla ordinata $ (x, y, z) $ (asse x, y, e z)
  \item Coordinate cilindriche:
    come per le coordinate polari, definisco un $ r > 0 $ distanza dall'origine, un'angolo $\theta$ sull'asse z,
    e un'elevazione $z$ sempre sull'asse z, ottenendo la tripla $ (r, \theta, z) $.
    Posso convertire agevolmente in coordinate cartesiane attraverso:
    $$ x = r \cos{\theta} \quad
    y = r \sin(\theta) \quad
    z = z $$
    e viceversa:
    $$ z = z \quad
    r = \sqrt{x^2 + y^2} \quad
    \theta = \arctan{\frac{y}{x}} $$
  \item Coordinate sferiche:
    questa volta scelgo due angoli, $\phi \in [0, 2\pi]$ sull'asse z e $\theta \in [0, \pi]$ sulla perpendicolare del segmento che forma 
    angolo $\phi$ con l'asse x (che deciderà l'elevazione del punto), e sempre una distanza $r$ dall'origine.
    Questa volta dovrò usare le formule di conversione:
    $$ z = r \cos{\theta} \quad
    x = r \sin{\theta} \cos{\phi} \quad
    y = r \sin{\theta} \sin{\phi} $$
    e viceversa:
    $$ \cos{\theta} = \frac{z}{\sqrt{x^2 + y^2 + z^2}} \quad
    r = \sqrt{x^2 + y^2 + z^2} \quad
    \tan{\phi} = \frac{y}{x} $$
    Si può notare che questo sistema è lo stesso delle coordinate geografiche, sebbene con intervalli di angoli
    differenti:
    \begin{itemize}
      \item la longitudine corrisponde a $\phi$ con $\phi \in [-180^{\circ}, 180^{\circ}] $, mentre la latitudine
        corrisponde a $\theta$ con $\theta \in [-90^{\circ}, 90^{\circ}] $
    \end{itemize}
\end{itemize}

\section{Sommatorie ed integrali di vettori}
Notiamo brevemente la differenza fra le forme:
$$ \textrm{(1)} \quad | \sum_{i = 1}^n \vec{l}_i| \quad \textrm{e} \quad \textrm{(2)} \quad \sum_{i = 1}^n | \vec{l}_i | $$
con $\vec{l}_i$ un'insieme di vettori in un determinato spazio. Se la forma (1) tiene conto solamente della
somma finale dei vettori (prendi e.g. la posizione di un corpo traslato dai suddetti), la forma (2) restituisce
invece la somma di ogni singolo vettore, qualsiasi la sua direzione. Allo stesso modo, su profili (prendo $y_1$) non discreti:

$$ \textrm{(3)} \quad |\int_{y_1} d\vec{l} | \quad \textrm{e} \quad \textrm{(4)} \quad \int_{y_1} | d\vec{l} | $$
la forma (3) resituisce la sola somma integrale dei differenziali $d\vec{l}$, mentre la forma (4) considera lo
spostamento (in qualsiasi direzione) totale di $\vec{l}$.

\section{Prodotto scalare e prodotto vettoriale}

Si riportano adesso le formule per il prodotto scalare e il prodotto vettoriale ottenute sfruttando i versori 
$\hat{i}$, $\hat{j}$ e $\hat{k}$. prenderemo in esame due vettori, $a = (a_x, a_y, a_z)$ e $b = (b_x, b_y, b_z)$.
\par\medskip
\textbf{Prodotto scalare} \\
$$ (\hat{i}A_x + \hat{j}A_y + \hat{k}A_z) \cdot (\hat{i}B_x + \hat{j}B_y + \hat{k}B_z) = $$ $$
(\hat{i})^2A_xB_x + \hat{i}\hat{j}A_xB_y + \hat{i}\hat{k}A_xB_z + $$ $$
\hat{j}\hat{i}A_yB_x + (\hat{j})^2A_yB_y + \hat{j}\hat{k}A_yB_z + $$ $$
\hat{k}\hat{i}A_zB_x + \hat{k}\hat{j}A_zB_y + (\hat{k})^2A_zB_z $$
visto che il prodotto scalare di due vettori perpendicolari (quindi di due versori diversi fra di loro) vale 0,
e che il prodotto scalare di un vettore per se stesso vale se stesso al quadrato (su vettori di modulo 1), possiamo
riscrivere come:
$$
A_xB_x+A_yB_y+A_zB_z
$$
che è la definizione di prodotto scalare.

\par\medskip
\textbf{Prodotto vettoriale} \\
$$ (\hat{i}A_x + \hat{j}A_y + \hat{k}A_z) \times (\hat{i}B_x + \hat{j}B_y + \hat{k}B_z) = $$ $$
(\hat{i})^2A_xB_x + \hat{i}\hat{j}A_xB_y + \hat{i}\hat{k}A_xB_z + $$ $$
\hat{j}\hat{i}A_yB_x + (\hat{j})^2A_yB_y + \hat{j}\hat{k}A_yB_z + $$ $$
\hat{k}\hat{i}A_zB_x + \hat{k}\hat{j}A_zB_y + (\hat{k})^2A_zB_z $$
di nuovo, visto che il prodotto vettoriale fra due vettori paralleli vale 0, e il prodotto vettoriale fra due
versori unitari vale quanto il loro perpendicolare secondo la regola della mano destra in un sistema di riferimento
levogiro, possiamo riscrivere come:
$$
\hat{i}(A_yB_z - A_zB_y) + \hat{j}(A_zB_x - A_xB_z) + \hat{k}(A_xB_y - AyB_x)
$$
che è la definizione di prodotto vettoriale sui versori $\hat{i}$, $\hat{j}$ e $\hat{k}$. Si noti il fattore $\hat{j}$,
che ha un segno negativo (distribuito sui fattori). Questo fatto viene dal sistema dalle proprietà del prodotto
vettoriale su un sistema levogiro. Si ricorda inoltre una mnemonica per il prodotto vettoriale come determinante della matrice:
\[
\begin{pmatrix}
  \hat{i} & \hat{j} & \hat{k} \\
    A_x & A_y & A_z \\
    B_x & B_y & B_z
  \end{pmatrix}
  \]
\end{document}
