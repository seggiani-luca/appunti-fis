\documentclass[a4paper,12pt]{article}

\usepackage[french,italian]{babel}
\usepackage[T1]{fontenc}
\usepackage[utf8]{inputenc}
\frenchspacing 
\title{Appunti Fisica I}
\author{Luca Seggiani}
\date{25 Marzo 2024}

\usepackage{amsmath}

\begin{document}
\maketitle
\textbf{Lavoro svolto nel moto parabolico} \\
Prendiamo l'esempio del moto parabolico. La forza peso compirà un lavoro in qualsiasi punto della parabola pari a:
$$ L_p = -mg(y_f - y_i) = -mgy $$
$$ \Delta K = \frac{1}{2}mv_f^2 - \frac{1}{2}mv_i^2 = \frac{1}{2}m(v_y^2-v_0^2\sin^2{\theta})$$
Eguagliando:
$$ \Delta K = -mgy = \frac{1}{2}m(v_y^2-v_0^2\sin^2{\theta})$$
Possiamo applicare il teorema delle forze vive:
$$ -2gy=v_y^2-(v_0\sin{\theta})^2 \Rightarrow v_y^2 = (v_0\sin{\theta})^2 - 2gy$$
che con quanto trovato attraverso la cinematica:
$$ 2a_y(y(t)-y_0)=v_y(t)^2-v_{0y}^2$$
\par\smallskip
\textbf{Lavoro della forza elettrostatica o gravitazionale} \\
Possiamo dire, per le forze che dipendono dall'inverso del quadrato della distanza:
$$ \vec{F} = -\frac{K}{r^2}\hat{r} $$
attrattiva con $k>0$, repulsiva con $k<0$, che il lavoro svolto è:
$$ L = \int_i^f \vec{F}\cdot d\vec{s} = -k\int_i^f \frac{\hat{r} \cdot d\vec{s}}{r^2} = -k\int_{r_i}^{r_f}\frac{dr}{r^2} = k(\frac{1}{r_f} - \frac{1}{r_i})$$

\section{Potenza}
La potenza è la velocità con cui viene sviluppata una forza:
$$ P = \vec{F} \cdot \vec{v} = \frac{L}{t}$$
La potenza è una quantita scalare e si misura in $\frac{\mathrm{J}}{\mathrm{s}} = \mathrm{W}$ (watt).
Possiamo dimostrare la formula sopra riportata con:
$$ P = \frac{dL}{dt} = \frac{d}{dt}\int\vec{F}\cdot d\vec{r}= \frac{d}{dt}(\vec{F}\cdot\vec{v})dt = \vec{F}\cdot\vec{v} $$

\section{Forze conservative}
Una forza $\vec{F}$ è detta conservativa se il lavoro che svolge non dipende dal percorso scelto $\gamma$, ma solamente dagli estremi.
In simboli:
$$ L = \int_{\vec{r_0} (\gamma)}^{\vec{r_1}} \vec{F} \cdot d\vec{L} = \int_{\vec{r_0}}^{\vec{r_1}} \vec{F} \cdot d\vec{L}$$
non dipende dal $\gamma$, ma solamente da $r_0$ e $r_1$.
Esempi di forze conservative sono la gravità, la forza elettrostatica e la forza elastica. Esempi di forze non conservaative sono
invece l'attrito dinamico e l'attrito viscoso.
\par\smallskip
Si consideri una forza $\vec{F}$. Per ogni linea chiusa $\gamma$ si ha:
$$ \oint_\gamma \vec{F} \cdot d\vec{l} = 0 \Rightarrow L = \int_{\vec{R_0}}^{\vec{R_1}}\vec{F} \cdot d\vec{l}$$
Ciò si dimostra prendendo sul percorso chiuso due linee arbitrarie $\gamma_1$ e $\gamma_2$, con gli estremi coincidenti:
$$ \oint_\gamma \vec{F}\cdot d\vec{l} = 0 = \int_{\vec{R_0} (\gamma_1)}^{\vec{R_1}} \vec{F}\cdot d\vec{L} + \int_{\vec{R_1} (\gamma_2)}^{\vec{R_0}} \vec{F}\cdot d\vec{L} = \int_{\vec{R_0} (\gamma_1)}^{\vec{R_1}} \vec{F}\cdot d\vec{L} - \int_{\vec{R_0} (\gamma_2)}^{\vec{R_1}} \vec{F}\cdot d\vec{L} $$
ovvero:
$$ \int_{\vec{R_0} (\gamma_1)}^{\vec{R_1}} \vec{F}\cdot d\vec{L} = \int_{\vec{R_0} (\gamma_2)}^{\vec{R_1}} \vec{F}\cdot d\vec{L} $$

Quindi, se per una forza $\vec{F}$ su un percorso chiuso $\gamma$ si ha:
$$ \oint_\gamma \vec{F} \cdot d\vec{L} = 0 \Rightarrow \int_{\vec{R_0} (\gamma_1)}^{\vec{R_1}} \vec{F}\cdot d\vec{L} = \int_{\vec{R_0} (\gamma_2)}^{\vec{R_1}} \vec{F}\cdot d\vec{L} $$
Ovvero la forza dipende solamente dalla posizione iniziale e finale per qualsiasi $\gamma$.
In altre parole:
$$ L = \int_{\vec{R_0}}^{\vec{R_1}} \vec{F}\cdot d\vec{L} = f(\vec{R_0}, \vec{R_1}) $$
il lavoro è funzione di $\vec{R_0}$ e $\vec{R_1}$.

\section{Campo di forza e linee di forza}
Un campo di forze è un campo vettoriale, ovvero una funzione $\vec{F}(x,y,z)$ delle componenti $x, y, z$. Rappresenta
l'insieme di valori che una grandezza fiisca (in questo caso una forza) assume in un certo punto dello spazio.
Una \textbf{linea di forza} è una linea formata dall'inviluppo delle direzioni della forza nella regione del campo. La direzione
della forza è in ogni punto tangente alla linea di forza. Notiamo che le linee di forza non possono intersecarsi. Inoltre,
un campo di forza conservativo non può contenere linee chiuse. Una \textbf{superficie equipotenziale} è una superficie perpendicolare
in ogni punto alla direzinoe delle linee di forza.

\section{Energia potenziale}
Se una forza è conservativa allora esiste una funzione della posizione del punto materiale tramite la quale si può
esprimere il lavoro:
$$ L_{{A \rightarrow B}_1} = L_{{A \rightarrow B}_2} = L_{A\rightarrow B} = f(A,B) $$
Chiamiamo allora variazione di energia potenziale l'integrale:
$$ \Delta U = U(\vec{R}) - U(\vec{R_0}) = -L_{\vec{R_0} \rightarrow \vec{R}} = -\int_{\vec{R_0}}^{\vec{R}} \vec{F} \cdot d\vec{l} $$
al pari di qualche costante. L'energia potenziale è l'opposto del lavoro, ed entrambe si misurano in Joule.
Sulle superfici equipotenziali l'energia potenziale è costante.
\par\smallskip
\textbf{Energia potenziale gravitazionale nelle vicinanze della superficie terrestre} \\
Abbiamo che, vicino alla crosta terrestre, l'energia potenziale è maggiore tanto quanto aumenta la nostra altitudine, ovvero:
$$ U(Z) - U(Z_0) = L_{Z_0 \rightarrow Z} = -\int_{Z_0}^{Z}M\vec{g} \cdot d\vec{l} = MgZ - MgZ_0 = Mgh $$
$$ U(Z) = MgZ + \mathrm{const.} $$
Prendiamo $U(Z_0) = MGZ_0$:
$$ U(Z) = MgZ $$

\par\smallskip
\textbf{Energia potenziale di una molla} \\
Ricordiamo che per una molla:
$$ L_{x_i, x_f} = \int_{x_i}^{x_f} \vec{F} \cdot d\vec{r} = -\int_{x_i}^{x_f} kxdx = -\frac{k}{2}(x_f^2 - x_i^2) = \frac{k}{2}(l_i-l_0)^2-\frac{k}{2}(l_f-l_0)^2$$
Dove si nota che il lavoro dipende unicamente dalla lunghezza iniziale $l_i$ e finale $l_f$ della molla.
Possiamo cambiare variabile indicando con $x$ la lunghezza della molla:
$$ U(x) - U(x_i) = -L = \frac{k}{2}(x-l_0)^2-\frac{k}{2}(l_f-l_0)^2$$
Ponendo l'origine delle coordinate in $l_0$:
$$ U(x) - U(x_i) = -L = \frac{k}{2}(x-l_0)^2$$
\par\smallskip
\textbf{Conseguenze delle forze conservative} \\
Dalla definizione di differenza di energia potenziale deriva:
$$ U(\vec{R})-U(\vec{R_0}) = -\int_{\vec{R_0}}^{\vec{R}} \vec{F} \cdot d\vec{l} \Rightarrow U(\vec{R}) = U(\vec{R_0}) -\int_{R_0}^{R} \vec{F}\cdot d\vec{l} $$
Che sulle componenti diventa:
$$ U(x, y, z) = U(x_0, y_0, z_0) - \int_{(x_0, y_0, z_0)}^{(x,y,z)} (F_xdx + F_ydy + F_zdz) $$
da cui:
\begin{align*}
  F_x = - \frac{\partial U(x,y,z)}{\partial x} && F_y = - \frac{\partial U(x,y,z)}{\partial y} && F_z = - \frac{\partial U(x,y,z)}{\partial z}
\end{align*}
Che si riassume in:
$$ F = -\vec{\nabla} U$$
Ovvero, in qualsiasi punto del campo dell'energia potenziale possiamo ricavare su un intorno il vettore forza come gradiente.

\par\smallskip
\textbf{Modelli d'analisi basati sull'energia} \\
L'energia e il lavoro forniscono un modello d'analisi fisico spesso più conveniente di quello della cinematica.
Prendiamo in esempio il moto di un punto materiale sul piano inclinato: il corpo inizia il suo moto alla base
del piano (inclinato di angolo $\theta$) con una certa velocità $v_0$. Risale il piano per una certa distanza $l$, ostacolato dalla forza d'attrito
(con coefficiente di attrito dinamico $\mu_d$) e dalla forza gravitazionale $g$. Ci chiediamo quale sia la posizione massima
raggiunta dal punto materiale, e quale sia, una volta iniziata la sua discesa da dato punto, la velocità con cui ripassa dalla posizione
di partenza. \\
Vediamo le forze agenti sul punto materiale: la forza peso $mg$, diretta verso $-\hat{j}$, che possiamo dividere nelle due componenti
parellele e perpendicolari alla superficie del piano. Abbiamo poi la reazione vincolare del piano $N$, parallela alla componente
perpendicolare della forza peso che andrà ad annullare (e quindi perpendicolare alla superficie del piano). Abbiamo infine la forza di attrito,
proporzionale ad $|N|$, e diretta nella direzione opposta alla velocità $\vec{v_0}$ del corpo. 

\par\smallskip
\textbf{Spostamento massimo, approccio cinematico} \\
Possiamo impostare:
$$ F_p = -mg, \quad F_{p\perp} = -mg\cos{\theta}, \quad F_{o\parallel} = -mg\sin{\theta} $$
con $N = -F_{p\perp} = -mg\cos{\theta}$. A questo punto l'attrito sarà:
$$ F_{att} = -\mu_d|N| = -\mu_dmg\cos{\theta} $$
Abbiamo già detto che la forza gravitazionale nella direzione perpendicolare al piano viene annullata dalla reazione $N$, e quindi
l'unica componente che rimane a determinare l'accelerazione è:
$$ F = ma = \mu_dN + mg\sin{\theta} \Rightarrow a = g(\sin{\theta} + \mu_d\cos{\theta}) $$
Siamo nel caso del moto rettilineo uniformemente accelerato (nella direzione parallela al piano), e possiamo quindi impostare le leggi
orarie:

$$
\begin{aligned}
\left\{\begin{array}{l}
  s = v_0t + \frac{1}{2}at^2 \cr \\
  v = v_0 + at
\end{array}\right.
\end{aligned}
$$

Imponendo $v = 0$, come accadrà nel punto di altezza massima (e quindi di cambio direzione), avremo che:
$$ v_0 = g(\sin{\theta} + \mu_d\cos{\theta}) \Rightarrow t = \frac{v_0}{g(\sin{\theta} + \mu_d\cos{\theta})} $$
Da cui ricaviamo lo spostamento (chiamiamiolo $l = s$):
$$ l = v_0t + \frac{1}{2}at^2 = \frac{v_0^2}{2g(\sin{\theta} + \mu_d\cos{\theta})} $$
\par\smallskip
\textbf{Spostamento massimo, approccio energetico} \\
Impostiamo la differenza dell'energia cinetica, applicando il teorema delle forze vive:
$$ K_f - K_i = L_{att} + L_{mg} $$
Ricaviamo il lavoro svolto dalla forza di attrito:
$$ L_{att} = \int_0^l \mu_d mg dx = -\mu_d mg l\cos{\theta} $$
e dalla forza peso:
$$ L_{mg} = \int_0^l mg\sin{\theta} = mgl\sin{\theta} $$
Diciamo che la variazione di energia cinetica $\Delta K$ è uguale a a:
$$ \Delta K = -\frac{1}{2}mv^2 = -\mu_dmgl\cos{\theta} - mgl\sin{\theta} $$
da cui si ottiene:
$$ l = v_0t + \frac{1}{2}at^2 = \frac{v_0^2}{2g(\sin{\theta} + \mu_d\cos{\theta})} $$
\par\smallskip
Notiamo che attraverso entrambi gli approcci otteniamo risposte identiche, ma l'approccio energetico richiede
meno calcoli (sopratutto vettoriali!). Concludiamo ricavando attraverso entrambi gli approcci la velocità
di ritorno attraverso il punto di partenza.
\par\smallskip
\textbf{Velocità di ritorno, approccio cinematico} \\
In questo il corpo parte da distanza $l$ dall'inizio del piano,  con velocità iniziale nulla. Forza gravitazionale e
forza di attrito saranno discordi:
$$ F = ma = mg\sin{\theta} - \mu_dmg\cos{\theta}, \quad a = g(\sin{\theta} - \mu_d\cos{\theta}) $$
che sostituito nelle legge orarie:
$$
\begin{aligned}
\left\{\begin{array}{l}
  s = \frac{1}{2}at^2 \cr \\
  v = at
\end{array}\right.
\end{aligned}
$$
fornisce:
$$ t = \sqrt{\frac{2l}{a}}, \quad v^2 = 2la = 2lg(\sin{\theta} - \mu_d\cos{\theta}) $$
\par\smallskip
\textbf{Velocità di ritorno, approccio cinematico} \\
Impostiamo in modo analogo a prima il lavoro svolti dalla forza di gravità e di attrito sul punto materiale,
ponendoli adesso discordi:
$$ L_{att} = -\mu_d mgl\cos{\theta}, \quad L_{mg} mg\sin{\theta} $$
che possiamo eguagliare alla variazione di energia cinetica:
$$ \Delta K = \frac{1}{2}Mv^2 = -\mu_d mgl\cos{\theta} + mgl\sin{\theta} $$
Da cui si ricava subito $v$:
$$ v^2 = 2lg(\sin{\theta} - \mu_d\cos{\theta}) $$
\par\smallskip
Ancora una volta, i risultati coincidono (che è come dovrebbe essere!), e l'approccio energetico risulta più agile nei calcoli.
\end{document}
