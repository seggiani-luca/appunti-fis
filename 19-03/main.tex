\documentclass[a4paper,12pt]{article}

\usepackage[french,italian]{babel}
\usepackage[T1]{fontenc}
\usepackage[utf8]{inputenc}
\frenchspacing 
\title{Appunti Fisisca I}
\author{Luca Seggiani}
\date{19 Marzo 2024}

\begin{document}
\maketitle
\section{Moto circolare uniforme sul piano verticale}
Vediamo adesso il moto circolare uniforme, dal punto di vista delle forze, svolto su un piano verticale, ovvero
sotto l'effetto della forza di gravità. Supponiamo di avere un corpo che ruota su un giro della morte. Poniamo
$T$ come la reazione vincolare della superficie, che mantiene il corpo sulla traiettoria circolare. L'accelerazione 
del corpo corpo sarà allora:
$$ \vec{T} + m\vec{g} = m\vec{a} $$
o sulle componenti di un sistema di riferimento polare:
$$ -T + mg\cos{\theta} = ma_r = -m\frac{v^2}{r} $$
$$ -mg\sin{\theta} = ma_\theta = m \frac{dv}{dt} $$
Queste sono le  equazioni di un moto vario, ovvero dove l'accelerazione sulla direzione tangenziale del corpo varia.
Vediamo qual'è la velocità minima perchè il corpo raggiunga la posizione $\theta = \pi$, ovvero riesca a svolgere un
giro completo. Abbiamo da prima:
$$ g\cos{\theta} \leq \frac{v^2}{r}, \quad v_{min} = \sqrt{gr} $$
visto che $T$ sarà maggiore di zero nel caso di un contatto con la superficie della rampa.
\section{Forze fra corpi deformabili}
Vediamo ora le forze dovute all'interazione fra due corpi deformabili in contatto fra di loro. Questo tipo di forze si
dividono in:
\begin{itemize}
  \item Forze fra due corpi deformabili anelasticamente, spesso in modo permanente. In questo caso non
    esistono leggi, e si può solo applicare il terzo principio della dinamica.
  \item Forze fra due corpi deformabili elasticamente, che si possono modellizzare come molle.
\end{itemize}
\par\smallskip
\textbf{Forza elastica} \\
Vediamo nel dettaglio l'ultima categoria. Una mmolla con un certo coefficiente elastico $k$, sottoposta ad una sollecitazione
che riesce a spostarla di una certa $\Delta R$, esprime in direzione opposta una forza pari a:
$$ \vec{F_{el}} = -k\Delta R$$
La costante $k$ è detta costante della molla e si misura in $\frac{\mathrm{N}}{\mathrm{m}}$. Notiamo che le molle
che prenderemo in considerazione sono molle ideali, ovvero prive di massa e indeformabili (a $k$ costante).
\section{Moto armonico}
Vediamo una configurazione dove una massa collegata da una molla ad una parete oscilla attorno alla posizione di riposo.
Abbiamo che in ogni istante:
$$ ma = m \frac{d^2x}{dt^2} = -kx $$
ovvero
$$ \frac{d^2x(t)}{dt^2} = -\frac{k}{m}x(t) = -\omega^2x(t)$$
introducendo $ \omega = \sqrt{\frac{k}{m}}$, detta pulsazione della molla. \\
L'equazione differenziale ha soluzione:
$$ x(t) = Acos(\omega t + \phi) $$
dove $A$ è l'ampiezza, $\omega$ la pulsazione e $\phi$ la fase.
Posso ricavare poi la velocità:
$$ v(t) = -A\omega\sin{\omega t + \phi} $$
e l'accelerazione:
$$ a(t) = -A\omega^2\cos(\omega t + \phi) $$
\newpage
\textbf{Dimostrazione} \\
\begin{itemize}
  \item \textbf{Equazione differenziale del moto armonico} \\
prendiamo la legge elastica $\vec{F} = -kx$. Applicando la seconda legge
della dinamica, abbiamo che:
$$ m\vec{a} = -kx, \quad \vec{a} = -\frac{kx}{m}, \quad \frac{d^2x(t)}{dt^2} = -\frac{kx}{m} $$
da cui si ricava l'equazione differenziale ordinaria di secondo grado:
$$ \ddot{x}(t) +\frac{k}{m}x(t) = 0 $$
il polinomio caratteristico dell'equazione è:
$$ \lambda^2 + \frac{k}{m} = 0 $$
da cui si ottengono 2 soluzioni immaginarie coniugate, $\pm i\sqrt{\frac{k}{m}}$. L'integrale generale è quindi:
$$ x(t) = c_1e^{+i\frac{k}{m}} + c_2e^{-i\frac{k}{m}} $$
Conviene a questo punto chiamare $\sqrt{\frac{k}{m}} = \omega$, detta pulsazione dell'oscillatore armonico (impropiamente
velocità angolare). A questo punto potremo, tra l'altro, riscrivere la differenziale come:
$$ \ddot{x}(t) + \omega^2x(t) = 0 $$
Applicando la legge di Eulero all'integrale generale, avremo:
$$ x(t) = c_1\cos{\sqrt{\frac{k}{m} t}} + c_2\sin{\sqrt{\frac{k}{m}} t} = c_1\cos{\omega t} + c_2\sin{\omega t} $$
 \item \textbf{Forma cosinusoidale} \\
A questo punto riportiamo la formula nella forma $A\cos{\omega t + \phi}$, in funzione di ampiezza, pulsazione e fase.
Diciamo innanzitutto:
$$ A = \sqrt{c_1^2 + c_2^2} \Rightarrow \frac{A^2}{A^2} = \frac{c_1^2+c_2^2}{A^2}, \quad (\frac{c_1^2}{A})^2 + (\frac{c_2^2}{A})^2 = 1 $$
Notiamo la somiglianza con l'identità trigonometrica fondamentale $\sin^2{\theta} + \cos^2{\theta} = 1$. Possiamo infatti
avere che:
$$ \frac{c_1}{A} = \cos{\psi}, \quad \frac{c_2}{A} = \sin{\psi} \Rightarrow c_1 = A\cos{\psi}, \quad c_2 = A\sin{\psi} $$
per un qualsiasi angolo $\psi$. Sostituiamo quindi nell'integrale generale:
$$ x(t) = A\cos{\psi}\cos{\omega t} + A\sin{\psi}\sin{\omega t} = A(\cos{\psi}\cos{\omega t} + \sin{\psi}\sin{\omega t}) $$
su cui possiamo applicare la formula della sottrazione del coseno $\cos{(\alpha - \beta)} = \cos{\alpha}\cos{\beta} + \sin{\alpha}{\sin{\beta}}$:
$$ A(\cos{\psi}\cos{\omega t} + \sin{\psi}\sin{\omega t}) = A\cos{(\omega t - \psi)} $$
dove $ \cos{(\omega t - \psi)} = \cos{(\psi - \omega t)}$ (per la parità del coseno o per la commutatività della moltiplicazione
nella formula di sottrazione, fai un po' te). Troviamo allora $\psi$ in funzione di $c_1$ e $c_2$, accorgendoci di poter
ricavare dalle formule sopra riportate $\tan{\psi}$:
$$ \tan{\psi} = \frac{\sin{\psi}}{\cos{\psi}} = \frac{c_1}{A} \cdot \frac{A}{c_2} = \frac{c_2}{c_1} $$
definiamo, per convenienza, l'opposto di $\psi$ come $\phi = -\psi$. Questo ci permette di scrivere:
$$ x(t) = A\cos{(\omega t + \phi)} $$
Dalla disparità della tangente, avremo
che:
$$ \tan{\phi} = -\tan{(-\phi)} = -\tan{\psi} = \tan{-\psi} $$
$$ \phi = \arctan{(-\frac{c_2}{c_1})}, \quad \psi = \arctan{\frac{c_2}{c_1}} $$
  \item \textbf{Condizioni iniziali} \\
Diamo infine un significato fisico alle costanti $c_1$, e $c_2$, imponendo le condizioni iniziali dell'equazione differenziale.
Iniziamo dalla posizione al tempo 0, ovvero $x(t = 0)$:
$$ x(0) = c_1\cos{0} + c_2\sin{0} = c_1, \quad c_1 = x(0) $$
Deriviamo poi $x(t)$ per ricavare la velocità $\frac{d}{dt}x(t) = v(t)$:
$$ \frac{d}{dt}x(t) = -c_1\omega\sin{\omega t} + c_2\omega\cos{\omega t} $$
e quindi la velocità al tempo 0, ovvero $v(t = 0)$:
$$ v(0) = -c_1\omega\sin{0} + c_2\omega\cos{0} = c_2\omega, \quad c_2 = \frac{v(0)}{\omega} $$
chiamiamo $v(0) = v_0$ e $x(0) = x_0$ e sostituiamo nella formula cosinusoidale:
$$ A\cos{(\omega t + \phi)} = \sqrt{c_1^2 + c_2^2}\cos{(\omega t + \arctan{(-\frac{c_2}{c_1})})} $$
$$ A = \sqrt{c_1^2+c_2^2} = \sqrt{x_0^2 + (\frac{v_0}{\omega})^2}, \quad \phi = \arctan{(-\frac{c_2}{c_1})} = \arctan{(-\frac{v_0}{\omega x_0})} $$
$$ A\cos{(\omega t + \phi)} = \sqrt{x_0^2 + (\frac{v_0}{\omega})^2}\cos{(\omega t + \arctan{(-\frac{v_0}{\omega x_0})})} $$
\end{itemize}
\end{document}
