\documentclass[a4paper,12pt]{article}

\usepackage[french,italian]{babel}
\usepackage[T1]{fontenc}
\usepackage[utf8]{inputenc}
\frenchspacing 
\title{Appunti Fisica I}
\author{Luca Seggiani}
\date{2 Maggio 2024}

\begin{document}
\maketitle
\par\smallskip
\textbf{Potenziale elettrico generato da una sfera uniformemente carica} \\
Calcoliamo il potenziale elettrico generato da una sfera uniformemente carica di raggio $R$. Stabilito il raggio $R$ della sfera, calcoliamo innanzitutto il campo ad una distanza $r > R$ dalla sfera. 
Avevamo ottenuto che:
$$ -\nabla V = E $$
Questo è equivalente a dire:
$$ v = -\int E dr $$
sull'unità infinitesima di distanza $dr$. Potremo allora calcolare il campo della sfera, che sappiamo essere equivalente al campo generato da una carica puntiforme posta al centro della stessa:
$$ E = \frac{Kq}{r^3}r = \frac{Kq}{r^2}\hat{r} $$
Impostiamo adesso l'integrale. Quello che vogliamo immaginare è di prendere una carica, portarla da un punto ad infinita distanza (sapendo che $V(\infty) = 0$) fino ad $r$. La differenza potenziale
tra il punto di partenza e il punto d'arrivo sarà il potenziale in quel punto:
$$ V = -qk\int_\infty^r \frac{1}{r^2} dr = -kq(-r^{-1} + 0) = \frac{kq}{r} $$
\begin{itemize}
  \item[$\rightarrow$] \textbf{Calcolo alternativo:} possiamo altrimenti dire che il potenziale esterno attraverso l'"integrale indefinito":
$$ V = -qk\int \frac{1}{r^2} dr = -kq(-r^{-1}) = \frac{kq}{r} + c $$
dove il $c$ dipende dallo "0" scelto per il potenziale (che sappiamo avere significato solo in termini di variazione).
\end{itemize}
Possiamo calcolare il potenziale per una distanza $r < R$: calcoliamo innanzitutto il potenziale sulla superficie della sfera, imponendo $r=R$:
$$ r = R \Rightarrow V = \frac{kq}{R} $$
A questo punto, il potenziale all'interno della sfera sarà dato dal potenziale da infinito fino alla superficie della sfera più il potenziale dalla superficie della sfera fino al punto interno $r$, ovvero dai due integrali:
$$ V(r) =-\int_\infty^R \frac{kq}{r^2}dr - \int_R^r \frac{kq}{R^3}dr = kq\left(\frac{1}{r}\Big|^1_\infty - \frac{1}{R^3} \frac{r^2}{2}\Big|^r_R\right) = kq\left(\frac{1}{R} - \frac{1}{R^3}\left(\frac{r^2}{2} - \frac{R^2}{2}\right)\right) $$
$$ kq\left(\frac{1}{R} - \frac{1}{2}\frac{r^2}{R^3} + \frac{1}{2R}\right) = kq\left(\frac{1}{R} + \frac{1}{2R} - \frac{1}{2} \frac{r^2}{R^3}\right) = \frac{3}{2} \frac{kq}{R} - \frac{kqr^2}{2R^3} $$
Questa funzione equivale al frammento di parabola che ha massimo in $r = 0$ di $\frac{3}{2} \frac{kq}{R}$, e si congiunge con i rami del potenziale esterno nei punti $r = \pm R$ (la cui unione a suddetta funzione è fra l'altro differenziabile pure in quei punti).
\par\smallskip
\textbf{Potenziale elettrico generato da una sfera conduttrice carica} \\
Calcoliamo adesso il potenziale elettrico generato da una sfera carica, ma stavolta conduttrice. Il potenziale all'esterno della sfera ($r>R$) potrà essere trovato in modo analogo a prima:
$$ V = -qk\int_\infty^r \frac{1}{r^2} dr = -kq(-r^{-1} + 0) = \frac{kq}{r} $$
Per calcolare il potenziale all'interno della sfera, potremo a questo punto partiere dal calcolare il potenziale sulla superficie della sfera ($r=R$):
$$ r = R \Rightarrow V = \frac{kq}{R} $$
Il potenziale su qualsiasi punto interno alla sfera non potrà essere che uguale al potenziale sulla sua superficie, in quanto il potenziale all'interno di un conduttore è costante (così come il campo elettrico è nullo).
Avremo quindi che, per ogni $r <R$:
$$ r < R \Rightarrow V = \frac{kq}{R} $$
\section{Capacità}
La capacità è una grandezza che misura la quantità di carica accumulata su una certa differenza di potenziale all'interno di un condensatore:
$$ C = \frac{Q}{\Delta V} $$
La sua unità di misura è il Farad (F), equivalente a 1 Coulomb su 1 Volt: $ 1F = \frac{1C}{1V}$. Si noti che la capacità è una quantita (per definizione) sempre positiva. Si noti che la carica complessiva di un condensatore è nulla:
per capacità di un condensatore si intende comunemente la carica accumulata su una sola armatura. Possiamo considerare, a scopo di esempio, il caso più semplice di un condensatore: una singola carica sferica, con
densità superficiale di carica $\sigma$, che funge da prima armatura.
La seconda armatura sarà assimilabile ad un guscio sferico di raggio infinito. Potremo allora dire:
$$ C = \frac{Q}{\Delta V}, \quad Q = 4\pi R^2 \sigma, \quad \Delta V = k\frac{Q}{R} \Rightarrow C = \frac{R}{k} = 4\pi\epsilon_0 R$$
Notiamo come la capacità della sfera non dipende dalla sua carica, ne dalla differenza di potenziale che essa genera: dipende solo dal suo raggio, che una quantità estensiva. Questo resterà vero anche per i condensatori tradizionali, a lastre parallele.
\par\smallskip
\textbf{Condensatore come elemento circuitale} \\
Il condensatore (o capacitore) è un'importante elemento circuitale: un condensatore collegato ad un circuito che fornisce corrente continua accumulerà carica, che potrà poi essere rilasciata una volta scollegato il circuito di carica.
Questa energia viene usata nei modi più disparati: per realizzare memorie, flash di fotocamere, defibrillatori, ecc...
\par\smallskip
\textbf{Calcolo delle capacità} \\
Calcoliamo le capacità di diversi tipi di condensatore.
\begin{itemize}
  \item \textbf{Capacità di un condensatore a lastre parallele} \\
    Calcoliamo la capacità di un condensatore a lastre parallele. Si avrà che la carica su una singola armatura è data da:
    $$ Q = A\sigma$$
    Possiamo poi calcolare il potenziale assumendo il campo fra le due lastre come uniforme:
    $$ \Delta V = Ed = \frac{\sigma}{\epsilon_0}d $$
    usando il fatto che $E = \frac{\sigma}{\epsilon_0}$ in un condensatore, come già calcolato più volte.
    Possiamo allora ottenere la capacità:
    $$ C = \frac{Q}{\Delta V} = \frac{A\sigma}{\frac{\sigma}{\epsilon_0}d} = \frac{\epsilon_0A}{d} $$
    \par\smallskip
    \textbf{Nota sugli effetti di bordo} \\
    Quanto calcolato è effettivamente vero solo nel caso ideale, di un condensatore a lastre parallele di area infinita. O meglio, lo sarebbe su una \textit{sezione} di tale condensatore, cui capacità totale sarebbe sennò, anch'essa infinita.
    Questo è perche si vanno a trascurare i cosiddetti \textbf{effetti di bordo}: visto che il campo elettrico è conservativo, il campo ai bordi delle lastre del condensatore non potrà semplicemente svanire. Si avrà quindi una certa conformazione radiale
    delle linee di campo che introdurrà un'errore nei calcoli. Le specifiche di questo meccanismo verrano trattate più avanti.
  \item \textbf{Capacità di un condensatore sferico} \\
    Calcoliamo la capacità di un condensatore sferico, formato da una carica sferica di raggio $a$, e da un guscio sferico esterno ad essa di raggio $b$, con $a < b$. Le due distribuzioni avranno densità di carica superficiale $\sigma_1$ e $\sigma_2$, rispettivamente.
    Potremo allora impostare le cariche sulle superfici del condensatore come uguali ed opposte:
    $$ 4\pi a^2 \sigma_1 = q, \quad 4\pi b^2 \sigma_2 = -q$$
    Da questo potremo calcolare la differenza di potenziale, utilizzando l'equazione trovata prima:
    $$ V(r) = \frac{qk}{r} + c, \quad 0 < \Delta V = V(a) - V(b) = kq\left(\frac{1}{a} - \frac{1}{b}\right) $$
    Potremo quindi calcolare la capacità:
    $$ C = \frac{q}{\Delta V} = \frac{q}{qk\left(\frac{1}{a} - \frac{1}{b}\right)} = \frac{4\pi \epsilon_0 ab }{b - a} = \frac{ab}{k(b-a)} $$
    Notiamo come da questo risultato possiamo trovare la capacità del condensatore a lastre parallele: impostiamo $b$ come $a + \epsilon$ su una piccola variazione di distanza $\epsilon$.
    Avremo allora, con del calcolo un po' discutibile:
    $$ c= \frac{4\pi\epsilon_0 a (b+\epsilon)}{b+\epsilon-a}, \quad \lim_{a,b\rightarrow\infty} \frac{4\pi\epsilon a^2}{\epsilon} = \frac{\epsilon_0 A}{d} $$
    che è esattamente la capacità del condensatore piano, assunto $A = 4\pi a^2$ e $\epsilon = d$.
  \item \textbf{Capacità di un condensatore cilindrico} \\
    Calcoliamo la capacità di un condensatore cilindrico, come quello che forma un cavo coassiale. Consideriamo quindi due cilindri infinitamente lunghi di raggio $a$ e $b$, con $a < b$, quindi contenuti l'uno dentro l'altro.
    Possiamo impostare le cariche in modo analogo a prima, facendo una considerazione: sarà inutile calcolare la capacità lungo tutto il cavo (essa è semplicemente infinita). Svolgeremo allora il calcolo della capacità \textit{per unità
    di lunghezza}: $ C_h  = \frac{C}{h}$.
    $$ 2\pi a h \sigma_1 = q, \quad 2 \pi b h \sigma_2 = -q $$
    Calcoliamo allora la differenza di potenziale attraverso il teorema di Gauss. Troviamo quindi per primo il campo elettrico:
    $$ \Phi(E) = h 2\pi r E(r) = \frac{q}{\epsilon_0} = \frac{2\pi a h\sigma_1}{\epsilon_0}, \quad E(r) = \frac{a\sigma}{\epsilon_0 r} $$
    E integriamo su $r$, in modo da ottenere una funzione potenziale che rispetti $-\frac{\Delta V}{\partial r} = E(r)$:
    $$ V(r) = -\int E(r)dr = -\frac{a\sigma_1}{\epsilon_0}\ln{(r)} $$
    Infine, calcoliamo esplicitamente la differenza di potenziale da $a$ a $b$:
    $$ \Delta V = V_b - V_a = -\frac{a\sigma_1}{\epsilon_0} (\ln{(b)} - \ln{(a)}) =  = -\frac{a\sigma_1}{\epsilon_0}\left(\ln{\frac{b}{a}}\right) $$
    Ottenuto il potenziale, possiamo calcolare la capacità come:
    $$ C = \frac{Q}{\Delta V} = \frac{2\pi a h \sigma_1}{\frac{a\sigma_1}{\epsilon_0}\ln{(\frac{b}{a})}} = \frac{2\pi \epsilon_0}{\ln{(\frac{b}{a})}}h = \frac{h}{2k\ln{(\frac{b}{a})}} $$
    Quindi, la capacità su unità di lunghezza che stavamo cercando è:
    $$ \frac{C}{h} = \frac{2\pi\epsilon_0}{\ln{(\frac{b}{a})}} = \frac{1}{2k\ln{(\frac{b}{a})}} $$
\end{itemize} 
\par\smallskip
\textbf{Energia immagazzinata da un condensatore} \\
Mentre viene caricato, il condensatore accumula una certa quantità di energia: questa quantita di energia corrisponde al lavoro fatto da una certa forza elettromotrice (all'esempio quella di una batteria) per spostare
le cariche da un armatura all'altra del condensatore. La resistenza che il condensatore oppone a questa forza cresce linearmente con la carica che esso accumula. Si ha che in un campo uniforme:
$$ V(t) = E(t)d = \frac{\sigma(t)}{\epsilon_0}d $$
Il lavoro sarà allora:
$$ \mathcal{L} = \int \frac{\sigma(t)}{\epsilon_0}d\,dq = \int \frac{q(t)d\, dq}{A\epsilon_0} = \frac{d}{A\epsilon_0}\int q\, dq = \frac{d}{A\epsilon_0} \frac{q^2}{2} = \frac{1}{2} \frac{q^2}{\frac{A\epsilon_0}{d}} = \frac{1}{2} \frac{q^2}{C} $$
Notando che $ \frac{A\epsilon_0}{d}$ non è altro che $C$. A questo punto applichiamo $C = \frac{Q}{V}$:
$$ \mathcal{L} = \frac{1}{2} \frac{q^2}{C} = \frac{1}{2} \frac{C^2V^2}{C} = \frac{1}{2}CV^2  = U$$
che coincide con l'energia che un condensatore è in grado di immagazzinare. Esiste un'altra importante formulazione dello stesso risulato, tenendo conto che $C = \frac{A\epsilon_0}{d}$ e $V=Ed$ nel campo uniforme:
$$ U = \frac{1}{2} \left( \frac{\epsilon_0 A}{d} \right)(Ed)^2 = \frac{1}{2}\epsilon_0AE^2d$$
Possiamo quindi dividere per $Ad$, che non è altro che la regione dello spazio occupato dal campo elettrico del condensatore, per ottenere la \textbf{densità di energia}:
$$ \frac{U}{Ad} = u = \frac{1}{2} \epsilon_0 E^2 $$
\end{document}
