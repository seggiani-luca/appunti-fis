\documentclass[a4paper,12pt]{article}

\usepackage[french,italian]{babel}
\usepackage[T1]{fontenc}
\usepackage[utf8]{inputenc}
\frenchspacing 
\title{Appunti Fisica 1}
\author{Luca Seggiani}
\date{9 Aprile 2024}

\begin{document}
\maketitle
\section{Conservazione della quantità di moto}
Consideriamo un sistema a 2 masse, $m_1$ e $m_2$, connesse da una molla su un piano privo di attrito.
Estendendo la molla, i due punti materiali si allontaneranno fra di loro e la molla esprimerà sui due capi (e quindi
sui punti materiali stessi) due forze identiche in modulo ma in verso opposto. In simboli, visto che la derivata rispetto
al tempo della quantità di moto equivale alla forza, avremo che:
$$ \frac{d\vec{p_1}}{dt} = \vec{F_1}, \quad \frac{d\vec{p_2}}{dt} = \vec{F_2} \Rightarrow \vec{F_2} = -\vec{F_1} $$
Definendo il vettore quantità di moto totale del sistema $\vec{P}$:
$$ \vec{P} = \vec{p_1} + \vec{p_2}, \quad \frac{\vec{P}}{dt} = \frac{d\vec{p_1}}{dt} + \frac{d\vec{p_2}}{dt} = F_1 + F_2 = 0 $$
Ciò significa che la variazione di quantita di moto è nulla, ergo la quantità di moto non cambia nel tempo. Più
propriamente, se su un sistema di punti materiali agiscono solamente forze interne, o se la risultante delle forze esterne è
nulla (*), allora la quantità di moto $\vec{P}$ del sistema è un vettore costante nel tempo. Ricordando il teorema dell'impulso
(riportato negli appunti come principio cardinale dei sistemi):
$$ \frac{d}{dt}\vec{P_{CM}} = \sum \vec{R}^{(est)} $$
la condizione (*) diventa $\sum \vec{R^{(est)}} = 0$, che come vediamo già risultava in variazione nulla della quantità
di moto. \\
Questa legge di conservazione può essere applicata, più generalmente, in due casistiche:
\begin{itemize}
  \item La risultante delle forze esterne è nulla;
  \item Le forze esterne non sono nulle, ma esse sono limitate, e la durata dell'interazione è trascurabile
    ($t \rightarrow 0$).
\end{itemize}
Possiao inoltre applicare la conservazione su un solo asse, nel caso le condizioni precedenti non siano rispettate
su assi diversi. \\
\par\smallskip
\textbf{Applicazioni della conservazione della quantità di moto} \\
La conservazione della quantità di moto, assieme alla conservazinoe dell'energia, può essere utilizzata per
modellizzare semplicemente numerosi fenomeni fisici riguardanti l'interazione fra più corpi (si ricordano ad 
esempio gli urti). Vediamo un esempio interessante: una pallina di massa $m_1$ parte dalla cima di una superficie a profilo
semicircolare di massa $m_2$, situata su un piano senza attrito. Cadendo, la pallina spinge la superficie, così che al termine
della caduta entrambe saranno in moto rettilineo uniforme verso direzioni opposte. Si vuole calcolare le velocità finali di entrambi
i corpi. Iniziamo dalla conservazione dell'energia. L'energia potenziale della pallina sospesa sulla superficie nell'istante iniziale
verrà interamente convertita in energia cinetica sia della pallina che della superficie, ergo:
$$ m_1gh = \frac{1}{2}m_1v_1^2 + \frac{1}{2}m_2v_2^2 $$
Per eliminare le due incognite $v_1$ e $v_2$, potremo sfruttare il fatto che la conservazione della quantità di moto
valle sull'asse x del sistema, in quanto non intervengono forze esterne. Si ha quindi:
$$ m_1v_1 + m_2v_2 = p_i $$
dove $p_i$ è la quantità di moto nell'istante $t=0$, che possiamo subito impostare come nulla in quanto prima della
caduta entrambi i corpi sono in quiete. Si avrà quindi che:
$$ m_1v_1 +m_2v_2 = 0, \quad m_1v_1 = -m_2v_2 \Rightarrow v_1 = -\frac{m_2v_2}{m_1}$$
Sostituendo nella conservazione dell'energia, avremo:
$$ gh = \frac{1}{2}(\frac{m_2v_2}{m_1})^2+\frac{1}{2}\frac{m_2}{m_1}v_2^2 \Rightarrow v_2 = -\sqrt{\frac{m_1^2gh}{m_2(m_1+m_2)}}$$
Con ragionamenti simili avrà poi:
$$ v_2 = -\frac{m_1v_1}{m_2}, \quad gh = \frac{1}{2}v_1^2 + \frac{1}{2}\frac{m_2}{m_1}(\frac{m_2v_2}{m_1})^2 \Rightarrow v_1 = \sqrt{\frac{2m_2gh}{m_1+m_2}} $$
N.B.: i segni sono stati scelti in modo da rispettare la conservazione della quantità di moto. Si assume che la pallina venga
spinta verso destra, e la superficie verso sinistra.
\end{document}
