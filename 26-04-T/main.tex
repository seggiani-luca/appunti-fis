\documentclass[a4paper,12pt]{article}

\usepackage[french,italian]{babel}
\usepackage[T1]{fontenc}
\usepackage[utf8]{inputenc}
\frenchspacing 
\title{Appunti Fisica I}
\author{Luca Seggiani}
\date{26 Aprile 2024}

\usepackage{amsmath}

\begin{document}
\maketitle
\par\medskip
Vediamo alcune applicazioni del teorema di Gauss per il calcolo di flussi e campi elettrici:
\begin{itemize}
  \item \textbf{Flusso di un dipolo} \\
Calcoliamo il flusso generato da un dipolo su una superficie gaussiana ad esso circostante. Si ha che, su una superficie
gaussiana che comprende entrambe le cariche, tutte le linee di campo escono ed entrano nella superficie, ergo il flusso
totale vale 0:
$$ \Phi(\vec{E)} = 0 $$
Nel caso si prenda una superficie gaussiana che circonda solamente una delle cariche di dipolo, dovremo considerare tale come carica
puntiforme. Ricordiamo che, data la proporzionalità del campo elettrico $\vec{E}$ generato dal dipolo, e quella della superficie $S$ al crescere
del raggio:
$$ \vec{E} \propto \frac{1}{r^3}, \quad S \propto r^2 \Rightarrow \Phi(\vec{E}) \propto \frac{1}{r} $$
Ergo un eventuale flusso dovrebbe scalare come l'inverso del raggio.
N.B.: In realtà, non dovrebbe essere possibile applicare il teorema di Gauss ad un dipolo, in quanto non è possibile
trovare una superficie gaussiana che "racchiuda" l'interezza del campo. Il professore è quantomai ombroso sull'argomento, il risultato
viene riportato solo a scopo indicativo.
\item \textbf{Campo generato da un filo uniformemente carico con Gauss} \\
Possiamo usare il teorema di Gauss per calcolare il campo elettrico generato da un il filo uniformemente carico. 
Riprendiamo la nozione di densità lineare di carica:
$$ dQ = \lambda dl $$
su una qualche lunghezza infinitesima $dl$. La carica complessiva di un filo di lunghezza $h$ sarà allora: 
$$ Q=h\lambda $$
Vediamo poi che il sistema formato dal filo (disposto sull'asse z) è invariante a rotazione assiale. Il campo
deve quindi essere radiale. Prendiamo una superficie gaussiana cilindrica che contiene interamente il filo.
Impostiamo il flusso attraverso il teorema di Gauss:
$$ \Phi(\vec{E}) = \frac{h\lambda}{\epsilon_0} $$
e attraverso la definizione, tenendo conto del fatto che il flusso totale sarà la somma del flusso sulle basi $\Phi_B(\vec{E})$ e sull'area
laterale del cilindro $\Phi_L(\vec{E})$:
$$ \Phi(\vec{E}) = \Phi_B(\vec{E}) + \Phi_L(\vec{E}) $$
Notiamo quindi il campo lungo le basi è nullo, in quanto aventi vettori normali ortogonali alle linee di campo. La totalità del flusso
sarà quindi data dalla superficie laterale, che forma in ogni punto un angolo $\theta = 0$ con il campo, e restituisce quindi:
$$ \Phi(\vec{E}) = \int_L d\vec{A} \cdot d\vec{E}(\vec{r}) = \int_L d\vec{A}E(d) = E(d) 2\pi d h $$
dove $d$ è la distanza dal centro del filo dove si calcola il campo elettrico. Notiamo il passaggio da $\vec{E}$ a $E$ in solo modulo, possibile
in quanto il campo a una distanza $d$ dal centro (e quindi su una circonferenza) è costante.
Possiamo a questo punto eguagliare le due espressioni trovate per $\Phi(\vec{E})$:
$$ E(d)2\pi d h = \frac{h\lambda}{\epsilon_0} \Rightarrow E(d) = \frac{\lambda}{2\pi \epsilon_0 d} $$
Che è in accordo con quanto trovato precedentemente attraverso il principio di sovrapposizione.
\item \textbf{Campo generato da una sfera uniformemente carica} \\
Per calcolare il campo elettrico generato da una sfera uniformemente carica, possiamo sfruttare la sua simmetria radiale: in ogni punto posto a distanza
$r$ avremo campo costante. Applichiamo allora la densità (volumetrica) di carica $\rho$:
$$ dq = \rho dV, \quad Q = \rho V = \rho \frac{4}{3}\pi R^3 $$
dove $\frac{4}{3}\pi R^3$ è il volume della sfera. Possiamo quindi applicare Gauss:
$$ \Phi(\vec{E}) = \frac{Q}{\epsilon_0} = \frac{\rho4\pi R^3}{3\epsilon_0} $$
Troviamo un'altra espressione per il flusso applicando la definizione, scegliendo una certa superficie gaussiana $S$ (anch'essa sferica)
di raggio $r$, che contenga la circonferenza. Avremo che il flusso su tale superficie vale:
$$ \Phi{\vec{E}} = \int_S d\vec{A} \cdot d\vec{E} = E(r)\int_S d\vec{A} $$
Avremo che l'integrale di superficie di $\vec{A}$ su $S$ non sarà altro che la superficie della sfera, ergo:
$$ \int_S d\vec{A} = 4\pi r^2, \quad \Phi{\vec{E}} = E(r)4\pi r^2 $$
Eguagliando, come prima, questo con il risultato ottenuto da Gauss abbiamo:
$$ E(r) 4\pi r^2 = \frac{\rho4\pi R^3}{3\epsilon_0} = \frac{Q}{\epsilon_0}, \quad E(r) = \frac{\rho R^3}{3r^2 \epsilon_0} = k \frac{Q}{r^2} $$
Questo ci assicura che il campo elettrico prodotto da una distribuzione di carica a simmetria sferica è
proporzionale a $\frac{1}{r^2}$. Potremmo essere interessati al campo generato dalla carica al suo interno: in questo caso
dovremo scegliere una superficie gaussiana di raggio $r < R$. Si avrà in questo caso che la carica su cui applichiamo il teorema di Gauss
(che chiameremo $Q_{int}$)è minore: comprende soltanto del volume di sfera in $r$:
$$ Q_{int} = \rho \frac{4}{3} \pi r^3 $$
Possiamo allora applicare la stessa uguaglianza precedente:
$$ E(r)4\pi r^2 = \frac{Q_{int}}{\epsilon_0}, \quad E(r) = k \frac{Q_{int}}{r^2} $$
Sostituire la formula di $Q_{int}$ sarebbe possibile, ma non conveniente: calcoliamo piuttosto il rapporto $\frac{Q}{Q_{int}}$.
$$ \frac{Q}{Q_{int}} = \frac{\rho \frac{4}{3} \pi R^3}{\rho \frac{4}{3} \pi r^3} = \frac{R^3}{r^3} $$
e moltiplichiamo numeratore e denominatore del campo:
$$ E(r) = k \frac{Q_{int} \cdot \frac{Q}{Q_{int}}}{r^2 \cdot \frac{R^3}{r^2}} = k \frac{Q}{R^3}r $$
Notiamo che per $r = R$, ovvero avvicinandosi alla superficie della sfera, abbiamo:
$$ k \frac{Q}{r^2} = k \frac{Q}{R^2} = k \frac{Q}{R^3}R $$
ovvero quanto calcolato per $r$ interni ed esterni alla sfera va a coincidere. Riassumendo, abbiamo quindi
che il campo generato dalla sfera vale esattamente 0 nel suo punto di centro, cresce linearmente fino ad un massimo
di $k\frac{kQ}{R^2}$ sulla sua superficie, e poi decresce quadraticamente al suo esterno.
\item \textbf{Campo generato da un guscio sferico uniformemente carico}
Calcoliamo il campo elettrico generato da un guscio sferico completamente cavo, di spessore trascurabile. Avremo bisogno
della densità di carica superficiale del disco $\sigma$, e potremo calcolare la carica:
$$ Q = 4\pi R^2 \sigma $$
Racchiudiamo adesso il guscio con una superficie gaussiana non dissimile da quella adoperata per la sfera carica. Avremo l'uguaglianza:
$$ E(r)4\pi r^2 = \frac{Q}{\epsilon_0}, \quad E(r) = k\frac{Q}{r^2} $$
Risultato esattamente identico a quello della sfera. Vediamo adesso cosa succede se prendiamo $r < R$. In questo caso dovremo considerare
una superficie gaussiana completamente interna al guscio. Questa superficie non conterrà alcuna carica! In altre parole, il campo elettrico
generato da un guscio sferico uniformemente carico al suo interno è nullo. Questo risultato viene spiegato più generalmente nel cosiddetto
\textbf{teorema del guscio sferico}, riguardante perlopiù la forza gravitazionale, che stabilisce:
\begin{itemize}
  \item Un guscio sferico di massa uniforme genera lo stesso campo di una carica puntiforme posta esattamente al suo centro;
  \item Il campo generato dal guscio sferico in un punto al suo interno è nullo.
\end{itemize}
\item \textbf{Campo generato da un guscio spesso uniformemente carico}
Facciamo un esempio analogo al precedente, ma considerando il nostro guscio come dotato di un certo spessore. Questo da solo renderà
necessario l'utilizzo della densità (volumetrica) di carica $\rho$, da cui potremo impostare, chiamando il raggio interno $a$ e quello esterno $b$:
$$ Q = \rho \frac{4}{3}\pi(b^3 - a^3) $$
Su un raggio $r$ esterno al guscio, il risultato sarà coincidente con i due precedenti: prendiamo sempre una superficie gaussiana sferica che racchiuda
l'intero oggetto, applichiamo Gauss ed otteniamo:
$$ E(r)4\pi r^2 = \frac{Q}{\epsilon_0}, \quad E(r) = k\frac{Q}{r^2} $$
Il caso interessante è quello in cui il campo viene calcolato all'interno del guscio. Avremo che la carica a questo punto varrà:
$$ Q_{int} = \rho \frac{4}{3} \pi (r^3-a^3)$$
questo ci darà:
$$ E(r)4 \pi r^2 = \frac{Q_{int}}{\epsilon_0}, \quad E(r) = \rho \frac{4}{3} \pi \frac{r^3-a^3}{4\pi R^2} $$
che, a differenza del caso sferico, non è una funzione lineare di r. Notiamo che per $a\rightarrow 0$,
questa forma tende al caso lineare: ciò ha senso in quanto un guscio sferico con raggio minore nullo non è altro che
una sfera.
\item \textbf{Campo generato da una distribuzione di carica piana, uniforme e infinita} \\
Vediamo il caso di una distribuzione di carica piana, uniformemente carica e infinita. Ricordiamo che tale distribuzione di carica può essere rappresentata
da un disco uniformemente carico di raggio infinito, il cui campo abbiamo già calcolato attraverso il principio di sovrapposizione. Ne risulterà che il limite per il raggio
che tende ad infinito della formula che avevamo trovato corrisponderà a quanto ricaveremo adesso.
Iniziamo col considerare una lastra piana e uniformemente carica: definita la densità di carica superficiale $\sigma$ potremo applicare Gauss:
$$ Q = A\sigma, \quad \Phi{\vec{E}} = \frac{A\sigma}{\epsilon_0} $$
Possiamo scegliere un'altra superficie gaussiana: per un qualsiasi punto $\vec{r}$ sulla lastra a distanza $d$, consideriamo un cilindro perpendicolare alla lastra.
Avremo che il flusso lungo tale cilindro è diviso nel flusso lungo il lato, che sarà nullo, e il flusso sulle basi:
$$ \Phi{\vec{E}} = \Phi_L{\vec{E}} + \Phi_B{\vec{E}} = 2AE(d) $$
Imponendo l'uguaglianza delle formule per il flusso si avrà:
$$ 2AE(d) = \frac{A\sigma}{\epsilon_0}, \quad E(d) = \frac{\sigma}{2\epsilon_0} $$
\item \textbf{Campo generato dalle lastre di un condensatore ad armature parallele} \\
Il risultato appena trovato si presta al calcolo del campo generato all'interno ed all'esterno di un componente circuitale assai comune: il condensatore.
Assumiamo un condensatore come formato da due lastre piane, che prendiamo come infinite (come si è visto, questa approssimazione descrive piuttosto bene ciò
che succede lontano dai bordi). Assegneremo alle due lastre le densità di carica superficiale $\sigma_1$ e $\sigma_2$. Un calcolo immediato si può fare attraverso il principio di sovrapposizione,
per due lastre generanti campo $\frac{\sigma_i}{2\epsilon_0}$, il campo totale esterno sarà:
$$ E_{int} = \frac{\sigma_1 + \sigma_2}{2\epsilon_0} $$
mentre il campo totale interno sarà (dovremo considerare segni invertiti, in quanto aprrocciamo le lastre da direzioni opposte):
$$ E_{est} = \frac{\sigma_1 - \sigma_2}{2\epsilon_0} $$
Un calcolo simile può essere fatto usufruendo del teorema di Gauss, giungendo allo stesso risultato. Sia $L$ una superficie gaussiana, identica a quella usata sulla lastra
nell'esempio precedente, posta sulla lastra $\sigma_1$; $R$ una superficie simile sulla lastra $\sigma_2$, e infine $C$ una superficie gaussiana che comprenda di entrambe le lastre.
Possiamo impostare uguaglianze simili a quelle già usate finora (fra flusso calcolato "a mano" e con Gauss) su queste superfici. Ad esempio, riguardo a $L$:
$$ A E_L = \frac{Q_{est}}{\epsilon_0} = \frac{A\sigma_1}{2\epsilon_0}, \quad E_L = \frac{\sigma_1}{2\epsilon_0} $$
Ciò ci permette di ricavare il sistema:
$$ 
\left\{\begin{array}{l}
    E_L = \frac{\sigma_1}{2\epsilon_0} \cr \\
    E_R = \frac{\sigma_2}{2\epsilon_0} \cr \\ 
    E_C = E_L - E_R
\end{array}\right.
$$
da cui:
$$ E_C = E_{est} = \frac{\sigma_1 - \sigma_2}{2\epsilon_0} $$
Per il campo esterno. Per quanto riguarda il campo interno, basterà prendere una superficie gaussiana che comprende solamente la parte interna delle
lastre $\sigma_1$ e $\sigma_2$. Impostando Gauss si avrà:
$$ \int E \cdot dA = E\cdot A = \frac{Q_{int}}{\epsilon_0} = \frac{(\sigma_1 + \sigma_2)A}{2\epsilon_0}, \quad E_{int} = \frac{\sigma_1+\sigma_2}{2\epsilon_0} $$
Che corrisponde con quanto trovato prima. Possiamo adesso considerare il caso particolare del condensatore, dove $\sigma_1 = -\sigma_2$:
$$ E_{est} = \frac{\sigma - \sigma}{2\epsilon_0} = 0, \quad E_{int} = \frac{ \sigma + \sigma}{2\epsilon_0} = \frac{\sigma}{\epsilon_0} $$
\item \textbf{Distribuzione di carica cilindrica}
  Vediamo un'esempio di distribuzione di carica cilindrica che non è un filo monodimensionale, ma un vero e proprio cilindro con un dato volume $V$. Diciamo che $R$ è il raggio del cilindro, $h$ la sua altezza, e che il campo elettrico
  è ortogonale alla sua parete: questo significherà che il flusso lungo le basi è nullo. Diciamo allora di calcolare il campo a una distanza $r>R$. Possiamo dire che la carica è, stabilita una densità di carica $\rho$:
  $$ Q = \rho V = \rho \pi R^2 h$$
  A questo punto potremo impostare il campo:
  $$ \Phi_L = \int \vec{E} \cdot dA = E(r) 2\pi r h $$
  Su una superficie cilindrica di raggio $r$. Lo stesso valore per Gauss:
  $$ \Phi_L = \frac{\rho V}{\epsilon_0} = \frac{\rho \pi R^2 h}{\epsilon_0} $$
  da cui diciamo:
  $$ E(r)2\pi r h = \frac{\rho \pi R^2 h}{\epsilon_0}, \Rightarrow E(r) = \frac{\rho R^2}{2r\epsilon_0} $$
  Calcoliamo allora il campo su un punto interno $r < R$. Dovremo considerare la carica interna, e non totale:
  $$ Q_int = \rho V_int = \rho \pi r^2 h$$
  E impostare la stessa uguaglianza di prima:
  $$ E(r)2\pi r h = \frac{\rho \pi r^2 h}{\epsilon_0}, \Rightarrow E(r) = \frac{\rho r}{2\epsilon_0} $$
  Notiamo che i campi sulla frontiera del cilindro si equivalgono. Infatti:
  $$ r = R \Rightarrow \frac{\rho R^2}{2\pi \epsilon_0} = \frac{\rho R}{2\epsilon_0} = \frac{\rho r}{2\epsilon_0}\Big|_{r=R}$$
\end{itemize}
\end{document}
