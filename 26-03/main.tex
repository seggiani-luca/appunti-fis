\documentclass[a4paper,12pt]{article}

\usepackage[french,italian]{babel}
\usepackage[T1]{fontenc}
\usepackage[utf8]{inputenc}
\frenchspacing 
\title{Appunti Fisica I}
\author{Luca Seggiani}
\date{26 Marzo 2024}

\begin{document}
\maketitle
\section{Equilibrio}
Il grafico dell'energia potenziale ci permette di ottenere informazioni riguardanti l'equilibrio del sistema studiato. In particolare, distinguiamo i casi:
\begin{itemize}
  \item Punto stazionario, massimo locale: si parla di equilibrio instabile. Una minima variazione delle condizioni
    del sistema porterebbe all'alterazione di esso, probabilmente verso un punto di equilibrio stabile
  \item Punto stazionario, minimo locale: si parla di equilibrio stabile. Una variazione delle condizioni
    del sistema sufficientemente piccola non ha grandi conseguenze sul sistema, che torna naturalmente al suo stato 
    di equilibrio
  \item Punto stazionario, funzione costante in un suo intorno: si parla di equilibrio indifferente. Il sistema
    non reagisce a minime variazioni delle sue condizioni.
\end{itemize}
\section{Energia meccanica e principio di conservazione dell'energia meccanica}
Riprendiamo il teorema delle forze vive:
$$ K_f - K_i = L_{totale} = L_{cons} + L_{non \ cons} = -(U_f - U_i) + L_{non \ cons}$$
Definiamo allora l'\textbf{energia meccanica}:
$$ E = K + U $$
per cui la formula precedente diventa:
$$ K_f + U_f - (K_i + U_i) = E_f - E_i = L_{non \ cons} $$
noto anche come \textbf{principio di conservazione dell'energia meccanica}. In particolare, se on ci sono forze
non conservative, o se ci sono ma compiono lavoro nullo:
$$ E(P) = K(P) + U(P) = \mathrm{const.} = K(P_1) + U(P_1) $$
su due punti $P$ e $P_1$ arbitrari.
\section{Applicazioni della conservazione dell'energia}
Vediamo alcuni esempi di problemi che richiedono l'applicazione del principio di conservazione dell'energia.
\par\smallskip
\textbf{Pendolo sul piano verticale senza attrito} \\
Poniamo un pendolo, ovvero una massa legata da un filo di lunghezza $r$ a un perno centrale. Vogliamo spingere
la massa, modellizzata come un punto materiale, dal punto più basso $P_1$ a quello più alto $P_2$ della circonferenza, e vogliamo
quindi calcolare la velocità minima con cui il punto deve partire da $P_1$. Notiamo che non basterà raggiungere
$P_1$, ma dovremo raggiungerlo con una velocità tale da proseguire il moto circolare senza cadere per via della gravita.
Notiamo inoltre che l'unico punto critico sarà $P_2$, in quanto il suo raggiungimento (e superamento) significa anche il
raggiungimento di tutti gli altri punti ad esso sottostanti. \\
Cerchiamo innanzitutto la velocità minima necessaria all'arrivo sulla verticale. La reazione del filo sarà:
$$ T = mg\cos{\theta} + m\frac{v^2}{r} \geq 0 $$
Una $T$ maggiore di zero ci assicura che il filo sta esercitando una forza per mantenere la massa in rotazione, e quindi
che è in tensione. Poniamo quindi:
$$ \frac{v^2}{r} \geq -g\cos{\theta}, \quad v_{P_2} = \sqrt{gr\cos{\theta}}$$
Infine, prendiamo $v_{P_2}$, con $\theta = \pi$: $v_{P_2} = \sqrt{gr}$ in valore assoluto. \\
Utilizziamo allora il principio della conservazione dell'energia: sappiamo che:
$$ \Delta K = -\Delta U $$
ovvero la variazione di energia cinetica è uguale all'opposto della variazione di energia potenziale. 
La variazione di energia cinetica equivale alla differenza fra l'energia cinetica alla velocità minima appena trovata (nel caso peggiore, più lento
possibile) e quella alla velocità $v_i$ che volevamo cercare:
$$ \Delta K = \frac{1}{2}mv_{P_2}^2 - \frac{1}{2}mv_i^2 = \frac{1}{2}mgr - \frac{1}{2}mv_i^2$$
Posto poi l'origine del sistema di riferimento alla base della circonferenza, avremo che l'energia potenziale è 0 in $P_1$,
ergo la sua variazione una volta toccato $P_2$ sarà:
$$ \Delta U = 2mgr$$
Possiamo allora impostare:
$$ \frac{1}{2}mgr - \frac{1}{2}mv_i^2 = -2mgr $$
da cui si ricava la velocità iniziale necessaria, $vi = \sqrt{5gr}$.
\par\smallskip
\textbf{Pendolo sul piano verticale senza attrito, $\theta$ arbitrario} \\
Cerchiamo di trovare la velocità necessaria al raggiungimento di angoli $\theta_0$ arbitrari. Definiamo l'energia potenziale,
invece che in funzione dell'altezza massima $2r$, in funzione dell'angolo, con:
$$ U(\theta) = mgL(1-\cos{\theta}) $$
Avremo dalla conservazione dell'energia che:
$$ U(0) + \frac{1}{2}mv^2 = U(\theta_0) + \frac{1}{2mv^2} \Rightarrow \frac{1}{2}mv^2 = mgL(1-\cos{\theta_0})$$
visto che la potenziale all'istante 0 è nulla, come è nulla l'energia cinetica nel punto che vogliamo raggiungere
(che sarà di cambio direzione). Da questo otteniamo $v_0 = \sqrt{2gL(1-\cos{\theta})}$.
\par\smallskip
\textbf{Pendolo oscillante} \\
Descriviamo adesso il fenomeno oscillatorio presentato dal pendolo. Possiamo individuare le forze agenti sulla sua massa in direzione
radiale:
$$ -T + mg\cos{\theta} = ma_r = -m \frac{v^2}{r} $$
e tangenziale:
$$ -mg\sin{\theta} = ma_\theta = m \frac{dv}{dt} $$
Il moto analizzato è effettivamente un moto circolare vario, con accelerazione centripeta in direzione radiale, e inoltre
accelerazione variabile in direzione longitudinale. Conviene definire quest'ultima in funzione della velocità tangenziale,
sfruttando l'equivalenza dell'arco $s = L\theta$:
$$ v(t) = \frac{ds}{dt} = \frac{dL\theta}{dt} = L \frac{d\theta}{dt} = L\omega(t) $$
$$ a_\theta(t) = \frac{dv(t)}{dt} = \frac{d\omega}{dt} = L\frac{d\omega}{dt} = L\frac{d^2\omega}{dt^2} $$
da cui ricaviamo nuovamente la componente tangenziale:
$$ -mg\sin{\theta} = mL \frac{d^2 \theta}{dt^2} $$
ma stavolta in funzione di una derivata di $\theta$. Questo ci permette di impostare l'equazione differenziale:
$$ \frac{d^2\theta}{dt^2} + \frac{g}{L}\sin{\theta} = 0$$
Notiamo che compare la funzione seno. Per piccoli angoli $\theta$ conviene usare l'approssimazione $x$ per il seno,
ovvero la sua formula di Taylor di primo grado centrata in 0 (quindi formula di MacLaurin...):
$$ \frac{d^2\theta}{dt} + \frac{g}{L}\theta = 0 \Rightarrow \frac{d^2\theta}{dt} + \Omega^2\theta = 0, \quad \Omega = \sqrt{\frac{g}{L}} $$
Che è l'equazione differenziale del moto armonico, e ha come soluzione:
$$ \omega{t} = A\cos{\Omega t + \phi} $$
con pulsazione $\Omega$ e periodo $T = 2\pi\sqrt{\frac{L}{g}}$. Come sempre, le costanti si determinano dalle condizioni iniziali di velocità
e posizione.
\end{document}
