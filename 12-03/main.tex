\documentclass[a4paper,12pt]{article}

\usepackage[french,italian]{babel}
\usepackage[T1]{fontenc}
\usepackage[utf8]{inputenc}
\frenchspacing 
\title{Appunti Fisica I}
\author{Luca Seggiani}
\date{12 Marzo 2024}

\begin{document}
\maketitle
\section{Relatività galileiana}
Consideriamo due sistemi di riferimento. Il primo, $S$, detto riferimento stazionario o di laboratorio, è 
"relativamente" in quiete. Il secondo, $S'$, è in moto rettilineo uniforme con una certa velocità $v_0$, detta
velocità di trascinamento. Al tempo $t=0$ le origini di $S$ e $S'$ coincidono. Vale allora, che per un qualsiasi
vettore $\vec{r'}$ individuato da $S'$, lo stesso vettore nel riferimento $S$ sarà:
$$ \vec{r} = \vec{r'} + \vec{v_0}t$$
Andiamo a derivare questa espressione (trasformazione di Galileo):
$$ \vec{v} = \vec{v'} + \vec{v_0}$$
Derivando nuovamente avremo:
$$ \vec{a} = \vec{a'}$$
ovvero l'accelerazione non cambia al cambiare del sistema di riferimento (questo finchè i sistemi di riferimento
considerati sono inerziali, ovvero in quiete o in moto rettilineo uniforme). Si dice che l'accelerazione è un invariante
della relatività galileiana. \\
Diciamo inoltre che nelle formule appena riportate, $\vec{v}$ prende il nome di velocità assoluta, $\vec{v'}$ di velocità
relativa, e come già detto, $\vec{v_0}$ di velocità di trascinamento.
\par\smallskip
\textbf{Riferimenti non inerziali} \\
Esaminiamo il caso in cui il nostro sistema di riferimento $SM$ (sistema mobile) si muove di velocita $\vec{v_t}$ e accelerazione
costante $\vec{a_t}$ rispetto al sistema di laboratorio $SL$. La relazione fra le posizioni diventa:
$$ \vec{r} = \vec{r'} + \vec{r}_{OO'}(t) $$
che deriveremo per ottenere la velocità:
$$ \vec{v} = \vec{v'} + \vec{v_t}, \quad \vec{v_t} = \frac{d\vec{r_{OO'}}}{dt} $$
e l'accelerazione (notiamo che non è più invariante):
$$ \vec{a} = \vec{a'} + \vec{a_t}, \quad \vec{a_t} = \frac{d\vec{r_{OO'}}^2}{d^2t} $$
con $\vec{r}_{OO'}$ uguale al vettore spostamento dall'origine in $SL$ all'origine in $SM$. \\
Ricordiamo poi che i sistemi di riferimento non sono sistemi di coordinate: più propriamente sono sistemi di coordinate
dotati di una certa misura dello spazio (un metro) e del tempo (un orologio).

\section{Introduzione alla dinamica}
Introduciamo ora le leggi fondamentali della dinamica. Innanzitutto possiamo dire che queste leggi si applicano
nei casi in cui:
\begin{itemize}
  \item Le velocità prese in considerazione sono molto più piccole di quelle della luca ($\vec{V} < c$);
  \item Le dimensioni dei corpi non sono in ordini di grandezza subatomici;
  \item I campi gravitazionali sono deboli.
\end{itemize}

nel primo e nel secondo caso, avremo bisogno rispettivamente della relatività ristretta e generale. Nel terzo caso
invece ci servirà la meccanica quantistica.
\par\smallskip
\textbf{Definizione di Forca} \\
La forza è un'interazione fra due corpi, che può essere:
\begin{itemize}
  \item \textbf{A distanza}, ovvero attraverso le cosiddette forze fondamentali o di campo (gravitazionali, elettriche,
    magnetiche, ecc...);
  \item \textbf{Per contatto}, ovvero le forze che si studiano nella meccanica (attrito, forza elastica, ecc...). Derivano
    da manifestazioni macroscopiche delle interazioni elettromagnetiche.
\end{itemize}

Le forze sono grandezze fisiche vettoriali (hanno punto d'applicazione, direzione modulo e verso). Si misurano
staticamente con il dinamometro.
\newpage
\textbf{Prima legge di Newton}
Il primo principio della dinamica, detto anche prima legge di Newton o legge d'inerzia, riguarda l'assenza di interazioni
(forze):
\begin{center}
  Esiste almeno un sistema di riferimento in cui un corpo non soggetto a forze (oppure soggetto ad un sistema di
  forze a risultante nulla $\sum\vec{F} = \vec{0}$) prosegue nel suo stato di quiete o di moto rettilineo uniforme.
\end{center}
in simboli:
$$ \sum\vec{F} = \vec{0} \Rightarrow \vec{a} = \vec{0} $$
\par\smallskip
\textbf{Sistemi di riferimento inerziali} \\
Un sistema di riferimento inerziale è un sistema in cui vale il principio d'inerzia (e quindi $\vec{F} = m\vec{a}$). 
I sistemi di riferimento inerziali formano una classe di sistemi (godono di riflessività, simmetria e transitività, ergo tutti i sistemi in movimento
fra di loro con velocità costante sono inerziali).
\par\smallskip
\textbf{Seconda legge di Newton} \\
Il secondo principio della dinamica, detto anche seconda legge di Newton o legge fondamentale della dinamica, mette
in relazione massa, forza e accelerazione di un corpo:
$$ \sum \vec{F} = m \vec{a} $$
dove $m$ è la massa, una grandezza scalare, proprietà intrinseca della composizione di un corpo.
\par\smallskip
\textbf{Terza legge di Newton} \\
Il terzo principio della dinamica, detto anche terza legge di Newton o principio di azione-reazione, stabilisce
che ogni forza esercitata su di un corpo è seguita da una reazione, ovvero una forza con lo stesso modulo e direzione
in verso opposto. Un esempio classico della terza legge di Newton è la forza vincolante espressa da ogni corpo
in collisione con un'altro corpo, che ad esempio permette al terreno di impedirci di sprofondare verso il centro della terra
sotto l'effetto della gravità.
\par\smallskip
\textbf{Le quattro interazioni fondamentali} \\
Adesso può essere conveniente esaminare, almeno a livello superficiale, le quattro interazioni (forze) fondamentali
individuate ad oggi dalla comunità scientifica:
\begin{itemize}
  \item \textbf{Forza gravitazionale} \\
    La gravità è la forza che attrae fra di loro corpi come la Terra e il sole, o ancora i corpi sulla superficie terrestre alla Terra stessa.
    Di gran lunga è la forza più debole fra le quattro fondamentali. La comprensione più completa che ne abbiamo
    è quella come espressione della geometria spazio-tempo secondo la relatività generale.\\
    La forza di gravità si applica fra due corpi nella direzione della retta che li congiunge, è direttamente
    proporzionale alle loro masse e inversamente proporzionale alla distanza fra i due:
    $$ \vec{F_g} = \frac{m_1m_2}{R^2}G\hat{R}, \quad G = 6.67\times 10^{-11} \mathrm{\frac{Nm^2}{kg^2}} $$
    dove $\hat{R}$ è il versore nella direzione della distanza fra i due corpi. Chiaramente i corpi sono attratti fra di loro
    quanto più sono vicini e le loro masse sono grandi. Inoltre, la forza gravitazionale a raggio di azione infinito (seppur insignificante
    su distanze abbastanza grandi).
  \item \textbf{Forza elettromagnetica} \\
    La forza elettromagnetica dà origine ai legami chimici, alle proprietà di atomi e molecole, alla luce nonchè a gran parte
    delle forze di contatto studiate nella meccanica classica. La forza elettromagnetica si applica fra cariche: 
    la carica elettrica, misurata in Coulomb (C), ne determina infatti il verso. Carice concordi si respingono, cariche
    discordi si attraggono. In simboli:
    $$ \vec{F_c} = \frac{q_1q_2}{R^2}k\hat{R} $$
    dove k è la costante di Coulomb, ovvero:
    $$ k = \frac{1}{4\pi\epsilon_0} = 9 \times 10^9 \mathrm{\frac{Nm^2}{C^2}} $$
    Come per la forza gravitazionale, la forza elettromagnetica ha raggio di azione infinito. La differenza però sta nel verso, che può essere
    sia attrattivo (come la gravità) che repulsivo.
  \item \textbf{Forza nucleare (forte)} \\
    La forza nucleare entra in gioco su distanze estremamente piccole, inferiori ai $10^{-15} \mathrm{m}$. La si può osservare
    nei nuclei di atomi, formati da protoni carichi positivamente, che però non si disperdono a causa della forza elettromagnetica proprio
    grazie all'azione della forza nucleare che li mantiene uniti. La forza nucleare determina le interazioni fra particelle elementari come
    quark e gluoni.
  \item \textbf{Forza nucleare debole} \\
    La forza nucleare debole è responsabile delle forze che entrano in gioco nei meccanismi di decadimento nucleare. Ha un range estremamente limitato,
    di $10^{-18} \mathrm{m}$.
\end{itemize}
\textbf{Forze di contatto} \\
Vediamo quali sono i tipi principali di forze di contatto fra corpi, o fra corpi e fluidi:
\begin{itemize}
  \item Fra corpi rigidi
    \begin{itemize}
      \item Forze vincolari
        \begin{itemize}
          \item normali fra superfici
          \item ganci o fili inestensibili
          \item cerniere
          \item ...
        \end{itemize}
      \item Attriti (paralleli alla superficie di contatto):
        \begin{itemize}
      \item statico
      \item dinamico
    \end{itemize}
    \end{itemize}
  \item Fra corpi deformabili:
    \begin{itemize}
      \item Forze elastiche
      \item Forze anelastiche
    \end{itemize}
  \item Fra un corpo solido e un fluido
    \begin{itemize}
      \item Attrito viscoso
      \item Forze di pressione
    \end{itemize}
\end{itemize}
\par\smallskip
\textbf{Forza gravitazionale} \\
Riprendendo la legge di gravitazione universale:
$$ \vec{F_m} = \frac{mM_t}{r^2}G $$
con $M_t$ massa della Terra $(6\times 10^{24})$ e $m$ massa dell'oggetto in questione, possiamo dire:
$$ \vec{a} = \frac{M_t}{R^2}G $$
in accordo con:
$$ \vec{F_m} = m\vec{a} $$
Chiaramente, nei pressi della Terra $\vec{a}$ sarà uguale a $g$, ovvero l'accelerazione gravitazionale terrestre.
La forza $F_m$ prende il nome di forza peso (da non confondersi con la massa!). La forza peso è diretta approssimativamente
verso il centro della terra e vale circa $9.81\cdot m$ per un corpo di massa $m$, sulla superficie terrestre attorno
al $45^\circ$ parallelo.
\par\smallskip
\textbf{Forze normali} \\
Vediamo adesso tutte quelle forze vincolari che i corpi esercitano l'uno sull'altro perpendicolarmente alle loro superfici.
Le forze normali si verificano solamente in caso di compressione. Notiamo che la forza normale esercitata da un corpo
che sorregge un'altro corpo non è sempre uguale al peso: ad esempio, il corpo sorretto potrebbe da parte sua subire altre
forze, con una componente parallela alla forza peso ma diretta verso l'alto o verso il basso, che andrebbero rispettivamente
a diminuire o decrementare la reazione vincolare.
\par\smallskip
\textbf{Piano inclinato} \\
Notiamo adesso come la forza normale non è necessariamente diretta verso l'asse verticale. Prendiamo in esempio un piano inclinato
di angolo $\theta$. Potremo allora dividere la forza di gravità che agisce sul corpo in due componenti: quella 
parallela e quella perpendicolare alla superfice del piano. Chiamandole rispettivamente $F_\parallel$ e $F_\perp$ potremo dire:
$$ F_\parallel = F_g \sin{\theta}, \quad F_\perp = -F_g \cos{\theta} $$
In questo caso, il piano inclinato compenserà con una reazione vincolare soltanto la componente $F_\perp$, mentre la componente
$F_\parallel$ restera invariata nella direzione parallela al piano.
\par\smallskip
\textbf{Contatto diretto} \\
Due oggetti di massa $m_1$ e $m_2$, in contatto diretto e su cui viene applicata una forza (e che quindi si spingono fra di loro),
possono essere considerati dal punto di vista della forza come un unico oggetto di massa $m_1 + m_2$, da cui l'accelerazione complessiva:
$$ a = \frac{F}{m_1 + m_2} $$
Possiamo a questo punto trovare le due forze di reazione $R_1$ e $R_2$ che i corpi applicano l'uno sull'altro:
$$ R_1 = \frac{Fm_1}{m_1 + m_2}, \quad R_2 = \frac{Fm_2}{m_1 + m_2} $$
Notiamo inoltre che $R_1 = F - R_2$ e $R_2 = F - R_1$, ovvero:
$$ F = R_1 + R_2 $$
\textbf{Funi e corde} \\
Consideriamo funi e corde come vincoli monodimensionali privi di massa inerziale e inestensibili. Trasferiscono la forza,
detta tensione ($T$) fra i loro estremi. 
\end{document}
