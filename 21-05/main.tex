\documentclass[a4paper,12pt]{article}

\usepackage[french,italian]{babel}
\usepackage[T1]{fontenc}
\usepackage[utf8]{inputenc}
\frenchspacing 
\title{Fisica I}
\author{Luca Seggiani}
\date{21 Maggio 2024}

\begin{document}
\maketitle
\par\smallskip
\textbf{Energia cinetica nel moto di puro rotolamento} \\
L'energia cinetica di un corpo in puro rotolamento può essere trovata con un polo all'origine, o (in modo molto più conveniente) con un polo al punto di contatto:
questo perché il punto di contatto è effettivamente fermo, e la forza di attrito statico non compie lavoro. Abbiamo quindi 
$$ K = \frac{1}{2}I_p\omega^2 = \frac{1}{2} I_{CM}\omega^2 + \frac{1}{2}Mv_{CM}^2 $$
Possiamo notare l'uguaglianza delle due forme attraverso Steiner-Huygens:
$$ I_p = I_{CM} + MR^2$$
Da cui:
$$ \frac{1}{2}I_p\omega^2 = \frac{1}{2}(I_{CM} + MR^2)\omega^2 = \frac{1}{2}I_{CM}\omega^2 + \frac{1}{2}MR^2\omega^2 = \frac{1}{2}I_{CM}\omega^2 + \frac{1}{2}Mv_{CM}^2 $$
che è uguale a prima.

\par\smallskip
\textbf{Teorema dell'impulso o del momento dell'impulso} \\
Sappiamo che all'applicazione di una forza impulsiva $\vec{F}$ in un tempo molto ristretto $t - t_0 = \tau$ comporta
una variazione di quantità di moto:
$$ \vec{I} = \int_{t_0}^t \vec{F}dt' = \Delta \vec{P} = \vec{P}(t) - \vec{P}(t_0) $$
Analogamente, possiamo definire rispetto al momento delle forze l'impulso angolare:
$$ \vec{J} = \int_{t_0}^t \vec{\tau}dt' = \Delta \vec{L} = \vec{L}(t) - \vec{L}(t_0) $$
Impulso e impulso angolare sono legati dal \textbf{teorema del momento dell'impulso}:
$$ \vec{J} = \vec{r} \times \vec{I} $$
Questo significa che, con un applicazione a $r$ raggio diverso da zero, una forza impulsiva comporta anche una
variazione del momento angolare.
\end{document}
