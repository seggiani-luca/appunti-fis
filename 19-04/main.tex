\documentclass[a4paper,12pt]{article}

\usepackage[french,italian]{babel}
\usepackage[T1]{fontenc}
\usepackage[utf8]{inputenc}
\frenchspacing 
\title{Appunti Fisica 1}
\author{Luca Seggiani}
\date{19 Aprile 2024}

\usepackage{amsmath}

\begin{document}
\maketitle
\par\smallskip
\textbf{Rotazione in distribuzione di massa asimmetrica} \\
Vediamo nel dettaglio la situazione dove la distribuzione di massa è asimmetrica rispetto all'asse di rotazione (può
comunque essere simmetrica, solo su un asse sghembo rispetto a quello di rotazione). Immaginiamo un manubrio formato da un'asta
di massa trascurabile e lunghezza $l$ e due punti materiali di massa $M$ alle sue estremità. Il manubrio è piegato di un angolo
$\theta$ rispetto alla verticale, e ruota invece proprio attorno alla verticale. Avremo che il suo momento angolare sarà la somma
dei momenti delle due masse:
$$ \vec{L} = \vec{L_1} + \vec{L_2} $$
L'angolo formato tra la velocità angolare e il momento angolare sarà allora $(\frac{\pi}{2} - \theta)$, da cui:
$$ \vec{L} = \sum m_i\vec{R_i} \times \vec{v_i}, \quad v_i = \omega \frac{l}{2} \sin{\theta} \Rightarrow L_1 = L_2 = M \frac{l}{2} \omega \frac{l}{2} \sin{\theta} $$
Dividiamo allora questo momento sui due assi:
$$ L_{1_z} = L_{2_z} = L_1\cos{\frac{\pi}{2} - \theta} = L_1\sin{\theta} = 2L_1\sin{\theta} = \frac{M}{2} l^2 \omega \sin{\theta}^2 = I_z\omega$$
Che corrisponde al momento di inerzia rispetto all'asse di rotazione $\hat{\omega}$. Notiamo che qui il momento angolare vale $I_z = \frac{M}{2} l^2 \sin^2{\theta}$,
e $l\sin{\theta}$ non è altro che la distanza $d_i$ del punto materiale dall'asse di rotazione. Ergo vale la solita formula per il momento d'inerzia di una massa sospesa:
$$ I_z = md_i^2, \quad d_i = l\sin{\theta} \Rightarrow I_z = \frac{M}{2}l^2\sin{\theta}^2 $$
\\ Il momento angolare è inoltre costante lungo l'asse z: il momento delle forze sarà tutto sull'asse x, visto che l'unica forza in gioco è la forza peso.
Si avrà allora che:
$$ \tau_z = 0, \quad \frac{dL}{dt} = \vec{\tau} \Rightarrow \frac{dL_z}{dt} = 0, \quad L_z = \mathrm{const.}$$
La componente $L_\perp$ sarà invece in continua variazione, come avevamo già visto.
\par\smallskip
\textbf{Seconda equazione cardinale per momento angolare parallelo all'asse} \\
Abbiamo quindi che per distribuzioni di massa simmetrica ($\vec{L} \parallel \vec{\omega})$:
$$ \vec{L} = I_z\omega $$
Possiamo derivare $\vec{L}$:
$$ \vec{\tau} = \frac{d\vec{L}}{dt} = \frac{dI_z\omega}{dt} = I_z\vec{\alpha} $$
Questo significa che il momento delle forze esterne agenti rispetto all'asse di rotazione permette di calcolare l'accelerazione
angolare $\vec{\alpha}$.
\section{Lavoro nel moto rotazionale}
Prendiamo il caso di una carrucola spinta da una forza $T$ posta ad una distanza $R$ dal centro. Avremo che il lavoro su uno spostamento
infinitesimo $d\mathcal{L}$ è:
$$ d\mathcal{L} = \vec{T} \cdot \vec{dr} = T dr \cos{0}, \quad dr = Rd\theta $$
$$ d\mathcal{L} = TRd\theta = \tau_z d\theta $$
Quindi il momento torcento compie un lavoro pari al momento della forza per la variazione dell'angolo. In generale
il lavoro per una rotazione finita dall'angolo $\theta_i$ all'angolo $\theta_f$ sarà:
$$ \mathcal{L}_{\theta_i\rightarrow\theta_f} = \int_{\theta_i}^{\theta_f} \tau_z d\theta = \int_{\theta_i}^{\theta_f} I_z \dot{\omega} d\theta = \int_{\theta_i}^{\theta_f} I_z\dot{\omega}\omega dt = \int_{\theta_i}^{\theta_f} I_z \frac{d\omega}{dt}\omega dt = \int_{\theta_i}^{\theta_f} I_z \omega d\omega$$
Che da il risultato:
$$ \mathcal{L} = \frac{1}{2}I_z\omega_f^2 - \frac{1}{2}I_z\omega_i^2 $$
\par\smallskip
\textbf{Conservazione dell'energia rotazionale} \\
Riprendiamo la nostra carrucola. Sia una massa $m$ attaccata alla carrucola attraverso una fune. Calcoliamo allora la variazione di energia
meccanica quando la massa scende di un tratto $h$, notando che $h = R\theta$:
$$ \Delta E = L_{NC} = L_{\vec{T}} - L_{\vec{T}} + L_{R_v} = 0 $$
Ergo se non ci sono forze non conservative l'energia meccanica si conserva:
$$ E_i = E_f,\quad E_i = mgh_i, \quad E_f = \frac{1}{2}mv_m^2 + \frac{1}{2}I\omega^2 + mgh_f $$
$$ \Rightarrow mg(h_i-h_f) = \frac{1}{2}mv_m^2 + \frac{1}{2}I\omega^2 $$
\par\smallskip
\textbf{Esempio di applicazione della dinamica rotazionale} \\
Vediamo un problema che sapevamo già risolvere attraverso la dinamica, e applichiamo quello che adesso conosciamo sulla
dinamica rotazionale per trovare risultati più precisi. \\
Il problema presuppone una carrucola di raggio $R$ con due masse, $m_1$ e $m_2$. Le masse collegate dalla carrucola
agiranno su di essa con due tensioni, $T_1$ e $T_2$, dipendenti dalla forza peso. Possiamo immaginare queste due tensioni identiche
(ovvero sia il filo che la carrucola trascurabili in massa) e impostare le equazioni dinamiche:
$$ 
\left\{\begin{array}{l}
    m_1\vec{a_1} = \vec{T_1} - m_1g \cr \\
    m_2\vec{a_2} = \vec{T_2} - m_2g
\end{array}\right.
, \quad \vec{T_1} = \vec{T_2}, \quad \vec{a_1} = -\vec{a_2} \Rightarrow
m_1\vec{a}+m_1g = m_2g-m_2\vec{a}
$$
Questo si ottiene dal fatto che l'accelerazione in alto di una massa comporta l'accelerazione in basso dell'altra, e viceversa, e dà
il risultato:
$$ \vec{a} = g \frac{m_2-m_1}{m_1+m_2} $$
Questo risultato è corretto nel caso la carrucola, ricordiamo, abbia massa nulla: le equazioni della dinamica rotazionale ci forniscono
un sistema più completo che può invece tenere conto della sua massa. Partiamo dall'equazione trovata prima:
$$ I_z \vec{\alpha} = \vec{\tau_{tot}} $$
Dove $\vec{\alpha}$ è l'accelerazione angolare della carrucola, $I_z$ il suo momento d'inerzia, e $\vec{\tau_{tot}}$ il momento risultante delle
forze che agiscono su di essa. Possiamo calcolare separatamente questi termini. Iniziamo dal momento d'inzerzia, assumendo la carrucola
come un disco di massa uniforme $M$:
$$ I_z = \frac{MR^2}{2}$$
Vediamo poi il momento risultante delle forze $\tau_{tot}$. Sarà necessario, per tener conto della massa della carucola, adoperare due diverse tensioni,
una per ogni capo della corda, $T_1$ e $T_2$, da non assumere come uguali fra loro:
$$ \tau_{tot} = \vec{R} \times (T_2-T_1)$$
Infine $\vec{\alpha}$ accelerazione angolare è proporzionale di $R$ all'accelerazione $a$, che definiamo come prima come l'accelerazione in direzione di una sola massa
(la prima):
$$ \vec{\alpha} = \frac{a}{R} $$
Possiamo sostituire quanto trovato nella formula iniziale per ottenere che:
$$ I_z \vec{\alpha} = \vec{\tau_{tot}} = \frac{MR^2}{2} \frac{a}{R} = (T_2-T_1)R \Rightarrow \frac{Ma}{2} = (T_2 - T_1)$$
Ricaviamo quindi la $(T_2-T_1)$ dalle equazioni dinamiche ricavate precedentemente:
$$ T_2-T_1 = m_2g-m_2a-m_1a-m_1g$$
Da cui l'equazione finale:
$$ \frac{Ma}{2} =  m_2g-m_2a-m_1a-m_1g \Rightarrow a = g\frac{m_2-m_1}{m_1+m_2+\frac{M}{2}}$$
Notiamo che l'unica differenza dall'approccio precedente è il termine $\frac{M}{2}$ al denominatore. Se assumiamo $M = 0$, le formule
corrispondono (che è come dovrebbe essere!).
\par\smallskip
\textbf{Applicazione della conservazione dell'energia rotazionale} \\
Lo stesso risultato si può ottenere attraverso la conservazione dell'energia rotazionale. Nel sistema non agiscono
forze non conservative, o almeno le tensioni della fune non svolgono lavoro e non lo fa nemmeno la reazione vincolare della carrucola.
Avremo quindi che in qualsiasi momento, l'energia meccanica si conserva:
$$ \frac{d}{dt}E = 0 $$
Possiamo quindi impostare l'energia meccanica come:
$$ E = m_1gh_1 + m_2gh_2 + \frac{1}{2}m_1v_1^2 + \frac{1}{2}m_2v_2^2 + \frac{1}{2}I_z\omega^2 $$
Dove abbiamo le energie potenziali e cinetiche delle masse e l'energia rotazionale della carrucola. Problematici
potrebbero essere i fattori $h_1$ e $h_2$. Imponiamo allora l'inestensibilità della fune: diciamo che la sua lunghezza complessiva $l$ è sempre costante,
ergo le altezze $h_1$ e $h_2$ delle masse dovranno essere una l'opposta dell'altra:
$$ h_1 = -h_2 $$ 
Otteniamo così un'equazione da derivare, applicando la relazione già usata prima riguardo alle accelerazioni, $a_1 = -a_2$.
$$ E = m_1gh-mg_1h + \frac{1}{2}m_1v_1^2 + \frac{1}{2}m_2v_2^2+\frac{1}{2}I_z + \frac{1}{2}I_z\omega^2$$
$$ \frac{d}{dt}E = m_1gv-m_2gv+m_1av+m_2av+\frac{Mva}{2} = 0$$
Per quanto riguarda il fattore $\frac{Mva}{2}$, possiamo vedere come si svolgono i calcoli sostituendo $I_z$ prima o dopo la derivazione:
$$ \frac{1}{2}I_z\omega^2 = \frac{1}{2}\frac{MR^2}{2}\omega^2 = \frac{MR^2\omega^2}{4} =  \frac{Mv^2}{2}, \quad \mathrm{da} \ \omega R = v, \quad \frac{d}{dt}\frac{Mv^2}{2} = \frac{Mva}{4}$$
oppure:
$$ \frac{d}{dt}\frac{1}{2}I_z\omega^2 = I_z\omega\alpha = \frac{MR^2}{2}\omega\alpha = \frac{MR^2}{2} \frac{va}{R^2} = \frac{Mva}{4}$$
Come vediamo, dividendo per $v$ (che sappiamo essere $\neq 0$ in caso di non equilibrio) si ha:
$$ \frac{Ma}{2} =  m_2g-m_2a-m_1a-m_1g \Rightarrow a = g\frac{m_2-m_1}{m_1+m_2+\frac{M}{2}}$$
Che è esattamente identico a quanto avevamo ottenuto con gli altri metodi. Tomato tomato.
\end{document}
