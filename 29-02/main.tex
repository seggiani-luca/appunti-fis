\documentclass[a4paper,12pt]{article}

\usepackage[french,italian]{babel}
\usepackage[T1]{fontenc}
\usepackage[utf8]{inputenc}
\frenchspacing 
\title{Appunti Fisica I}
\author{Luca Seggiani}
\date{29 Febbraio 2024}

\begin{document}
\maketitle
\section{Concetto di campo}
In fisica definiamo un campo come una funzione multivariabile dallo spazio o dal piano ad un campo scalare
o ad uno spazio vettoriale, rispettivamente (nello spazio):
\begin{itemize}
  \item Campo scalare:
    $$ T = T(x, y, z) $$
  \item Campo vettoriale:
    $$ V = V(x, y, z) = V_x(x, y, z)\hat{i} + V_y(x, y, z)\hat{j} + V_z(x, y, z)\hat{k} $$
\end{itemize}

\section{Introduzione alla meccanica classica}
La meccanica classica è la branca della fisica che si occupa dello studio di corpi statici e in movimento,
con dimensioni superiori a quelle delle particelle subatomiche e velocità non comparabili a quella della luce,
(compito rispettivamente della meccanica quantistica e relativistica). Possiamo ulteriormente dividere la meccanica
classica in:
\begin{itemize}
  \item Cinematica:
    descrizione della traiettoria dei corpi in funzione del tempo, senza tener conto delle cause dei loro moti
  \item Dinamica:
    studio delle cause dei moti dei corpi (quindi delle forze)
  \item Statica:
    studio dell'equilibrio dei corpi in quiete.
\end{itemize}

\textbf{Il punto materiale} \\
Il punto materiale è un'approssimazione che adottiamo nel caso in cui l'estensione dei corpi di cui parliamo
sia irrilevante ala situazione che vogliamo analizzare. Un punto materiale non ha quindi estensione, ma solamente
una posizione ed una certa massa (da cui la dicitura "punto di massa,  \textit{mass point}").
\par \medskip
\textbf{Moto rettilineo}
Definisco la posizione $x$ di un corpo su una retta in funzione del tempo come:
$$ x(t) $$
che prende il nome di \textit{legge oraria} del moto .
Una variazione di spazio (quindi uno spostamento) da $x(t_2)$ a $x(t_1)$ punti distinti nel tempo sarà quindi:
$$ \Delta x = x(t_1)-x(t_2) $$

A questo punto la variazione di tempo fra $t_1$ e $t_2$, noto $t_2 > t_1$, sarà:
$$ \Delta t = t_2 - t_1 $$
e potrò definire la velocità media:
$$ v_x = \frac{\Delta x}{\Delta t} = \frac{x(t_2)- x(t_1)}{t_2- t_1} $$

si nota che questo corrisponde di fatto a $\tan{\alpha}$ con $\alpha$ uguale all'angolo formato dalla tangente
del grafico della legge oraria in un dato punto $x_0 \in [x(t_1), x(t_2)]$.

Si definisce poi la traiettoria di un punto materiale come il \textit{"Luogo geometrico dei punti occupati nel tempo da
un punto materiale"}, ergo tutti i punti che la mia legge oraria $x(t)$ restituisce in un dato intervallo di $t$.
\par\smallskip
Prendiamo adesso in esempio la legge oraria di un moto rettilineo uniformemente accelerato:
$$ x(t) = x_0+v_0t + \frac{1}{2}a_0t^2 $$
considerando le unità di misura di $x_0$, $v_0$ e $a_0$, ovvero m, $\frac{\mathrm{m}}{\mathrm{s}}$ e $\frac{\mathrm{m}}{\mathrm{s}^2}$,
posso studiare la coerenza dimensionale:
$$ [L] + \frac{[L]}{[T]} [T] + \frac{[L]}{[T]^2}[T]^2 = [L]+ [L]+ [L] $$
che è dimensionalmente coerente.
\par\smallskip
Adottiamo adesso alcuni degli strumenti del calcolo per ottenere da una curva continua della legge oraria
(magari ottenuta dal curve-fitting di dati sperimentali), il grafico della velocità del punto materiale.
Si renderà necessario un passaggio da velocità media a velocità \textit{istantanea}, definita come la derivata
della legge oraria $x(t)$ in un dato punto $x_0$. Formalmente, presa la precedente definizione di velocità media
possiamo dire:
$$ v_x = \frac{x_2-x_1}{t_2-t_1} \rightarrow \lim_{\Delta t \rightarrow 0} \frac{x(t + \Delta t) - x(t)}{t} $$
ovvero il limite del rapporto incrementale in un punto $x_0 = x(t)$ su variazioni infinitesimali di $t$.
Questo si può inoltre esprimere come:
$$ v(t_0) = \frac{dx}{dt} \bigg |_{t=t_0} $$
e si nota inoltre che in fisica è comune la notazione $\dot{x}$ per le derivate.

\end{document}
