\documentclass[a4paper,12pt]{article}

\usepackage[french,italian]{babel}
\usepackage[T1]{fontenc}
\usepackage[utf8]{inputenc}
\frenchspacing 
\title{Appunti Fisica I}
\author{Luca Seggiani}
\date{11 Aprile 2024}

\begin{document}
\maketitle
\section{Urti}
Ricordiamo la definizione di impulso per una forza:
$$ \vec{I}_{t_0\rightarrow t} = \int_{t_0}^t \vec{F}dt'$$
e il fatto che la sua variazione sulle forze esterne equivale alla variazione della quantità di moto di un sistema:
$$ \vec{p}(t) - \vec{p}(t_0) = \int_{t_0}^t \vec{F}dt' $$

\par\smallskip
\textbf{Forze impulsive} \\
Una forza impulsiva è una forsa che viene esercitata per un tempo estremamente limitato. Spesso
l'intensità di una forza impulsiva $F_{imp}$ è talmente grande che il suo effetto viene comunque apprezzato,
e anzi è talmente predominante da permettere di trascurare gli effetti di altre forze, ovvero:
$$ \sum_{i=0}^n \vec{F}_i^{(est)} \approx \vec{F_{imp}} $$
A volte si conosce la variazione della quantità di moto di un corpo su cui ha agito una forza impulsiva per un
certo periodo di tempo, ma di cui è sconosciuta l'intensità. Si può allora determinare la forza media impulsiva
$<F_{imp}>$ che avrebbe sortito il medesimo effetto fosse stata applicata sull'intervallo $\Delta t$:
$$ <\vec{F_{imp}}> = \frac{\int_{t_i}^{t_f} \vec{F_{imp}}}{\Delta t} = \frac{\Delta \vec{p}}{\Delta t}$$
Facciamo un'esempio non intuitivo sulla forza media. Un corpo in movimento con velocità costante $v_1$ subisce ad un certo
punto un'impulso che lo spinge verso l'alto, dandogli nell'istante immediatamente successivo velocità uguale in modulo
a $v_1$ ma orientata di un'angolo $\theta$ rispetto all'orizzontale. Possiamo impostare:
$$ v_1 = (v_1, 0), \quad v_2 = (v_1\cos{\theta}, v_1\sin{\theta}) $$
Possiamo impostare la conservazione per trovare la forza media:
$$  <\vec{F_{imp}}> + mg = \frac{\int_{0}^{t} \vec{F}dt'}{dt} = \frac{m(v_1-v_2)}{t} $$
possiamo impostare quindi le equazioni sui diversi assi, ricordando che $mg$ agisce solamente sull'asse y!
$$ <\vec{F_{imp}}_x> = \frac{m(v_{2x} - v_{1x})}{t} = m(\frac{v_1(\cos{\theta}-1)}{t}) $$
$$ <\vec{F_{imp}}_y> = \frac{m(v_{2y} - v_{1y})}{t} = m(\frac{v_1\sin{\theta}}{t}) $$
\par\smallskip
\textbf{Collisioni} \\
Quando due punti materiali si avvicinano fra di loro, la loro mutuale interazione produce un cambiamento del loro stato
di moto, comportando:
\begin{itemize}
  \item Scambio di quantità di moto;
  \item Scambio di energia cinetica.
\end{itemize}
Quando la variazione di quantità di moto è notevole, mentre la durata dell'interazione molto piccola, si parla
di un urto (o collisione). Le forze che producono un urto sono forze impulsive e interne al sistema dei due
corpi che collidono, quindi come sempre possiamo trascurare l'effetto delle altre forze non impulsive. Cosa
ancorà più importante, visto che le forze impulsive sviluppate sono forze interne al sistema, la quantità
di moto totale del sistema si conserva sempre durante un urto. Come vedremo non sempre vale lo stesso per
l'energia cinetica. \\ Analizziamo adesso il moto di ude sfere che si incontrano l'un l'altra. Possiamo dividere
il moto in 3 fasi:
\begin{itemize}
  \item Nella prima fase, entrambi i corpi si muovono l'uno verso l'altro, imperturbati da qualsiasi interazione reciproca;
  \item la seconda fase avviene durante l'urto, in cui avviene l'interazione vera e propria fra i due corpi.
    In questa fase entrambi i corpi tendono ad occupare lo spazio altrui provocando deformazioni reciproche. Queste
    deformazioni portano allo sviluppo di una forza elastica che si oppone alla deformazione stessa e cerca di ristabilire
    l'equilibrio precedente, allontanando quindi le sfere. La variazione è brusca, ergo esiste una forza impulsiva.
  \item Nella fase immediatamente successiva all'urto, i due corpi si trovano nuovamente in moto imperturbati, ma con
    moti diversi a quelli di partenza.
\end{itemize}
\par\smallskip
\textbf{Classificazione delle collisioni} \\
Possiamo classificare gli urti in due categorie principali:
\begin{itemize}
  \item \textbf{Urti anelastici}, dove l'energia cinetica $K$ non si conserva, perchè le forze di deformazione
    trasformano l'energia meccanica in altre forme. $\vec{P}$ come sempre si conserva. Un caso particolare è l'urto completamente
    anelastico, dove dopo l'urto i corpi restano attaccati fra di loro.
  \item \textbf{Urti elastici}, dove sia $K$ che $\vec{P}$ si conservano.
\end{itemize}
\end{document}
