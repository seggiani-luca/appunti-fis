\documentclass[a4paper,12pt]{article}

\usepackage[french,italian]{babel}
\usepackage[T1]{fontenc}
\usepackage[utf8]{inputenc}
\frenchspacing 
\title{Appunti Fisica I}
\author{Luca Seggiani}
\date{7 Marzo 2024}

\begin{document}
\maketitle
\section{Moto circolare non uniforme}
Data la velocità:
$$ \vec{V} = \omega R \hat{\theta} $$
determiniamo l'accelerazione:
$$ \vec{a} = \frac{d\vec{V}}{dt} = \frac{d(\vec{w} \times \vec{R})}{dt} = \frac{d\vec{\omega}}{dt} \times \vec{R}
+ \vec{\omega} \times \frac{d\vec{R}}{dt} = \vec{\alpha} \times \vec{R} + \vec{\omega} \times (\vec{\omega} \times \vec{R}) $$
dove $\alpha$ è l'accelerazione angolare:
$$ \vec{\alpha} = \frac{d\vec{\omega}}{dt} \Rightarrow \alpha = \dot{\omega} = \ddot{\theta} $$
$$ \vec{\alpha} = (\alpha_r, \alpha_{\theta}, \alpha_z) = (0, 0, \ddot{\theta}) = \ddot{\theta}\hat{z} $$
$$ \vec{a} = \vec{\alpha} \times \vec{R} - \omega^2 R \hat{R} = \alpha\hat{z}\times \vec{R} - \omega^2 R \hat{R} = \alpha R \hat{\theta} - \omega^2 R \hat{R} $$
da cui:
$$ \vec{a} = (-\omega^2 R, \alpha R) $$
In sostanza, l'accelerazione angolare ha 2 componenti:
\begin{itemize}
  \item Componente radiale centripeta:
    $$ \omega^2 R$$
  \item Componente tangenziale dipendente dall'accelerazione angolare
    $$ \alpha R $$
\end{itemize}
A questo punto potremo ritrovare la velocità angolare come:
$$ \omega(t) = \omega_{t_0}  + \int_{t_0}^t \alpha(t')dt' $$
\section{Moto piano vario}
Vediamo adesso il caso generale dove sia la velocità angolare che la distanza dalla'origine sono funzioni di $t$:
\par\smallskip
\textbf{Velocità} \\
Posto $|\vec{R}| = R(t)$ e $\vec{\omega} = \omega(t) $:
$$ \vec{V} = \hat{\theta}V_{\theta} + \hat{R}V_R = \vec{\omega} \times \vec{\vec{R}} + \hat{R} \frac{d|\vec{R}|}{dt}
= \dot{R}\hat{R} + \vec{\omega} \times \vec{R} $$
Ovvero la velocità ha componenti polari:
$$ V_R = \dot{R}, \quad V_{\theta} = \omega R $$
Ciò si dimostra da:
$$ \vec{V} = \frac{d\vec{R}}{dt} = \frac{d}{dt}(R\cos{\theta}, R\sin{\theta}, 0) 
= (-R\hat{\theta}\sin{\theta}, R\hat{\theta}\cos{\theta}, 0) + (\dot{R}\cos{\theta}, \dot{R}\sin{\theta}, 0) $$
$$ = \dot{R}\hat{R} + \omega R \hat{\theta} = \dot{r} \hat{R} + R \omega \hat{z} \times \hat{R} = \dot{R}\hat{R} + R\vec{\omega} \times \hat{R}$$
possiamo inoltre dimostrare che, riguardo al solo versore $\hat{R}$:
$$ \frac{d\vec{R}}{dt} = \frac{d(R\hat{®})}{dt} = \hat{R}\frac{dR}{dt} + R\frac{d\hat{R}}{dt} = \hat{R}\dot{R} + R\frac{d\hat{R}}{dt}$$
da cui;
$$ \frac{d\hat{R}}{dt} = \hat{\theta}\dot{\theta} = \vec{\omega} \times \hat{R} $$
\textbf{Accelerazione} \\
Poniamoci adesso il problema di descrivere l'accelerazione di un moto sul piano qualsiasi, descritto da coordinate
polari:
$$ \vec{a} = \frac{d\vec{V}}{dt} = \frac{d(\vec{\omega} \times \vec{R} + \dot{R}\hat{R})}{dt} = \frac{d\vec{\omega}}{dt} \times \vec{R} + \vec{\omega} \times \frac{d\vec{R}}{dt} + \ddot{R}\hat{{R}} + \dot{R}\frac{d\hat{R}}{dt}$$
$$ = \vec{\alpha} \times \vec{R} + \vec{\omega} \times (\vec{\omega} \times \vec{R} + \dot{R}\hat{R}) + \ddot{R}\hat{R} + \dot{R}\vec{\omega} \times \hat{R} $$
usiamo la stessa formula per il prodotto vettoriale vista prima:
$$ \vec{a} = \vec{\alpha} \times \vec{R} - \omega^2 \vec{R} + \ddot{R}\hat{R} + 2\dot{R}\vec{\omega} \times \hat{R} $$
notando le solite relazioni coi prodotti vettoriali, ovvero:
$$ \vec{\alpha} \times \vec{R} = \alpha R \hat{z} \times \hat{R} = \alpha R \hat{\theta}, \quad \vec{\omega} \times \hat{R} = \omega \hat{z} \times \hat{R} = \omega \hat{\theta} $$
potremo riscrivere come:
$$ \vec{a} = (-\omega^2 R + \ddot{R})\hat{R} + (\alpha R + 2\dot{R}\omega)\hat{\theta} $$
ricordando che $\hat{R}$ è il versore in direzione radiale, e $\hat{\theta}$ quello in direzione tangenziale alla traiettoria del
nostro punto materiale.
\end{document}
