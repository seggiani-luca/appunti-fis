\documentclass[a4paper,12pt]{article}

\usepackage[french,italian]{babel}
\usepackage[T1]{fontenc}
\usepackage[utf8]{inputenc}
\frenchspacing 
\title{Appunti Fisica I}
\author{Luca Seggiani}
\date{22 Aprile 2024}

\usepackage{amsmath}

\begin{document}
\maketitle
\section{Introduzione all'elettrostatica}
L'elettrostatica è la branca della fisica che si occupa di cariche elettriche. Possiamo interpretare la carica
come la capacità di un corpo di esercitare forza elettrostatica.
\par\smallskip
\textbf{Carica} \\
La carica è una caratteristica di un corpo data dalla struttura molecolare, o meglio dall'interazione degli elettroni 
e dei protoni che costituiscono le sue molecole. La carica ha un segno: può essere positiva o negativa. Si decide per convenzione
la carica dell'elettrone essere negativa. Vale il principio secondo cui cariche dello stesso segno si respingono e cariche di segno
discorde si respingono. Può essere utile osservare brevemente la struttura dell'atomo:
\par\smallskip
\textbf{Introduzione informale all'atomo} \\
L'atomo è la particella fondamentale degli elementi chimici. E' formato da un nucleo di neutroni e protoni attorno al quale
orbitano gli elettroni. Queste tre particelle elementari hanno le seguenti caratteristiche:
\begin{center}
\begin{tabular}{|c|c|c|}
  Particella & Massa & Carica \\
  Protone & $1.673 \times 10^{-27} \  \mathrm{kg}$ & +1 $e$ \\
  Elettrone & $9.109 \times 10^{-31} \  \mathrm{kg}$ & - 1 $e$ \\
  Neutrone & $1.675 \times 10^{-27} \  \mathrm{kg}$ & 0 \\
\end{tabular}
\end{center}
Introduciamo un'unità di misura per la carica: il Coulomb (C). Si definisce la carica fondamentale come la carica di un protone o l'opposto della carica di un elettrone,
che vale $1.602 \times 10^-19$ C. Visto che la carica viene espressa da quantità
discrete di particelle elementari, è deducibile che essa sia quantizzata. \\
Possiamo poi dire che la dimensione di un nucleo di atomo è nell'ordine del Fermi, unità di misura corrispondente a $10^{-15}$ m,
mentre il raggio atomico è nell'ordine dell'angstrom, corrispondente invece a $10^{-10}$ m.
Dell'atomo possiamo definire il numero di massa A e il numero atomico Z. Il numero di massa corrisponde al numero
di particelle elementari (protoni e neutroni) che compongono il nucleo, mentre il numero atomico corrisponde al solo
numero di protoni. In generale è quindi vero che $A>Z$. Si ha poi che il numero di elettroni, in un atomo con carica neutra,
è uguale al numero di protoni. Il numero atomico caratterizza periodicamente gli elementi chimici: si ha che un atomo con un numero di massa (quindi
di neutroni) diverso da quello del suo elemento è un isotopo, e un atomo con numero di elettroni diverso da quello dei suoi protoni è uno ione (che può essere positivamente carico,
quando perde elettroni, e negativamente carico, quando ne acquista).
\par\smallskip
\textbf{Materiali neutri e a reticolo polare} \\
Si possono distinguere comportamenti diversi della materia:
\begin{itemize}
  \item \textbf{Neutro}
    Di materiali caratterizzati da legami covalenti, e che quindi hanno una distribuzione neutra di carica.
  \item \textbf{Polare}
    Di materiali caratterizzati da legami ionici, dove alcuni ioni hanno carica positiva e altri carica negativa. Nel complesso,
    in genere, la carica è neutra, ma si potrà polarizzare il materiale in modo da renderlo localmente carico in una certa zona.
\end{itemize}
\par\smallskip
\textbf{Elettricità statica in una bacchetta} \\
Poniamo di avere una bacchetta di vetro e di strofinarla contro un panno di lana. Saremo adesso in grado di sollevare
dei piccoli pezzi di carta semplicemente avvicinandogli la bacchetta. Questo semplice esperimento può essere spiegato dall'elettrostatica:
strofinando la bacchetta sul panno, abbiamo costretto i due corpi a scambiarsi cariche. La carica complessiva resterà costante, per il principio di
conservazione della carica: la bacchetta cederà carica negativa al panno e si caricherà quindi positivamente. Quando la avviciniamo alla carta, la bacchetta,
ora carica positivamente, indurrà nella carta una carica negativa sul lato superiore (e di conseguenza una carica positiva sull lato inferiore). 
Questo produrrà una forza di attrazione (forza di Coloumb) che vincerà l'attrazione gravitazionale,
portando i pezzi di carta a sollevarsi.
\par\smallskip
\textbf{Induzione elettrostatica} \\
Il comportamento appena dimostrato dalla carta è un'esempio di induzione elettrostatica. Supponiamo di avere una sbarretta metallica:
avvicinando una carica positiva alla sua destra, o una negativa alla sua sinistra, costringiamo la carica positiva ad accumularsi a destra
e quella negativa a sinistra, caricando localmente la sbarretta. Nella pratica, la variazione di carica è data dalla migrazione degli elettroni,
liberi di spostarsi nel metallo, da destra verso sinistra: gli ioni rimasti a destra saranno carichi positivamente. Se a questo punto
dividessi l'oggetto a metà, otterrei due corpi carichi globalmente (o almeno che non potrebbero trasferire fra di loro carica per contatto).
\par\smallskip
\textbf{Distribuzione simmetrica di carica} \\
Intuitivamente, una sfera carica positivamente tende a concentrare le sue cariche verso l'esterno, in modo da minimizzare le forze repulsive. In modo analogo a prima,
non sono le "cariche" in sé per sé a muoversi, ma gli elettroni che vengono trasferiti dall'esterno verso l'interno, lasciando gli ioni sul bordo
carichi positivamente.
\par\smallskip
\textbf{Bilancia a torsione} \\
L'attrazione o repulsione elettromagnetica (quindi la forza elettromagnetica) può essere misurata attraverso 
uno strumento noto come bilancia a torsione. La bilancia a torsione è un'apparato formato da una sbarra avente due sferette cariche alle estremità:
la sbarra può ruotare su se stessa rispetto ad un asse verticale, torcendo il perno che la tiene ferma. Avvicinando cariche diverse alle sferette,
possiamo osservare una rotazione della sbarra proporzionale alla forza elettromagnetica esercitata dalle due sferette. Chiaramente, se le due sferette
tenderanno ad avvicinarsi, le loro cariche saranno opposte, uguali se viceversa. Un'apparato simile fu usato da Cavendish per misurare la costante gravitazionale
$G$, usando però masse molto più grandi.
\section{Forza di Coulomb}
La forza di Coulomb è la forza che due cariche esprimono fra di loro. Poste due cariche $a$ e $b$, 
la sua espressione è:
$$ \vec{F_{ab}} = k \frac{q_a\cdot q_b}{r^2}\hat{r}$$
Questa sarà la forza espressa su $b$ da $a$. $q_a$ e $q_b$ sono rispettivemente le cariche di $a$ e $b$, $r$ distanza
fra di loro, e $\hat{r}$ il vettore:
$$ \hat{r} = \frac{\vec{r_b} - \vec{r_a}}{|\vec{r_b} - \vec{r_a}|} $$
Notiamo che questa forza è espressa come inverso del quadrato delle distanze, proprio come lo era l'attrazione gravitazionale. La differenza
sta nel fatto che la forza di Coulomb può avere segno negativo, quindi respingere oltre che attrarre. Ovvero,
rispetta il terzo principio della dinamica:
$$ \vec{F_{ab}} = -\vec{F_{ba}} $$
e le caratteristiche di repulsività e attrattività che avevamo prima riportato: con $qA$ e $qB$ discordi, dirige 
l'uno verso l'altro corpo. In caso contrario, li allontana. Vediamo la costante $k$:
$$ k = \frac{1}{4\pi \epsilon_0} $$
dove $\epsilon_0$ è la \textbf{costante dielettrica nel vuoto}, anche detta permettività dielettrica nel vuoto, che
vale $9 \times 10^{19} \frac{\mathrm{Nm}^2}{\mathrm{C}^2}$.
\par\smallskip
\textbf{Principio di sovrapposizione} \\
Nel caso su una carica agiscano più cariche, la forza risultante sarà semplicemente la somma delle forze rispetto ad ogni singola carica:
$$ \vec{F} = k \sum_i^n\frac{q_1 q_i}{r_i^2}\hat{r_i} $$
Nel caso le cariche non siano allineate, occorrerà dividere la forza nelle componenti relative agli assi, e trovare quindi il vettore risultante.
\section{Campo elettrico}
Dal principio appena enunciato ricaviamo che la forza di Coulomb relativa a una singola carica $q_1$ è:
$$ \vec{F} = k \sum_i^n\frac{q_1 q_i}{r_i^2}\hat{r_i} = q_1 \sum k\frac{kq_i}{r_i^2}\hat{r_i} $$
Diciamo che $q_1$ è la nostra carica di prova, dal valore di 1 C. Possiamo allora dividere entrambi i lati dell'eguaglianza per $q_1$:
$$ \frac{\vec{F}}{q_1} = \sum k\frac{kq_i}{r_i^2}\hat{r_i} = \vec{E}$$
chiamiamo questa nuova misura campo elettrico. Il campo elettrico è un campo, ovvero una funzione:
$$ \vec{E}: \mathbf{R}^3 \rightarrow \mathbf{R}^3 $$
che associa ad ogni punto nello spazio un vettore, che rappresenta la forza che subirebbe una carica di 1 C in quel punto. Questa definizione
operativa può essere usata nella pratica per calcolare il campo elettrico in un punto: si collochi una carica nello spazio, ne se misuri la forza risultante,
il quoziente fra questi due valori sarà il campo elettrico in quel punto. Converrà che la nostra carica di prova sia piccola in modo
da non influire sulla configurazione di cariche che vogliamo analizzare.
\par\smallskip
\textbf{Campo generato da una carica} \\
Analizziamo un caso ad alta simmetria: una singola carica nello spazio. Applicando la definizione, abbiamo che il campo nei pressi
della carica sarà:
$$ \frac{F}{q_p} = \frac{k \frac{q_p q_1}{r^2}\hat{r}}{q_p} = k \frac{q_1}{r^2}\hat{r}$$
Il campo elettrico avrà quindi simmetria sferica attorno al punto generatore del campo stesso. Per avere una visione
migliore del campo, introduciamo le:
\par\smallskip
\textbf{Linee di campo} \\
Le linee di campo di un campo vettoriale sono un'insieme di linee parallele in ogni punto al campo vettoriale. Possiamo usare le linee
di campo per visualizzare gli andamenti dei campi.

\end{document}
