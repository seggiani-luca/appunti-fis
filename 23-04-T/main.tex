\documentclass[a4paper,12pt]{article}

\usepackage[french,italian]{babel}
\usepackage[T1]{fontenc}
\usepackage[utf8]{inputenc}
\frenchspacing 
\title{Appunti Fisica I}
\author{Luca Seggiani}
\date{24 Aprile 2024}

\begin{document}
\maketitle
\par\smallskip
\textbf{Dipolo elettrico} \\
Consideriamo il campo elettrico generato da una coppia di cariche $q$ di segno opposto $+q$ e $-q$ separate da una distanza $d$. Sia l'origine del piano cartesiano coincidente col centro del dipolo: avremo che $+q$ ha coordinata
$(0, 0, \frac{d}{2})$, e $-q$ ha coordinata $(0, 0, -\frac{d}{2})$. Calcoliamo il campo lungo l'asse del dipolo z e lungo un'asse ad esso perpendicolare (prendiamo x, y sarebbe equivalente).
\begin{itemize}
  \item \textbf{Lungo z}: sarà il caso più semplice, basterà applicare la formula del campo elettrico generato da una carica ad entrambe le cariche:
    $$ \vec{E} = \sum k\frac{kq_i}{r_i^2}\hat{r_i} $$
    considerando la distanza $\pm\frac{d}{2}$ dall'origine lungo z:
    $$ \vec{E}(0, 0, z) = \frac{kq\hat{z}}{\left(z-\frac{d}{2}\right)^2} - \frac{kq\hat{z}}{\left(z+\frac{d}{2}\right)^2} = kq\hat{z}\left(\frac{1}{\left(z-\frac{d}{2}\right)^2} - \frac{1}{\left(z+\frac{d}{2}\right)^2}\right) $$
    $$ = kq\hat{z}\left(\left(z-\frac{d}{2}\right)^{-2} - \left(z+\frac{d}{2}\right)^{-2}\right) = \frac{kq\hat{z}}{z^2}\left( \left(1 - \frac{d}{2z}\right)^{-2} - \left(1 + \frac{d}{2z}\right)^{-2} \right) $$
    A noi interessa il caso in cui $z$ sia molto maggiore di $d$, ovvero $z >> d$. Questo ci permette di usare l'approssimazione:
    $$ z >> d \Rightarrow \left(1 + \frac{d}{2z}\right)^{-2} \approx \left(1 - 2\frac{d}{2z}\right) = \left(1 - \frac{d}{z}\right) $$
    Questo ci dà il risultato:
    $$ \frac{kq\hat{z}}{z^2}\left( \left(1 - \frac{d}{2z}\right)^{-2} - \left(1 + \frac{d}{2z}\right)^{-2} \right) \approx \frac{kq\hat{z}}{z^2}\left(\left(1+\frac{d}{z}\right) - \left(1-\frac{d}{z}\right)\right) = 2\frac{kqd}{z^3}\hat{z} $$
    Ciò significa che a grandi distanze $z$, o a distanze $d$ trascurabili, il campo elettrico a $z$ dal dipolo lungo il suo asse vale $2\frac{kqd\hat{z}}{z^3}$ $\frac{N}{C}$. Definiamo allora la quantità $p$, detta \textbf{momento di dipolo}, come:
    $$ p = qd, \quad \vec{E}(0, 0, z) = 2\frac{kqd}{z^3}\hat{z} = 2\frac{kp}{z^3}\hat{z} $$
    Che ci dà il risultato finale. Notiamo $\hat{z}$ versore essere diretto verso $\vec{z}$ ovvero lo spostamento: in sostanza, il campo del dipolo è sempre diretto nella stessa direzione. Se siamo vicini al polo positivo, tenderemo ad allontanarci dal dipolo;
    se siamo vicini al polo negativo, tenderemo ad avvicinarci.
  \item \textbf{Lungo x/y}: assumiamo il caso x essere uguale a y: in entrambi i casi il campo sarà calcolato sul piano ortogonale al dipolo. Calcoliamo, come prima, il campo generato dalle cariche: avremo che la componente lungo x,
    parallela all'asse del dipolo, è nulla (le componenti ortogonali si annullano fra di loro). A restare sarà solamente la componente parallela all'asse del dipolo, che avrà il valore di due volte il campo generato da una singola carica in quel punto, negativo.
    Calcoliamo quest'ultimo valore: è utile costruire un triangolo rettangolo con un cateto sulla semidistanza fra i poli e l'altro sulla distanza tra il centro del dipolo e il punto $(x, 0, 0)$. A questo punto la distanza $r$ sarà l'ipotenusa del triangolo:
    $$ r = \sqrt{x^2 + \left(\frac{d}{2}\right)^2} $$
    e il campo:
    $$ \vec{E}(x, 0, 0) = -2\frac{kq\hat{z}}{\left(x^2 + \frac{d^2}{4}\right)} $$
    A noi interessa soltanto il campo lungo $z$, ergo dobbiamo moltiplicare il campo per il seno dell'angolo $\alpha$ che l'ipotenusa forma sull'asse $x$. $\sin{\alpha}$ è dato da cateto minore su ipotenusa, ergo:
    $$ \sin{\alpha} = \frac{\frac{d}{2}}{\sqrt{x^2+ \frac{d^2}{4}}}, \quad \vec{E}(x, 0, 0) = -2\frac{kq\hat{z}}{\left(x^2 + \frac{d^2}{4}\right)} \frac{\frac{d}{2}}{\sqrt{x^2+ \frac{d^2}{4}}} = -\frac{kqd\hat{z}}{\left( x^2 + \frac{d^2}{4} \right)^{\frac{3}{2}}} $$
    Come prima, imponiamo il fatto che x è molto maggiore di d:
    $$ x >> d \Rightarrow \vec{E}(x, 0, 0) = -\frac{kqd}{x^3}\hat{z} = -\frac{kp}{x^3}\hat{z} $$
    Notiamo dal risultato finale che il campo lungo l'asse ortogonale ad un dipolo è antiparallelo all'asse del dipolo stesso.
\end{itemize} 
\section{Densità di carica}
Passiamo al calcolo del campo elettrico su distribuzioni continue di cariche.
\begin{itemize}
  \item \textbf{Densità (di volume) di carica} \\
    Consideriamo l'incremento di carica $dq$ sul volume infinitesimo $dV$. Possiamo definire una densità $\rho_q$ di carica tale per cui:
    $$ \rho_q = \frac{dq}{dV} $$
    si misura in $\frac{C}{m^3}$. Possiamo calcolare il campo elettrico dalla densità di carica su un infinitesimo di volume in $\vec{r}$ visto da $\vec{r'}$ applicando la forza di Coulomb, come:
    $$ d\vec{E}(\vec{r}) = \frac{k\rho_q{\vec{r'}}dV}{(r-r')^3}(\vec{r}-\vec{r'})$$
    Dove notiamo il cubo al denominatore è li solamente per normalizzare $(\vec{r} - \vec{r'})$ in $(\hat{r-r'})$ versore. A questo punto potremo passare alla distribuzione completa di cariche sul volume $V$ prendendo l'integrale:
    $$ \vec{E}(\vec{R}) = \int_V \frac{k\rho_q{\vec{r'}}}{(r-r')^3}(\vec{r}-\vec{r'})dV $$
  \item \textbf{Densità superficiale di carica} \\
    Similmente, possiamo definire la densità superficiale di carica:
    $$ \sigma_q = \frac{dq}{dA} $$
    si misura in $\frac{C}{m^2}$.
  \item \textbf{Densità lineare di carica} \\
    Infine, possiamo definire la densità lineare di carica:
    $$ \lambda_q = \frac{dq}{dl} $$
    si misura in $\frac{C}{m}$. 
\end{itemize}
Vediamo alcune applicazioni della densità di carica per descrivere il campo generato da diversi oggetti:
\begin{itemize}
  \item \textbf{Anello uniformemente carico} \\
    Calcoliamo la campo generato da un'anello metallico $A$ uniformemente carico, giaciente sul piano xy dello spazio, lungo il suo asse di simmetria centrale. Consideriamo l'oggetto solo nella sua lunghezza, ergo usiamo la densità lineare di carica $\lambda_q$. Anche
    in questo caso è efficace costruire un triangolo rettangolo, con un cateto posto sulla distanza fra l'origine il punto $(0,0,z)$, e l'altro posto sul raggio dell'anello. Avremo che le componenti lungo il piano parallelo a xy si annullano, e l'unica risultante è quella in z.
    Calcoliamo allora il campo infinitesimo come il campo generato sul punto $(0,0,z)$ per il coseno dell angolo fra l'asse z e la congiungente del punto con un altro punto sul raggio, ovvero il rapporto fra la distanza $z$ e l'ipotenusa del triangolo:
    $$ d\vec{E}(0,0,z) = \frac{k\,dq\,\hat{z}}{(r^2+z^2)} \frac{z}{\sqrt{r^2+z^2}} $$
    Calcoliamo allora il campo totale come l'integrale:
    $$ \vec{E}(0,0, z) = \int_A  \frac{k\hat{z}}{(r^2+z^2)} \frac{z}{\sqrt{r^2+z^2}}\lambda_q dl $$
    Notiamo che nessun termine dentro l'integrale dipende da $l$, e $\int_A dl$ non è altro che $2\pi r$, ergo:
    $$ \vec{E}(0,0,z) =  \frac{k\hat{z}}{(r^2+z^2)} \frac{z}{\sqrt{r^2+z^2}}\lambda_q \int_A dl = \frac{k 2\pi r \lambda_q \hat{z}}{(r^2+z^2)} \frac{z}{\sqrt{r^2+z^2}} $$
    Abbiamo che la carica totale sull'anello è:
    $$ Q = 2\pi r \lambda_q, \quad \vec{E}(0,0,z) = \frac{kzQ}{(r^2+z^2)^\frac{3}{2}}\hat{z} $$
    Da cui la soluzione esatta. Possiamo ora considerare il comportamento "da lontano" dell'anello, applicando la stessa approssimazione di prima:
    $$ z >> r \Rightarrow \vec{E}(0,0,z) = \frac{kQ}{z^2}\hat{z} $$
    Puè essere interessante fare la considerazione inversa: in punti molto vicini all'anello, ovvero dove $z << r$, si ha:
    $$ z << r \Rightarrow \vec{E}(0,0,z) = z\frac{kQ}{r^3} $$
    Si nota che per una carica $-q$ di segno opposto alla $Q$ dell'anello, la forza ha effettivamente la forma della legge di Hooke, e forma un'oscillatore armonico.
  \item \textbf{Disco uniformemente carico}
    Vediamo un caso simile: il campo generato da un disco $D$ uniformemente carico, giaciente sul piano xy dello spazio, lungo il suo asse di simmetria centrale. In questo caso considereremo l'oggetto sulla sua superficie, ergo useremo la densità superficiale
    di carica $\sigma_q$. Consideriamo lo stesso schema precedente, ma variando $r$ dal centro della circonferenza fino al suo bordo: avremo che su un frammento infitesimo del disco (ergo una circonferenza con lunghezza $2\pi r$) il campo generato sarà:
    $$ d\vec{E}(0,0,z) = \frac{kz\hat{z}}{(r^2+z^2)^\frac{3}{2}} 2\pi r\sigma_m \,dr $$
    Troviamo allora il campo totale prendendo l'integrale:
    $$ \vec{E}(0,0,z) = \int_0^R \frac{kz\hat{z}}{(r^2+z^2)^\frac{3}{2}} 2\pi r\sigma_m \,dr = kz\sigma_m\hat{z}\pi \int_0^R \frac{2r\,dr}{(r^2+z^2)^\frac{3}{2}} $$
    $$ = kz\sigma_m\hat{z}\pi \left[ -\frac{2}{\sqrt{r^2+z^2}} \right]_0^R = kz\sigma_m\hat{z}\pi \left(\frac{2}{z} - \frac{2}{\sqrt{R^2 + z^2}}\right) $$
    $$ = 2k\sigma_m\pi\left(1 - \frac{z}{\sqrt{R^2+z^2}}\right) $$
    Da cui l'equazione finale. Notiamo come per:
    $$ \lim_{R\rightarrow \infty} \vec{E} = 2k\sigma_m = \frac{\sigma_m}{2\epsilon_0} $$
    è il campo generato da un piano infinito (a qualunque distanza!), che è una buona approssimazione per l'armatura di un condensatore lontano dai bordi.
  \item \textbf{Filo uniformemente carico} \\
    Come ultimo esempio prendiamo un filo di lunghezza $2l$, centrato sull'origine, e calcoliamone il campo a una certa distanza $x$. Applicando la forza di Coulomb osserviamo che di poter prendere due punti, ad un capo e all'altro del filo,
    e calcolarne il contributo: se prendiamo y uguali ma opposte abbiamo che le forze si annullano lungo l'asse verticale z, e la risultante è tutta diretta verso il piano ortogonale (prendo x) con modulo pari a 2 volte la componente x della forza
    esercitata da un singolo punto. Notiamo che questo approccio ci permette però di trovare la forza esercitata da 2 punti per volta, ergo dividiamo per due per ottenere il contributo:
    $$ d\vec{E}(x,0,0) = \frac{k\lambda dz}{z^2+x^2} \frac{x}{\sqrt{z^2+x^2}}\hat{x} $$
    Dove il primo denominatore è il quadrato della distanza del punto sul filo da quello dove calcoliamo il campo, ovvero l'ipotenusa di un triangolo costruito sull'asse x con cateti $l$ e $x$, e la seconda frazione rappresenta il coseno dell'angolo formato
    dall'ipotenusa e l'asse x. Prendiamo allora l'integrale per avere il campo totale:
    $$ \vec{E}(x,0,0) = \int_{-l}^l \frac{k\lambda dz}{z^2+x^2} \frac{x}{\sqrt{z^2+x^2}}\hat{x} = k\lambda x\hat{x} \int_{-l}^l \frac{dz}{(z^2+x^2)^\frac{3}{2}} $$
    L'integrale calcolato si risolve attraverso sostituzione trigonometrica, e dà il risultato:
    $$ \int \frac{dx}{(x^2+a^2)^{\frac{3}{2}}} = \frac{x}{a^2\sqrt{x^2+a^2}} + c $$
    con cui possiamo quindi risolvere:
    $$ \vec{E}(x,0,0) = k\lambda x\hat{x} \int_{-l}^l \frac{dz}{(z^2+x^2)^\frac{3}{2}} = k\lambda x \hat{x} \left[\frac{z}{x^2\sqrt{z^2+x^2}}\right]_{-l}^l$$
    $$ = k\lambda x\hat{x} \frac{2l}{x^2\sqrt{l^2+x^2}} = k\lambda\frac{2l}{x\sqrt{l^2+x^2}}\hat{x} $$
    Che è la formula per il campo generato dal filo di lunghezza finita. Sostituiamo a $k$ il suo valore, ovvero $\frac{1}{4\pi \epsilon_0}$:
    $$ \vec{E}(x,0,0) = k\lambda\frac{2l}{x\sqrt{l^2+x^2}}\hat{x} = \frac{l\lambda}{2\pi \epsilon_0 x\sqrt{l^2+x^2}} $$
    e vediamo il caso per un filo di lunghezza infinita prendendo il limite:
    $$ \lim_{l\rightarrow\infty} \frac{l\lambda}{2\pi \epsilon_0 x\sqrt{l^2+x^2}} = \frac{\lambda}{2\pi \epsilon_0 x} $$
\end{itemize}
\section{Flusso del campo elettrico}
Si introduce adesso il concetto di flusso di un campo vettoriale, e in particolare di flusso del campo elettrico. Si inizia con un esempio piuttosto semplice:
\par\smallskip
\textbf{Flusso lungo una superficie piana} \\
Prendiamo una superficie piana $A$, immersa in un campo vettoriale $\vec{E}$. Definiamo un vettore $\vec{A}$ corrispondente alla normale della superficie, con modulo pari alla sua area. Definiremo allora il flusso del campo sulla superficie come:
$$ \Phi_E = EA $$
Ruotiamo adesso la superficie di un certo angolo $\theta$ (o ruotiamo il campo dello stesso angolo, poco cambia). Avremo che il flusso non è più attraverso l'intera supeficie, ma attraverso una sua frazione, data dal prodotto scalare:
$$ \Phi_E = EA\cos{\theta} = \vec{E} \cdot \vec{A} $$
\par\smallskip
\textbf{Flusso su una superficie} \\
Le definizioni date finora sono valide su superfici localmente piane, ma spesso saremo interessati a calcolare flussi su superfici anche irregolari. Definiamo quindi, sulla base della scorsa formula, il flusso su un'elemento infinitesimo
di superficie $d\vec{A}$:
$$ d\Phi_E = \vec{E} \cdot d\vec{A} $$
Questo ci permette di calcolare il flusso lungo un'intera superficie $\Omega$, prendendo l'\textbf{integrale di superficie}:
$$ \Phi_E = \int_\Omega \vec{E} \cdot d\vec{A} $$
Notiamo che il flusso può avere verso positivo (superficie e campo sono allineati), verso negativo (l'esatto contrario) e, intuitivamente, può essere nullo (nel caso in cui il campo scorra perpendicolare alla superficie).
\par\smallskip
\textbf{Flusso del campo elettrico} \\
Il flusso può essere usato sul campo elettrico: cariche $q$ influenzeranno il campo elettrico $\vec{E}$ che avrà un certo flusso $\Phi_E$ su diverse superfici. Sarà a noi utile considerare superfici chiuse, ovvero superfici che delimitano
porzioni finite di spazio. Considerata una certa superficie chiusa $\Omega$, vediamo l'impatto sul flusso della stessa
che hanno cariche poste al suo interno e al suo esterno.
\begin{itemize}
  \item \textbf{Cariche esterne}: abbiamo una carica $q$ posta esternamente alla nostra superficie $\Omega$. Possiamo dividere il flusso lungo $\Omega$ in flusso in entrata e in uscita, ovvero il flusso del campo \textit{entrante} nella superficie
    e il flusso del campo \textit{uscente} dalla superficie:
    $$ d\Phi_E = d\Phi_{Ei} + d\Phi_{Ef} = -|d\vec{A_i} \cdot \vec{E_i}| + |d\vec{A_u} \cdot \vec{E_u}| $$
    $$ = -|d\vec{A_i} \cdot \vec{E_i}| + d\vec{A_i}\left(\frac{du}{di}\right)^2 \cdot \vec{E_i}\left(\frac{di}{du}\right)^2 = 0 $$
    ovvero, tutte le linee di campo che entrano nella superficie prima o poi escono dalla superficie, e il flusso netto vale necessariamente 0. I differenziali quadri rappresentano il fatto che il campo scala col quadrato della variazione dell'elemento di superficie,
    la superficie l'esatto opposto. In sostanza, il campo di una carica esterna alla superficie, una volta uscito, avrà valore proporzionale all'inverso del quadrato della distanza su un area che scala come il quadrato della distanza:
    i due effetti si annullano e il flusso totale risulta nullo.
  \item \textbf{Cariche interne}: poniamo adesso $q$ all'interno della superficie, e assumiamo temporanamente la suddetta essere sferica. Applicando la definizione di flusso abbiamo:
    $$ \Phi_E = \int_\Omega \vec{E} \cdot d\vec{A} = \vec{E} \int_\Omega d\vec{A} $$
    Visto che il campo, dipendente dal raggio, non varia su una sfera. Il secondo termine sarà semplicemente la superficie della sfera stessa. Applichiamo quindi questo e la definizione di campo sulla forza di Coulomb:
    $$ \Phi_E = k \frac{q}{r^2} (4\pi r^2) = 4\pi kq $$
    Questo spiega perchè $ k = \frac{1}{4\pi \epsilon_0}$:
    $$ \Phi_E = 4\pi kq = \frac{q}{\epsilon_0}$$
    Quest'ultimo risultato, fondamentale, prende il nome di:
\end{itemize}
\section{Teorema di Gauss}
Abbiamo appena dimostrato l'assunto fondamentale del teorema di Gauss: su una superficie sferica, il flusso del campo elettrico è:
$$ \Phi_E = \frac{q}{\epsilon_0}$$
Facciamo alcune generalizzazioni. Innanzitutto, generalizziamo il risultato a superfici chiuse di qualsiasi forma: notiamo che tutte le linee di campo uscenti dalla sfera sono anche uscenti da una qualsiasi superficie chiusa, ergo il teorema vale per
ognuna. Infine, possiamo dire che il risultato vale per qualsiasi combinazioni di cariche \textit{interne} alla superficie, ergo:
$$ \Phi_E = \sum_i^n \frac{q_i}{\epsilon_0} $$
per tutte le cariche $i < n$ comprese all'interno della superificie. Ricordiamo che le cariche esterne non hanno invece alcun contributo.
\end{document}
