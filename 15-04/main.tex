\documentclass[a4paper,12pt]{article}

\usepackage[french,italian]{babel}
\usepackage[T1]{fontenc}
\usepackage[utf8]{inputenc}
\frenchspacing 
\title{Appunti Fisica I}
\author{Luca Seggiani}
\date{15 Aprile 2024}

\begin{document}
\maketitle
\section{Dinamica rotazionale del corpo rigido}
Le forze agenti su un corpo rigido producono, oltre che una traslazione, anche una rotazione. Questa messa
in rotazione dipende dal punto di applicazione delle forza rispetto all'asse di rotazione. Ad esempio, immaginiamo
il banale gesto di aprire una porta: chiaramente risulterà più facile mettere la porta in rotazione spingendo sulla maniglia,
che non sui cardini. Possiamo quindi dire che la nostra azione sulla maniglia produce sulla porta un'accelerazione angolare,
accelerazione che è direttamente proporzionale al modulo della forza applicata, al seno che essa forma con l'asse
orizzontale della porta, e sopratutto alla distanza fra l'asse di rotazione e il punto di applicazione della forza:
$$ \alpha \propto |\vec{F}|R\sin{\theta}$$
notiamo che $R\sin{\theta}$, che chiameremo $d$ braccio della forza, rappresenta la lunghezza del segmento
perpendicolare alla proiezione dell'asse di applicazione delle forza che si congiunge al centro dell'asse di rotazione.
Questa grandezza sarà uguale a $\vec{R}$ nel caso la forza venga applicata al corpo nella direzione perfettamente perpendicolare
all'asse. Negli esempi successivi verrà assunto questo caso: in caso contrario servirà considerare $\vec{R}$ per quello che dovrebbe
veramente essere, ovvero il braccio $d = R\sin{\theta}$.
A questo punto $\alpha$ sarà perpendicolare ad $\vec{F}$ e $\vec{R}$, e quindi diretta lungo l'asse di rotazione.
\par\smallskip
\textbf{Momento di una forza} \\
Definiamo il momento di una forza rispetto a un certo polo $O$:
$$ \vec{\tau_0} = \vec{R} \times \vec{F} $$
$\vec{\tau}$ è una grandezza fisica vettoriale, misurata in $\mathrm{N}\cdot\mathrm{m}$, e non in Joule, in
quanto non ha nulla ha che fare con il lavoro. Sarà nulla nel caso la forza sia parallela a $\vec{R}$. Nel
caso si parli di rotazione assiale, definiamo poi il momento assiale specifico a quel singolo asse:
$$ \tau_z = (\vec{R} \times \vec{F})_Z$$
Il momento della forza è positivo nel caso se causa una rotazione in senso orario, negativo se in senso antiorario.
La componente lungo l'asse di rotazione del momento di una forza nota anche come momento assiale o momento torcente.
\par\smallskip
\textbf{Momento angolare} \\
Definiamo adesso il momento angolare rispetto ad un polo per un punto materiale. Questa grandezza è nota anche come
momento della quantità di moto:
$$ \vec{L_0} = \vec{R} \times m\vec{V} = \vec{R} \times \vec{P} $$
$\vec{L_0}$ ci dà informazioni riguardo all'effettiva rotazione di un punto attorno al polo ($\omega$ diretta lungo $\vec{R}\times\vec{V}$),
è una grandezza fisica vettoriale misurata in $\mathrm{J}\cdot\mathrm{s}$. Per un moto piano attorno al polo $O$ avremo:
$$ \vec{L_z} = mV_{\theta}R = mR^2\omega$$
sul solo asse z. Anche in questo caso otteniamo un vettore perpendicolare al piano della rotazione.
\par\smallskip
\textbf{Teorema del momento angolare} \\
La derivata rispetto al tempo del momento angolare di un punto materiale è uguale alla somma vettoriale dei momenti agenti
sul piano materiale:
$$ \vec{\tau}_{tot} = \frac{d\vec{L_0}}{dt} = \sum \vec{\tau_i} $$
N.B.: questo quando momento angolare e momento delle forze sono definiti rispetto allo stesso polo $O$!\\
Possiamo dimostrare questo risultato con dei semplici passaggi algebrici:
$$ \frac{d\vec{L_0}}{dt} = \frac{d}{dt} (\vec{R} \times m\vec{V}) = (\frac{d\vec{R}}{dt}) \times m\vec{V} + \vec{R} \times \frac{d}{dt}(m\vec{V}) = \vec{V}\times m\vec{V}+\vec{R}\times\sum{\vec{F}} $$
Notiamo che $\vec{V} \times m\vec{V} = 0$, ergo:
$$ \frac{d\vec{L_0}}{dt} = \sum \vec{R} \times \vec{F} = \tau_{tot} $$
\par\smallskip
\textbf{Legge di conservazione del momento angolare} \\
Una conseguenza importante del teorema appena dimostrato è che:
$$ \vec{\tau}_{tot} = \frac{d\vec{L_0}}{dt} \Rightarrow \vec{\tau}_{tot} = 0 \Leftrightarrow \vec{L} = \mathrm{const.}$$
ovvero, il momento della quantità di moto (momento angolare) è costante nel caso non agiscano altre forze esterne con momento $\tau>0$.
\par\smallskip
\textbf{Forze centrali} \\
Le forze centrali sono forze che possono essere scritte nella formula generica:
$$ \vec{F}(R) = \hat{R}f(R)$$
dove $f(R)$ è una funzione scalare della distanza dal centro di forza e il punto di applicazione della forza stessa. Esempi
di queste forze sono:
\begin{itemize}
  \item La forza gravitazionale;
  \item La forza coloumbiana
  \item La forza elastica;
  \item ecc...
\end{itemize}
Se cerchiamo di calcolare il momento di una forza centrale, troviamo subito che tale momento è nullo, in quanto $\vec{R}$
e $\hat{R}$ sono paralleli per definizione. In altre parole, sotto l'effetto di forze centrali il momento angolare si conserva,
ovvero è una costante del moto.
\par\smallskip
\textbf{Momento angolare di un sistema di punti materiali} \\
Possiamo estendere il concetto di momento angolare a sistemi di punti materiali:
$$ \vec{L} = \sum_i^n \vec{L_i} = \sum_i^n \vec{R_i} \times \vec{P_i} $$
Sempre rispetto al solito polo $O$ per tutti i punti. Questo ci permette di enunciare:
\par\smallskip
\textbf{Seconda equazione cardinale della meccanica} \\
$$ \vec{\tau}_{est} = \sum_i^n \vec{R_i} \times \vec{F}_{est_i} = \frac{d\vec{L}}{dt} $$
Possiamo dimostrare questo risultato:
$$ \vec{L} = \sum_i^n \vec{L_i} = \sum_i^n \vec{R_i} \times \vec{P_i}, \quad \frac{d\vec{L}}{dt} = \sum_i^n \frac{d\vec{L_i}}{dt} = \sum_i^n \tau_i = \sum_i^n (\tau_{int_i} + \tau_{est_i}) = \tau_{est} $$
Questo vale e si dimostra esattamente nello stesso modo per sistemi di punti materiali.
\par\smallskip
\textbf{Applicazione del momento angolare al pendolo} \\
Il momento delle forze e il momento angolare ci permettono di descrivere il moto del pendolo senza dover ricorrere
all'accelerazione centripeta. Posto un pendolo formato da una massa $M$ legata da un filo di lunghezza $l$ ad un perno
centrale, potremo calcolare il momento della risultante delle forze sulla massa, ovvero la gravità $-Mg$ e la tensione
del filo $\vec{T}$:
$$ \tau_{tot} = \vec{R} \times \vec{F_T} + \vec{R} \times \vec{F_{Mg}} = \vec{R} \times Mg\sin{\theta} = -Mgl\sin{\theta} $$
Calcoliamo poi il momento angolare della massa:
$$ \vec{L} = \vec{R} \times M\vec{V} = \vec{R} \times M(\omega \times \vec{R}) = M\omega l^2 $$
possiamo calcolarne la derivata:
$$ \frac{d}{dt}\vec{L} = \sum \tau = -Mgl\sin{\theta} $$
Ricordiamo a questo punto che $\omega = \dot{\theta}, \quad \alpha = \ddot{\theta}$, ovvero possiamo riscrivere
$\frac{d}{dt}\vec{L}$ anche come:
$$\frac{d}{dt} \vec{L} = Ml^2\alpha = -Mgl\sin{\theta} $$
da cui si ottiene l'equazione differenziale:
$$ \ddot{\theta} = -\frac{g}{l}\sin{\theta} $$
Possiamo applicare l'usuale approssimazione attraverso Taylor di $\sin{\theta} = \theta$ per valori piccoli di $\theta$,
porre $\omega = \sqrt{\frac{g}{l}}$, $T=\frac{2\pi}{\omega} = 2\pi\sqrt{\frac{l}{g}}$, ed ottenere il risultato:
$$ \ddot{\theta} = -\omega^2 \theta \Rightarrow \theta = A\cos(\omega t+ \phi)$$
Che è coerente con quanto già dimostrato riguardo al pendolo semplice in moto piano. Ovviamente $A$ e $\phi$ dipenderanno
dalle condizioni iniziali dell'eq. differenziale.
\par\smallskip
\textbf{Applicazione dell'energia al pendolo} \\
Tanto per restare in tema, vediamo come il moto del pendolo può essere agilmente spiegato anche attraverso la conservazione dell'energia.
Consideriamo il lavoro delle forze: sono tutte conservative, in quanto non c'è attrito, e in particolare $\vec{T}$ e la gravìtà $mg$ nella direzione
opposta al filo non compiono alcun lavoro. Possiamo impostare la l'energia, notando che l'altezza da terra $h$ della pallina è data
da $h = 1 - l\cos{\theta}$ e la velocità tangenziale $v$ è $v = \dot{\theta} l$:
$$ E = U + K = Mgh + \frac{1}{2}Mv^2 = Mg(1 - l\cos{\theta}) + \frac{1}{2}M\dot{\theta}^2l^2 $$
Dalla conservazione dell'energia, avremo che:
$$ \frac{d}{dt}E = 0 = 0 -Mgl\sin{\theta}\cdot{\theta} + \frac{2M\dot{\theta}\ddot{\theta}l^2}{2} = -g\sin{\theta} + \ddot{\theta}$$
da cui si ricava sempre la solita equazione differenziale:
$$ \ddot{\theta} = -\frac{g}{l}\sin{\theta} $$
\par\smallskip
\textbf{Coppia di forze} \\
Osserviamo come due forze con direzione uguale e verso opposto non generano necessariamente momento nullo.
Potremo infatti avere:
$$ F_{1_est} + F_{2_est} = 0, \quad \sum \tau_{est} = \vec{R_1} \times \vec{F_{1_est}} + \vec{R_2} \times \vec{F_{2_est}} = -\frac{d}{2}\vec{F_z} -\frac{d}{2}\vec{F_z} = -d\vec{F_z} \neq 0 $$
Questo è il caso di una coppia di forze, applicate su assi sfasati, oppure ad un certo angolo fra di loro, su una certa distanza
$d$ lungo l'oggetto (equivalente a due volte il braccio). Una coppia di forze genera sul corpo una rotazione, ma non una traslazione. Per quanto
riguarda le forze interne, vediamo che condividono la retta di applicazione (un braccio passante per il polo), ergo:
$$ \sum \tau_{int} = 0 $$
\section{Dinamica del corpo rigido}
Possiamo a questo punto riportare le formule trovate finora sul corpo rigido. Ricordiamo che un corpo rigido non è
altro che un sistema di punti materiali le cui distanze l'uno dall'altro non cambiano. Una conseguenza di questa limitazione
è che gli unici due moti possibili di un corpo rigido sono la pura traslazione (del centro di massa) e la pura rotazione attorno ad un asse. Abbiamo
le leggi:
\begin{itemize}
  \item \textbf{Prima equazione cardinale}
$$ \frac{d\vec{P_{CM}}}{dt} = M\vec{A_{CM}} = \vec{R}_{est} $$
  \item \textbf{Seconda equazione cardinale}
$$ \tau_{est} = \sum \vec{R} \times \vec{F_{est}} = \frac{d\vec{L}}{dt}$$
\end{itemize}

\end{document}
