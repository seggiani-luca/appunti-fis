\documentclass[a4paper,12pt]{article}

\usepackage[french,italian]{babel}
\usepackage[T1]{fontenc}
\usepackage[utf8]{inputenc}
\frenchspacing 
\title{Appunti Fisica I}
\author{Luca Seggiani}
\date{24 Maggio 2024}

\begin{document}
\maketitle
\par\smallskip
\textbf{Campi solenoidali} \\
Parliamo brevemente dal campo magnetico. Si ha che la divergenza del campo magnetico è uguale a zero (non esistono monopoli magnetici).
Questo significa che per ogni superficie (non chiusa) nel campo magnetico, la circuitazione sul bordo dipende solamente dalle correnti
concatenate. Campi di questo tipo si dicono campi solenoidali:
$$ \nabla \cdot \vec{B} = 0 $$
\par\smallskip
\textbf{Campo generato dal solenoide con Ampere} \\
Si può usare la legge di ampere per calcolare il campo magnetico generato da un solenoide. Si ha che:
$$ \vec{B} = \mu_0 I_{conc} $$
Prendiamo allora un solo lato del solenoide, dove tutte le correnti scorrono nello stesso verso. Definiamo
una curva chiusa, detta $\gamma$, che contiene N di queste spire, di altezza $h$ e lunghezza $l$. Avremo
che la circuitazione lungo $\gamma$:
$$ \int_\gamma \vec{B} \cdot d\vec{l} = \vec{B} \cdot l = \mu_0 n I, \quad B = \mu_0 n I$$
che è esattamente quanto avevamo trovato con Biot-Savart. Il campo dipende solamente da $l$ (nemmeno da $2l$) in quanto
l'unica parte della linea $\gamma$ immersa nel campo, tutto interno al solenoide, è quella di lunghezza $l$.
Notiamo infine che questo campo, idealmente, è tutto interno (il solenoide, tendendo in lunghezza ad infinito,
"scherma" il campo magnetico interno), ergo si può dire che l'espressione appena trovata rappresenta, per un solenoide
disposto sull'asse x:
$$ \mu_0nI \hat{x} = \vec{B}(y,z)  \theta(R_|\sqrt{y^2+z^2}|)$$
Dove $\theta$ è la funzione gradino di Heaviside (qui serve solo ad annullare il campo fuori dal solenoide).
\par\smallskip
\textbf{Campo generato da un solenoide toroidale} \\
Poniamo il caso di un solenoide formato da una filo avvolto attorno a un anima toroidale. Avremo che il campo magnetico è interno
alle spire e circolare, ergo:
$$ \vec{B} = B(r)\hat{\theta} $$
Considerando i versori $\hat{\theta}$ e $\hat{r}$ tangenziali e radiali alla circonferenza. Prendiamo, come prima, un percorso
chiuso $\gamma$: ad esempio una linea chiusa contenuta all'interno delle spire. Avremo che la circuitazione lungo tale curva è:
$$ \int_\gamma \vec{B} \cdot d\vec{l} = 2\pi r B = \mu_0 I N, \quad B = \mu_0I\frac{N}{2\pi R} $$
Possiamo seguire la forma del campo generato da un solenoide rettilineo definendo una certa "densità radiale di spire" $n' = \frac{N}{2\pi r}$:
$$ \vec{B} = \mu_0 n' I $$
\section{Legge di Faraday}
Finora abbiamo visto come le correnti elettriche generano campi magnetici. Vedremo adesso che è vero, in qualche
modo anche il contrario: una variazione del flusso magnetico genera corrente.
\par\smallskip
\textbf{Flusso magnetico} \\
Avevamo visto la relazione fra il flusso del campo magnetico e la sua circuitazione stabilita dalla legge di Ampere, ovvero
come la circuitazione del campo magnetico dipende dalle correnti concatenate al percorso chiuso scelto. Questo significa
che il flusso del campo su una superficie chiusa è nullo ($\int_\Sigma \vec{B} \cdot d\vec{A} = 0$ su $\Sigma$ chiusa).
Questo però non significa che sia nullo il flusso su una qualsiasi superficie. In particolare, potremo dire riguardo
a superfici piane di area $A$ immerse in campi (almeno localmente) uniformi:
$$ \int \vec{B} \cdot d\vec{A} = BA\cos{\theta} $$
dove $\theta$ è l'angolo fra la normale della superficie $A$ e la direzione del campo $B$. L'unita di misura del flusso nel sistema
S.I. è il Weber (Wb), definito come $ 1 \mathrm{Wb} = \mathrm{T} \cdot \mathrm{m}^2 $.
\par\smallskip
\textbf{Legge dell'induzione di Faraday} \\
Su un qualsiasi circuito chiuso, si ha che viene indotta una forza elettromotrice all'opposto della variazione del flusso all'interno
della superficie sottesa al circuito:
$$ \mathcal{E} = - \frac{d\Phi_B}{dt} $$
Questa $\mathcal{E}$, a sua volta, genera una corrente all'interno del circuito. Il segno negativo è dato da una conseguenza
della conservazione dell'energia: il verso della corrente corrisponde alla generazione di un campo magnetico che si oppone alla variazione
di flusso che induce la corrente stessa. Questo risultato prende il nome di \textbf{legge di Lenz}. La legge di Faraday, riguardo alla generazione
di una forza elettromotrice, può essere applicato anche a circuiti non chiusi, e anzi vale su qualsiasi regione dello spazio
delimitata da una linea chiusa. Chiaramente, si sperimenta una corrente solo nel caso il circuito sia chiuso.
\par\smallskip
\textbf{Generatori e motori elettrici} \\
Il principio dell'induzione elettrica può essere usato per costruire generatori e motori elettrici. Un semplice generatore
può essere costruito da una spira messa in rotazione all'intero di un campo magnetico costante (rotore e statore). La spira rotante è collegata
ad un circuito mediante delle spazzole, che vanno a chiudere un circuito entrando in contatto con un commutatore. La presenza
del commutatore è resa necessaria se si vuole generare corrente continua. Nel caso si voglia generare corrente alternata,
il commutatore sarà sostituito da due collettori rotanti in presa continua con le spazzole, e si parlerà quindi di alternatore.
Vediamo una decrizione quantitativa più dettagliata della fisica dietro il funzionamento di generatori e alternatori:
\begin{itemize}
  \item \textbf{Generatore di corrente continua} \\
    Si ha che il flusso attraverso la spira, dalla definizione di flusso su un campo magnetico uniforme, è:
    $$ \Phi_b = BA\cos{\theta} = BA\cos{\omega t}$$
    Dove la posizione angolare $\theta$ è dato dalla velocità angolare $\omega$ con cui si muove il rotore. A
    questo punto potremo usare la legge di Faraday per calcolare la forza elettromotrice esercitata sulla spira
    dalla variazione di flusso:
    $$ \mathcal{E} = |-N \frac{d}{dt}BA\cos{\omega t}| = NBA\omega |\sin{\omega t}| $$
    dove si è introdotto un termine $N$ per indicare eventuali avvoglimenti multipli della spira (nella realtà
    non si parlerà tanto di spire quanto di bobine), e il valore assoluto per rappresentare il verso
    della corrente forzato dal commutatore. La funzione di $t$ della fem avrà la forma di una sinusoide rettificata.
    Dall'equazione possiamo quindi ricavare il valore massimo della fem:
    $$ \mathcal{E}_{max} = NBA\omega $$
    e sapere che tale valore viene raggiunto per $\omega t = \frac{\pi}{2} + k\pi $, mentre per $\omega t  = k\pi$
    la fem è nulla.
  \item \textbf{Generatore di corrente alternata} (alternatore) \\
    Come prima, si ha flusso:
    $$ \Phi_b = BA\cos{\theta} = BA\cos{\omega t}$$
    E forza elettromotrice:
    $$ \mathcal{E} = -N \frac{d}{dt}BA\cos{\omega t} = NBA\omega \sin{\omega t} $$
    Da questa equazione possiamo, come prima, ricavare il valore massimo della fem:
    $$ \mathcal{E}_{max} = NBA\omega $$
    e sapere che tale valore viene raggiunto per $\omega t = \frac{\pi}{2} + k\pi $, mentre per $\omega t  = k\pi$
    la fem è nulla. Il tipo di corrente generata da un'alternatore si chiama \textbf{corrente alternata}. La corrente
    alternata è caratterizzata dal variare nel tempo della sua direzione, secondo una funzione sinusoidale. La frequenza
    avrà quindi il valore $f = \frac{\omega}{2\pi}$, dove $\omega$ sarà la velocità angolare del generatore (o più in
    generale la pulsazione).
\end{itemize}
Un motore elettrico (o almeno un certo tipo di motore elettrico, detto \textit{brushed} per distinguerlo dai modelli \textit{brushless}, senza spazzole)
è effettivamente un generatore al contrario: invece di applicare un momento che porti il rotore
a ruotare con velocità angolare $\omega$, si dà corrente alla spira in modo che la forza di Lorentz la porti a ruotare.
Si ha che, per la legge di Lenz, la rotazione della spira porta alla formazione di una forza sulla spira che si oppone
alla variazione del flusso magnetico, e quindi della forza elettromotrice che aveva creato la corrente di partenza. Questa
forza prende il nome di \textit{forza controelettromotrice}, e influenza la corrente totale all'interno della spira:
$$ \mathcal{E}_{opp.} = NBA\omega $$
Visto che la corrente all'interno di un circuito dipende dalla sua resistenza $R$, e che la resistenza totale all'interno di un motore
\textit{in rotazione} è aumentata dalla presenza della forza controelettromotrice, si deve allora fornire al motore una forza elettromotrice
maggiore di quella che sarebbe bastata a creare la stessa corrente senza tale effetto.
Si noti che, si dovesse fermare il motore, la corrente potrebbe crescere a livelli inaspettati senza più l'effetto della forza controelettromotrice.
Questo è il motivo per cui un motore elettrico bloccato tende a surriscaldarsi.
\end{document}

