\documentclass[a4paper,12pt]{article}

\usepackage[french,italian]{babel}
\usepackage[T1]{fontenc}
\usepackage[utf8]{inputenc}
\frenchspacing 
\title{Appunti Fisica I}
\author{Luca Seggiani}
\date{18 Aprile 2024}

\usepackage{hyperref}

\begin{document}
\maketitle
\section{Momento di inerzia rispetto ad un asse}
Consideriamo una massa $M$ in rotazione a distanza $R$ dal suo asse. Potremo calcolare il suo momento di inerzia come:
$$ I_0 = \sum m_ir_i^2 = MR^2$$
\par\smallskip
\textbf{Momento di inerzia di una sbarretta sottile}\\
Calcoliamo il momento di inerzia di una sbarretta sottile di massa $M$ e lunghezza $L$ in rotazione rispetto al suo punto medio.
Dovremo conoscere la densità lineare di massa $\lambda$ della sbarretta:
$$ \lambda = \frac{M}{L} = \frac{dm}{dx}, \quad dm = \lambda dx $$
Potremo allora calcolare il momento di inerzia applicando la definizione:
$$ I= dm d^2, \quad I = \int_{\frac{L}{2}}^{\frac{L}{2}} \lambda dx x^2 = \frac{2\lambda}{3} \left(\frac{L}{2}\right)^3 = \frac{ML^2}{12} $$
\par\smallskip
\textbf{Momento di inerzia di un anello omogeneo} \\
Calcoliamo il momento di inerzia di un'anello di massa $M$ e raggio $R$, vuoto, in rotazinoe rispetto al suo centro. Ragioniamo in
modo analogo a prima. La densità lineare di massa sarà:
$$ \lambda = \frac{M}{2\pi R} $$
mentre la variazione di lunghezza sull'anello $dl$ sarà:
$$ dm = \lambda dl, \quad dl = Rd\phi $$
da cui possiamo quindi calcolare la massa su $dl$:
$$ dm = \lambda dl = \frac{M}{2\pi R}Rd\phi = \frac{M}{2\pi}d\phi$$
Impostiamo allora il momento di inerzia dalla definizione:
$$ I = \int_{anello} R^2 dm = R^2\int_0^{2\pi} \frac{M}{2\pi}d\phi $$
Notiamo che l'integrale rappresenta semplicemente tutta la massa dell'anello. Avremo quindi che, come prima:
$$ I = MR^2 $$
\par\smallskip
\textbf{Momento di inerzia di un disco omogeneo} \\
Studiamo il caso analogo al precedente, ma dove l'anello è pieno. Avremo bisogno della densità superficiale
di massa:
$$ \rho = \frac{dm}{ds} = \frac{M}{\pi R^2}, \quad dm = \rho ds, \quad ds = 2\pi r dr $$
Calcoliamo allora il momento d'inerzia:
$$ I = \int dm d^2 = \int \rho ds r^2 = \rho \int 2\pi rdrr^2 = \rho 2\pi \frac{R^4}{M} = M \frac{R^2}{2} $$
\par\smallskip
\textbf{Cubo di masse puntiformi} \\
Calcoliamo il centro di massa di un cubo formato da un punto materiale per ogni suo vertice. Possiamo iniziare
sull'asse x: 4 delle masse avranno coordinata $x=0$, e le altre 4 avranno coordinata $x=l$ con $l$ lato del cubo.
Questo mi porta a dire:
$$ x_{CM} = \frac{(4x_l + 4x_0)m}{8m} = \frac{l}{2} $$
Possiamo ripetere quest'operazione su tutti e tre gli assi, ottenendo:
$$ \vec{r_{CM}} = \left(\frac{l}{2}, \frac{l}{2}, \frac{l}{2}\right) $$
Com'è abbastanza intuitivo. \\
Vediamo ora di calcolare il momento di inerzia dello stesso corpo. Sull'asse x si avrà che:
$$ I_x = \sum m_i (y_i^2 + z_i^2) $$
Se abbiamo centrato la figura facendo corrispondere il centro di massa all'origine degli assi cartesiani, è vero che:
$$ x = \pm \frac{l}{2}, \quad y = \pm \frac{l}{2}, \quad z = \pm \frac{l}{2} $$
Potrò sostituire queste coordinate nella formula precedente ottenendo:
$$ I_x = 8m\left(\frac{l^2}{2}\right) = 4ml^2, \quad \vec{I_{CM}} = 4ml^2 $$
\par\smallskip
\textbf{Cubo omogeneo} \\
Rifacciamo quanto appena fatto per un cubo omogeneo di massa $M$. Per quanto riguarda il centro di massa, avremo
che il baricentro è dato da:
$$ \vec{r_{CM}} = \frac{1}{M} \int dm \vec{r} $$
Con la densità volumetrica di massa: 
$$ dm = \rho dv, \quad \rho = \frac{dm}{dv} = \frac{M}{l^3} $$
Possiamo quindi impostare l'integrale:
$$ \vec{x_{CM}} = \frac{\rho}{M} \int_0^l x dx \int_0^l dy \int_0^l dz = \frac{\rho}{M}\frac{l^4}{2} = \frac{l}{2} $$
Che è identico al caso discreto. \\ Calcoliamo allora il momento di inerzia:
$$ I_x = \int dm(y^2+z^2) $$
è perfettamente analogo al caso precedente, solo considerando variazioni continue di massa. Si riprende la
densità volumetrica di massa:
$$ dm = \rho dv, \quad \rho = \frac{dm}{dv} = \frac{M}{l^3} $$
e si imposta l'integrale:
$$ I_x = \rho\int dxdydz(y^2+z^2) = \rho\int_{\frac{l}{2}}^{\frac{l}{2}} dx \int_{\frac{l}{2}}^{\frac{l}{2}} dy \int_{\frac{l}{2}}^{\frac{l}{2}} (y^2+z^2)dz = \rho l \int_{\frac{l}{2}}^{\frac{l}{2}} (lx^2 + \frac{y^3}{3}\Bigg|^{\frac{l}{2}}_{\frac{l}{2}})dx = [...] $$
che dà il risultato finale di:
$$ I = \frac{Ml^2}{6}$$
\par\smallskip
\textbf{Teorema degli assi paralleli} \\
Il teorema degli assi paralleli, anche noto come teorema di Steiner, ci permette di calcolare in modo più agevole il momento d'inerzia
di un corpo. Supponiamo di aver calcolato il momento di inerzia di un corpo rispetto al suo centro di massa. Poniamo però che il corpo
stia adesso ruotando, ma non attorno al suo centro di massa, bensì attorno ad un asse che è parallelo a quello da cui eravamo partiti e sta
da esso ad una certa distanza $D$. Possiamo allora dire che:
$$ I' = I_{CM} + MD^2$$
Dimostriamo questo risultato. Se il momento d'inerzia era stato calcolato ad una distanza $d'$ da $dm$, allora:
$$ \vec{a'} = \vec{D} + \vec{d} $$
Da cui si imposta l'integrale:
$$ I' = \int dmd'^2 = \int dm(\vec{D} + \vec{d})\cdot(\vec{D} + \vec{d}) = \int dm(D^2 + 2\vec{d}\cdot\vec{D}+d^2)  $$
$$ = D^2\int dm + 2\vec{D} \cdot \int dm\vec{d} + \int dmd^2$$
Notiamo che:
$$2\vec{D}\cdot\int dm\vec{d} = 2D_x\int dmx + 2D_y\int dm y = 0$$
L'integrale allora potrà ridursi a:
$$ I' = D^2 \int dm + \int dm(x^2+y^2) = MD^2 + I_{CM}$$
Vediamo un applicazione.
\par\smallskip
\textbf{Applicazione del teorema di Steiner} \\
Supponiamo di avere un manubrio formato da due sfere di massa $M$ e raggio $R$ unite fra di loro da un asta lunga $L$ e di massa $m$. Cerchiamo
di calcolare il momento di inerzia del corpo rispetto a un asse passante per il centro dell'asta.
Innanzituttto potremo calcolare il momento di inerzia dell'asta, rispetto all'asse che passa attravero il suo centro:
$$ I_{asta} = I_{asta}^{CM} = \frac{mL^2}{12} $$
Si calcolerà poi il momento di inerzia di una sfera, applicando il teorema di Steiner:
$$ I_{sfera} = I_{sfera}^{CM} + M(R+\frac{L}{2})^2 = \frac{2}{5}MR^2 + M(R+ \frac{L}{2})^2 $$
Il momento di inerzia totale a questo punto sarà dato da:
$$ I_{tot} = I_{asta} + 2I_{sfera} = \frac{mL^2}{12} + 2(\frac{2}{5}MR^2 + M(R+ \frac{L}{2})^2)$$
\section{Momento angolare di un corpo rigido attorno ad un asse}
Vediamo come descrivere il momento angolare di un corpo rigido formato da una distribuzione di massa continua di punti materiali.
Avremo che, se il corpo ruota attorno ad un asse con $\omega$ velocità angolare costante, allora ogni suo punto ruoterà
attorno a tale asse con velocità angolare costante, descrivendo una circonferenza ortogonale all'asse e di raggio tale alla 
distanza del punto dall'asse. Possiamo allora applicare la definizione di momento angolare:
$$ \vec{L} = \sum m_i\vec{R_i} \times \vec{v_i} $$
Ricordiamo allora che $\vec{v_i} = \vec{\omega} \times \vec{R_i}$ per quanto riguarda il moto circolare uniforme. Allora:
$$ \vec{L}=\sum m_i\vec{R_i} \times (\vec{\omega} \times \vec{R_i})$$
Questo può essere risolto applicando il prodotto vettoriale triplo (niente baci sui taxi!):
$$ \vec{a} \times (\vec{b} \times \vec{c}) = \vec{b} \times (\vec{a} \cdot \vec{c}) - \vec{c} \times (\vec{a} \cdot \vec{b}) $$
da cui:
$$ \vec{L} = \sum m_i (R_i^2 \vec{\omega} - (\vec{R_i}\cdot\vec{\omega})\vec{R_i}) = \sum m_i (R_i^2 \vec{\omega} - \omega R_i \vec{R_i} \cos{\theta_i}) $$
A questo punto conviene scomporre $\vec{R_i}$ in due componenti (e da qui la grande intuizione): una parallela alla velocità angolare $\vec{\omega}$ ($\hat{\omega}$), e una ortogonale
ad essa ($\hat{u}$):
$$ \vec{R_i} = \hat{\omega}R_i\cos{\theta_i} + \hat{u}R_i\sin{\theta_i} $$
Questo ci porta a riscrivere $\vec{L}$ in due componenti, quella lungo $\hat{\omega}$ e quella lungo $\hat{u}$. Dai calcoli, si avrà che la componente lungo $\hat{\omega}$
(detta componente assiale) varrà:
$$ \vec{L_z} = \sum m_i d_i^2 \omega= I_z\omega $$
Ovvero momento di inerzia per velocità angolare, come si era già trovato. Notiamo che siamo passati da $R_i$ a $d_i$, ovvero $R_i\sin{\theta_i}$, la distanza del punto dall'asse di rotazione del corpo.
\\Particolare sarà invece la componente ortogonale $\hat{u}$, che però si osserverà
solo nel caso l'asse di rotazione non sia coincidente con uno degli assi di simmetria della distribuzione di massa del corpo rigido:
$$ L_u = \sum m_i\omega R_i^2 \sin{\theta_i}\cos{\theta_i} $$
Lo svolgimento completo dei calcoli si può trivare a pagina 68 di:
\url{https://github.com/Guray00/IngegneriaInformatica/blob/master/PRIMO%20ANNO/II%20SEMESTRE/Fisica%20Generale%20I/Dispense%20(A.A.%20corrente)/Lezione11-Moto_di_un_Corpo_RIgido.pdf}
\end{document}
