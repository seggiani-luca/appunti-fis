\documentclass[a4paper,12pt]{article}

\usepackage[french,italian]{babel}
\usepackage[T1]{fontenc}
\usepackage[utf8]{inputenc}
\frenchspacing 
\title{Appunti Fisica I}
\author{Luca Seggiani}
\date{16 Aprile 2024}

\usepackage{amsmath}

\begin{document}
\maketitle
\section{Energia potenziale gravitazionale per sistemi di punti materiali}
Possiamo dare una definizione di energia potenziale gravitazionale per sistemi di punti materiali, e quindi corpi rigidi,
tale per cui l'altezza $h$ in $mgh$ (come avevamo trovato per i punti materiali) vale quanto l'altezza del corpo rigido, ovvero:
$$ U_{grav} = \sum_i^n m_igz_i = Mg \frac{\sum_i^f m_iz_i}{M} = Mgz_{CM}$$
notando che:
$$ \frac{\sum_i^f m_iz_i}{M} = \frac{1}{M}\int zdm = z_{CM}$$
\section{Teorema di König per i sistemi di punti materiali}
Dimostriamo adesso un'importante risultato riguardante l'energia cinetica.
Stiamo considerando un sistema di punti materiali: definiamo due sistemi di riferimento inerziali. Il primo è
il sistema di laboratorio $S_0(x,y,z)$, e il secondo è il sistema del centro di massa $S_{CM}(x',y',z')$ con assi
paralleli a $S_0$ la cui origine è solidale al centro di massa. Vale la relazione:
$$ 
\left\{\begin{array}{l}
    \vec{r_i} = \vec{r_i} + \vec{v_i'}t \cr \\ 
    \vec{v_i} = \vec{v_{CM}} + \vec{v_i'} \cr \\
    \vec{a_i} = \vec{a_i'}
\end{array}\right.
$$
possiamo ricavare sulla base di questa relazione una formula per l'energia cinetica:
$$ K = \sum \frac{1}{2}m_1v_1^2 = \sum \frac{1}{2} m_i(\vec{v_{CM}} + \vec{v_i'})^2 = \sum \frac{1}{2} m_i\vec{v_{CM}}^2  + \sum \frac{1}{2}m_i\vec{v_i'}^2 + \vec{v_{CM}} \cdot \sum m_i\vec{v_i'}$$
Notiamo a questo punto che il termine $\sum m_i\vec{v_i'}$ equivale alla quantità di moto complessiva nel sistema di riferimento
del centro di massa, che è nulla. Possiamo quindi riscrivere il tutto come:
$$ K = \frac{1}{2}M\vec{v_{CM}}^2 + \sum \frac{1}{2}m_i\vec{v_i'}^2 $$
Questa formula rappresenta ciò che viene enunciato dal teorema di König:
L'energia cinetica totale di un sistema di punti materiali coincide con la somma dell'energia cinetica
di traslazione del centro di massa (ovvero l'energia cinetica che avrebbe un corpo di massa uguale alla massa totale
dei punti materiali e velocità pari a quella del centro di massa)
e dell'energia cinetica di tutti i punti materiali misurata in un sistema
di riferimento con origine solidale al centro di massa e assi paralleli a quelli di laboratorio. L'energia cinetica
del centro di massa è data da $\frac{1}{2}M\vec{v_{CM}}^2$, e quella dei punti materiali nel sistema di riferimento proprio da
$ \sum \frac{1}{2}m_i\vec{v_i'}^2 $, da cui la formula sopra riportata.
\section{Teorema di König per il corpo rigido}
Vediamo il significato del teorema di König per i corpi rigidi. Ripartiamo dalla definizione data per i sistemi
di punti materiali:
$$ K = \frac{1}{2}M\vec{v_{CM}}^2 + \sum \frac{1}{2}m_i\vec{v_i'}^2 $$
Si avrà che il primo termine, riguardante il centro di massa, avrà esattamente la stessa forma per il corpo rigido.
A cambiare sarà il secondo termine, riguardante i moti dei punti materiali rispetto al sistema di riferimento centrato
sul centro di massa del sistema: nel caso del corpo rigido, dove le distanze relative fra punti materiali sono costanti,
avremo che tale fattore dipende dal moto rotazionale del corpo stesso. Per la precisione, avremo che per qualsiasi
punto a distanza $d_i$ dal centro di massa, la velocità relativa $\vec{v_i'}$ sarà:
$$ \vec{v_i'} = \omega d_i \Rightarrow K = \frac{1}{2}M\vec{v_{CM}}^2 + \frac{1}{2}\sum m_id_i^2\omega^2$$
Definiamo la quantità $I_{CM}$, detta momento d'inerzia. 
$$ I_{CM} = \sum_i^n m_id_i^2 $$
Il momento di inerzia andrà inteso come il "corrispettivo rotazionale della massa", anche se è importante notare che, a differenza della massa, dipende dall'asse di rotazione scelto. In ogni caso,
possiamo ricavare la formula definitiva:
$$ K = \frac{1}{2}M\vec{v_{CM}}^2 + \frac{1}{2}I_{CM}\omega^2 $$
dove il primo termine dipende, come già detto, dal moto traslatorio del centro di massa, e il secondo termine dal moto rotazionale del corpo rigido e dal suo momento di inerzia.
\end{document}
